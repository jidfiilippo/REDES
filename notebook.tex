
% Default to the notebook output style

    


% Inherit from the specified cell style.




    
\documentclass[11pt]{article}

    
    
    \usepackage[T1]{fontenc}
    % Nicer default font (+ math font) than Computer Modern for most use cases
    \usepackage{mathpazo}

    % Basic figure setup, for now with no caption control since it's done
    % automatically by Pandoc (which extracts ![](path) syntax from Markdown).
    \usepackage{graphicx}
    % We will generate all images so they have a width \maxwidth. This means
    % that they will get their normal width if they fit onto the page, but
    % are scaled down if they would overflow the margins.
    \makeatletter
    \def\maxwidth{\ifdim\Gin@nat@width>\linewidth\linewidth
    \else\Gin@nat@width\fi}
    \makeatother
    \let\Oldincludegraphics\includegraphics
    % Set max figure width to be 80% of text width, for now hardcoded.
    \renewcommand{\includegraphics}[1]{\Oldincludegraphics[width=.8\maxwidth]{#1}}
    % Ensure that by default, figures have no caption (until we provide a
    % proper Figure object with a Caption API and a way to capture that
    % in the conversion process - todo).
    \usepackage{caption}
    \DeclareCaptionLabelFormat{nolabel}{}
    \captionsetup{labelformat=nolabel}

    \usepackage{adjustbox} % Used to constrain images to a maximum size 
    \usepackage{xcolor} % Allow colors to be defined
    \usepackage{enumerate} % Needed for markdown enumerations to work
    \usepackage{geometry} % Used to adjust the document margins
    \usepackage{amsmath} % Equations
    \usepackage{amssymb} % Equations
    \usepackage{textcomp} % defines textquotesingle
    % Hack from http://tex.stackexchange.com/a/47451/13684:
    \AtBeginDocument{%
        \def\PYZsq{\textquotesingle}% Upright quotes in Pygmentized code
    }
    \usepackage{upquote} % Upright quotes for verbatim code
    \usepackage{eurosym} % defines \euro
    \usepackage[mathletters]{ucs} % Extended unicode (utf-8) support
    \usepackage[utf8x]{inputenc} % Allow utf-8 characters in the tex document
    \usepackage{fancyvrb} % verbatim replacement that allows latex
    \usepackage{grffile} % extends the file name processing of package graphics 
                         % to support a larger range 
    % The hyperref package gives us a pdf with properly built
    % internal navigation ('pdf bookmarks' for the table of contents,
    % internal cross-reference links, web links for URLs, etc.)
    \usepackage{hyperref}
    \usepackage{longtable} % longtable support required by pandoc >1.10
    \usepackage{booktabs}  % table support for pandoc > 1.12.2
    \usepackage[inline]{enumitem} % IRkernel/repr support (it uses the enumerate* environment)
    \usepackage[normalem]{ulem} % ulem is needed to support strikethroughs (\sout)
                                % normalem makes italics be italics, not underlines
    

    
    
    % Colors for the hyperref package
    \definecolor{urlcolor}{rgb}{0,.145,.698}
    \definecolor{linkcolor}{rgb}{.71,0.21,0.01}
    \definecolor{citecolor}{rgb}{.12,.54,.11}

    % ANSI colors
    \definecolor{ansi-black}{HTML}{3E424D}
    \definecolor{ansi-black-intense}{HTML}{282C36}
    \definecolor{ansi-red}{HTML}{E75C58}
    \definecolor{ansi-red-intense}{HTML}{B22B31}
    \definecolor{ansi-green}{HTML}{00A250}
    \definecolor{ansi-green-intense}{HTML}{007427}
    \definecolor{ansi-yellow}{HTML}{DDB62B}
    \definecolor{ansi-yellow-intense}{HTML}{B27D12}
    \definecolor{ansi-blue}{HTML}{208FFB}
    \definecolor{ansi-blue-intense}{HTML}{0065CA}
    \definecolor{ansi-magenta}{HTML}{D160C4}
    \definecolor{ansi-magenta-intense}{HTML}{A03196}
    \definecolor{ansi-cyan}{HTML}{60C6C8}
    \definecolor{ansi-cyan-intense}{HTML}{258F8F}
    \definecolor{ansi-white}{HTML}{C5C1B4}
    \definecolor{ansi-white-intense}{HTML}{A1A6B2}

    % commands and environments needed by pandoc snippets
    % extracted from the output of `pandoc -s`
    \providecommand{\tightlist}{%
      \setlength{\itemsep}{0pt}\setlength{\parskip}{0pt}}
    \DefineVerbatimEnvironment{Highlighting}{Verbatim}{commandchars=\\\{\}}
    % Add ',fontsize=\small' for more characters per line
    \newenvironment{Shaded}{}{}
    \newcommand{\KeywordTok}[1]{\textcolor[rgb]{0.00,0.44,0.13}{\textbf{{#1}}}}
    \newcommand{\DataTypeTok}[1]{\textcolor[rgb]{0.56,0.13,0.00}{{#1}}}
    \newcommand{\DecValTok}[1]{\textcolor[rgb]{0.25,0.63,0.44}{{#1}}}
    \newcommand{\BaseNTok}[1]{\textcolor[rgb]{0.25,0.63,0.44}{{#1}}}
    \newcommand{\FloatTok}[1]{\textcolor[rgb]{0.25,0.63,0.44}{{#1}}}
    \newcommand{\CharTok}[1]{\textcolor[rgb]{0.25,0.44,0.63}{{#1}}}
    \newcommand{\StringTok}[1]{\textcolor[rgb]{0.25,0.44,0.63}{{#1}}}
    \newcommand{\CommentTok}[1]{\textcolor[rgb]{0.38,0.63,0.69}{\textit{{#1}}}}
    \newcommand{\OtherTok}[1]{\textcolor[rgb]{0.00,0.44,0.13}{{#1}}}
    \newcommand{\AlertTok}[1]{\textcolor[rgb]{1.00,0.00,0.00}{\textbf{{#1}}}}
    \newcommand{\FunctionTok}[1]{\textcolor[rgb]{0.02,0.16,0.49}{{#1}}}
    \newcommand{\RegionMarkerTok}[1]{{#1}}
    \newcommand{\ErrorTok}[1]{\textcolor[rgb]{1.00,0.00,0.00}{\textbf{{#1}}}}
    \newcommand{\NormalTok}[1]{{#1}}
    
    % Additional commands for more recent versions of Pandoc
    \newcommand{\ConstantTok}[1]{\textcolor[rgb]{0.53,0.00,0.00}{{#1}}}
    \newcommand{\SpecialCharTok}[1]{\textcolor[rgb]{0.25,0.44,0.63}{{#1}}}
    \newcommand{\VerbatimStringTok}[1]{\textcolor[rgb]{0.25,0.44,0.63}{{#1}}}
    \newcommand{\SpecialStringTok}[1]{\textcolor[rgb]{0.73,0.40,0.53}{{#1}}}
    \newcommand{\ImportTok}[1]{{#1}}
    \newcommand{\DocumentationTok}[1]{\textcolor[rgb]{0.73,0.13,0.13}{\textit{{#1}}}}
    \newcommand{\AnnotationTok}[1]{\textcolor[rgb]{0.38,0.63,0.69}{\textbf{\textit{{#1}}}}}
    \newcommand{\CommentVarTok}[1]{\textcolor[rgb]{0.38,0.63,0.69}{\textbf{\textit{{#1}}}}}
    \newcommand{\VariableTok}[1]{\textcolor[rgb]{0.10,0.09,0.49}{{#1}}}
    \newcommand{\ControlFlowTok}[1]{\textcolor[rgb]{0.00,0.44,0.13}{\textbf{{#1}}}}
    \newcommand{\OperatorTok}[1]{\textcolor[rgb]{0.40,0.40,0.40}{{#1}}}
    \newcommand{\BuiltInTok}[1]{{#1}}
    \newcommand{\ExtensionTok}[1]{{#1}}
    \newcommand{\PreprocessorTok}[1]{\textcolor[rgb]{0.74,0.48,0.00}{{#1}}}
    \newcommand{\AttributeTok}[1]{\textcolor[rgb]{0.49,0.56,0.16}{{#1}}}
    \newcommand{\InformationTok}[1]{\textcolor[rgb]{0.38,0.63,0.69}{\textbf{\textit{{#1}}}}}
    \newcommand{\WarningTok}[1]{\textcolor[rgb]{0.38,0.63,0.69}{\textbf{\textit{{#1}}}}}
    
    
    % Define a nice break command that doesn't care if a line doesn't already
    % exist.
    \def\br{\hspace*{\fill} \\* }
    % Math Jax compatability definitions
    \def\gt{>}
    \def\lt{<}
    % Document parameters
    \title{TP1\_EJ2}
    
    
    

    % Pygments definitions
    
\makeatletter
\def\PY@reset{\let\PY@it=\relax \let\PY@bf=\relax%
    \let\PY@ul=\relax \let\PY@tc=\relax%
    \let\PY@bc=\relax \let\PY@ff=\relax}
\def\PY@tok#1{\csname PY@tok@#1\endcsname}
\def\PY@toks#1+{\ifx\relax#1\empty\else%
    \PY@tok{#1}\expandafter\PY@toks\fi}
\def\PY@do#1{\PY@bc{\PY@tc{\PY@ul{%
    \PY@it{\PY@bf{\PY@ff{#1}}}}}}}
\def\PY#1#2{\PY@reset\PY@toks#1+\relax+\PY@do{#2}}

\expandafter\def\csname PY@tok@w\endcsname{\def\PY@tc##1{\textcolor[rgb]{0.73,0.73,0.73}{##1}}}
\expandafter\def\csname PY@tok@c\endcsname{\let\PY@it=\textit\def\PY@tc##1{\textcolor[rgb]{0.25,0.50,0.50}{##1}}}
\expandafter\def\csname PY@tok@cp\endcsname{\def\PY@tc##1{\textcolor[rgb]{0.74,0.48,0.00}{##1}}}
\expandafter\def\csname PY@tok@k\endcsname{\let\PY@bf=\textbf\def\PY@tc##1{\textcolor[rgb]{0.00,0.50,0.00}{##1}}}
\expandafter\def\csname PY@tok@kp\endcsname{\def\PY@tc##1{\textcolor[rgb]{0.00,0.50,0.00}{##1}}}
\expandafter\def\csname PY@tok@kt\endcsname{\def\PY@tc##1{\textcolor[rgb]{0.69,0.00,0.25}{##1}}}
\expandafter\def\csname PY@tok@o\endcsname{\def\PY@tc##1{\textcolor[rgb]{0.40,0.40,0.40}{##1}}}
\expandafter\def\csname PY@tok@ow\endcsname{\let\PY@bf=\textbf\def\PY@tc##1{\textcolor[rgb]{0.67,0.13,1.00}{##1}}}
\expandafter\def\csname PY@tok@nb\endcsname{\def\PY@tc##1{\textcolor[rgb]{0.00,0.50,0.00}{##1}}}
\expandafter\def\csname PY@tok@nf\endcsname{\def\PY@tc##1{\textcolor[rgb]{0.00,0.00,1.00}{##1}}}
\expandafter\def\csname PY@tok@nc\endcsname{\let\PY@bf=\textbf\def\PY@tc##1{\textcolor[rgb]{0.00,0.00,1.00}{##1}}}
\expandafter\def\csname PY@tok@nn\endcsname{\let\PY@bf=\textbf\def\PY@tc##1{\textcolor[rgb]{0.00,0.00,1.00}{##1}}}
\expandafter\def\csname PY@tok@ne\endcsname{\let\PY@bf=\textbf\def\PY@tc##1{\textcolor[rgb]{0.82,0.25,0.23}{##1}}}
\expandafter\def\csname PY@tok@nv\endcsname{\def\PY@tc##1{\textcolor[rgb]{0.10,0.09,0.49}{##1}}}
\expandafter\def\csname PY@tok@no\endcsname{\def\PY@tc##1{\textcolor[rgb]{0.53,0.00,0.00}{##1}}}
\expandafter\def\csname PY@tok@nl\endcsname{\def\PY@tc##1{\textcolor[rgb]{0.63,0.63,0.00}{##1}}}
\expandafter\def\csname PY@tok@ni\endcsname{\let\PY@bf=\textbf\def\PY@tc##1{\textcolor[rgb]{0.60,0.60,0.60}{##1}}}
\expandafter\def\csname PY@tok@na\endcsname{\def\PY@tc##1{\textcolor[rgb]{0.49,0.56,0.16}{##1}}}
\expandafter\def\csname PY@tok@nt\endcsname{\let\PY@bf=\textbf\def\PY@tc##1{\textcolor[rgb]{0.00,0.50,0.00}{##1}}}
\expandafter\def\csname PY@tok@nd\endcsname{\def\PY@tc##1{\textcolor[rgb]{0.67,0.13,1.00}{##1}}}
\expandafter\def\csname PY@tok@s\endcsname{\def\PY@tc##1{\textcolor[rgb]{0.73,0.13,0.13}{##1}}}
\expandafter\def\csname PY@tok@sd\endcsname{\let\PY@it=\textit\def\PY@tc##1{\textcolor[rgb]{0.73,0.13,0.13}{##1}}}
\expandafter\def\csname PY@tok@si\endcsname{\let\PY@bf=\textbf\def\PY@tc##1{\textcolor[rgb]{0.73,0.40,0.53}{##1}}}
\expandafter\def\csname PY@tok@se\endcsname{\let\PY@bf=\textbf\def\PY@tc##1{\textcolor[rgb]{0.73,0.40,0.13}{##1}}}
\expandafter\def\csname PY@tok@sr\endcsname{\def\PY@tc##1{\textcolor[rgb]{0.73,0.40,0.53}{##1}}}
\expandafter\def\csname PY@tok@ss\endcsname{\def\PY@tc##1{\textcolor[rgb]{0.10,0.09,0.49}{##1}}}
\expandafter\def\csname PY@tok@sx\endcsname{\def\PY@tc##1{\textcolor[rgb]{0.00,0.50,0.00}{##1}}}
\expandafter\def\csname PY@tok@m\endcsname{\def\PY@tc##1{\textcolor[rgb]{0.40,0.40,0.40}{##1}}}
\expandafter\def\csname PY@tok@gh\endcsname{\let\PY@bf=\textbf\def\PY@tc##1{\textcolor[rgb]{0.00,0.00,0.50}{##1}}}
\expandafter\def\csname PY@tok@gu\endcsname{\let\PY@bf=\textbf\def\PY@tc##1{\textcolor[rgb]{0.50,0.00,0.50}{##1}}}
\expandafter\def\csname PY@tok@gd\endcsname{\def\PY@tc##1{\textcolor[rgb]{0.63,0.00,0.00}{##1}}}
\expandafter\def\csname PY@tok@gi\endcsname{\def\PY@tc##1{\textcolor[rgb]{0.00,0.63,0.00}{##1}}}
\expandafter\def\csname PY@tok@gr\endcsname{\def\PY@tc##1{\textcolor[rgb]{1.00,0.00,0.00}{##1}}}
\expandafter\def\csname PY@tok@ge\endcsname{\let\PY@it=\textit}
\expandafter\def\csname PY@tok@gs\endcsname{\let\PY@bf=\textbf}
\expandafter\def\csname PY@tok@gp\endcsname{\let\PY@bf=\textbf\def\PY@tc##1{\textcolor[rgb]{0.00,0.00,0.50}{##1}}}
\expandafter\def\csname PY@tok@go\endcsname{\def\PY@tc##1{\textcolor[rgb]{0.53,0.53,0.53}{##1}}}
\expandafter\def\csname PY@tok@gt\endcsname{\def\PY@tc##1{\textcolor[rgb]{0.00,0.27,0.87}{##1}}}
\expandafter\def\csname PY@tok@err\endcsname{\def\PY@bc##1{\setlength{\fboxsep}{0pt}\fcolorbox[rgb]{1.00,0.00,0.00}{1,1,1}{\strut ##1}}}
\expandafter\def\csname PY@tok@kc\endcsname{\let\PY@bf=\textbf\def\PY@tc##1{\textcolor[rgb]{0.00,0.50,0.00}{##1}}}
\expandafter\def\csname PY@tok@kd\endcsname{\let\PY@bf=\textbf\def\PY@tc##1{\textcolor[rgb]{0.00,0.50,0.00}{##1}}}
\expandafter\def\csname PY@tok@kn\endcsname{\let\PY@bf=\textbf\def\PY@tc##1{\textcolor[rgb]{0.00,0.50,0.00}{##1}}}
\expandafter\def\csname PY@tok@kr\endcsname{\let\PY@bf=\textbf\def\PY@tc##1{\textcolor[rgb]{0.00,0.50,0.00}{##1}}}
\expandafter\def\csname PY@tok@bp\endcsname{\def\PY@tc##1{\textcolor[rgb]{0.00,0.50,0.00}{##1}}}
\expandafter\def\csname PY@tok@fm\endcsname{\def\PY@tc##1{\textcolor[rgb]{0.00,0.00,1.00}{##1}}}
\expandafter\def\csname PY@tok@vc\endcsname{\def\PY@tc##1{\textcolor[rgb]{0.10,0.09,0.49}{##1}}}
\expandafter\def\csname PY@tok@vg\endcsname{\def\PY@tc##1{\textcolor[rgb]{0.10,0.09,0.49}{##1}}}
\expandafter\def\csname PY@tok@vi\endcsname{\def\PY@tc##1{\textcolor[rgb]{0.10,0.09,0.49}{##1}}}
\expandafter\def\csname PY@tok@vm\endcsname{\def\PY@tc##1{\textcolor[rgb]{0.10,0.09,0.49}{##1}}}
\expandafter\def\csname PY@tok@sa\endcsname{\def\PY@tc##1{\textcolor[rgb]{0.73,0.13,0.13}{##1}}}
\expandafter\def\csname PY@tok@sb\endcsname{\def\PY@tc##1{\textcolor[rgb]{0.73,0.13,0.13}{##1}}}
\expandafter\def\csname PY@tok@sc\endcsname{\def\PY@tc##1{\textcolor[rgb]{0.73,0.13,0.13}{##1}}}
\expandafter\def\csname PY@tok@dl\endcsname{\def\PY@tc##1{\textcolor[rgb]{0.73,0.13,0.13}{##1}}}
\expandafter\def\csname PY@tok@s2\endcsname{\def\PY@tc##1{\textcolor[rgb]{0.73,0.13,0.13}{##1}}}
\expandafter\def\csname PY@tok@sh\endcsname{\def\PY@tc##1{\textcolor[rgb]{0.73,0.13,0.13}{##1}}}
\expandafter\def\csname PY@tok@s1\endcsname{\def\PY@tc##1{\textcolor[rgb]{0.73,0.13,0.13}{##1}}}
\expandafter\def\csname PY@tok@mb\endcsname{\def\PY@tc##1{\textcolor[rgb]{0.40,0.40,0.40}{##1}}}
\expandafter\def\csname PY@tok@mf\endcsname{\def\PY@tc##1{\textcolor[rgb]{0.40,0.40,0.40}{##1}}}
\expandafter\def\csname PY@tok@mh\endcsname{\def\PY@tc##1{\textcolor[rgb]{0.40,0.40,0.40}{##1}}}
\expandafter\def\csname PY@tok@mi\endcsname{\def\PY@tc##1{\textcolor[rgb]{0.40,0.40,0.40}{##1}}}
\expandafter\def\csname PY@tok@il\endcsname{\def\PY@tc##1{\textcolor[rgb]{0.40,0.40,0.40}{##1}}}
\expandafter\def\csname PY@tok@mo\endcsname{\def\PY@tc##1{\textcolor[rgb]{0.40,0.40,0.40}{##1}}}
\expandafter\def\csname PY@tok@ch\endcsname{\let\PY@it=\textit\def\PY@tc##1{\textcolor[rgb]{0.25,0.50,0.50}{##1}}}
\expandafter\def\csname PY@tok@cm\endcsname{\let\PY@it=\textit\def\PY@tc##1{\textcolor[rgb]{0.25,0.50,0.50}{##1}}}
\expandafter\def\csname PY@tok@cpf\endcsname{\let\PY@it=\textit\def\PY@tc##1{\textcolor[rgb]{0.25,0.50,0.50}{##1}}}
\expandafter\def\csname PY@tok@c1\endcsname{\let\PY@it=\textit\def\PY@tc##1{\textcolor[rgb]{0.25,0.50,0.50}{##1}}}
\expandafter\def\csname PY@tok@cs\endcsname{\let\PY@it=\textit\def\PY@tc##1{\textcolor[rgb]{0.25,0.50,0.50}{##1}}}

\def\PYZbs{\char`\\}
\def\PYZus{\char`\_}
\def\PYZob{\char`\{}
\def\PYZcb{\char`\}}
\def\PYZca{\char`\^}
\def\PYZam{\char`\&}
\def\PYZlt{\char`\<}
\def\PYZgt{\char`\>}
\def\PYZsh{\char`\#}
\def\PYZpc{\char`\%}
\def\PYZdl{\char`\$}
\def\PYZhy{\char`\-}
\def\PYZsq{\char`\'}
\def\PYZdq{\char`\"}
\def\PYZti{\char`\~}
% for compatibility with earlier versions
\def\PYZat{@}
\def\PYZlb{[}
\def\PYZrb{]}
\makeatother


    % Exact colors from NB
    \definecolor{incolor}{rgb}{0.0, 0.0, 0.5}
    \definecolor{outcolor}{rgb}{0.545, 0.0, 0.0}



    
    % Prevent overflowing lines due to hard-to-break entities
    \sloppy 
    % Setup hyperref package
    \hypersetup{
      breaklinks=true,  % so long urls are correctly broken across lines
      colorlinks=true,
      urlcolor=urlcolor,
      linkcolor=linkcolor,
      citecolor=citecolor,
      }
    % Slightly bigger margins than the latex defaults
    
    \geometry{verbose,tmargin=1in,bmargin=1in,lmargin=1in,rmargin=1in}
    
    

    \begin{document}
    
    
    \maketitle
    
    

    
    \begin{Verbatim}[commandchars=\\\{\}]
{\color{incolor}In [{\color{incolor}1}]:} \PY{c+c1}{\PYZsh{}paquetes }
        \PY{k+kn}{import} \PY{n+nn}{numpy} \PY{k}{as} \PY{n+nn}{np}
        \PY{k+kn}{import} \PY{n+nn}{networkx} \PY{k}{as} \PY{n+nn}{nx}
        \PY{k+kn}{import} \PY{n+nn}{matplotlib}\PY{n+nn}{.}\PY{n+nn}{pylab} \PY{k}{as} \PY{n+nn}{plt}
        \PY{o}{\PYZpc{}}\PY{k}{matplotlib} inline
        \PY{k+kn}{import} \PY{n+nn}{os}
        \PY{k+kn}{from} \PY{n+nn}{random} \PY{k}{import} \PY{n}{shuffle}
\end{Verbatim}


    \begin{Verbatim}[commandchars=\\\{\}]
{\color{incolor}In [{\color{incolor}2}]:} \PY{n}{G} \PY{o}{=} \PY{n}{nx}\PY{o}{.}\PY{n}{read\PYZus{}gml}\PY{p}{(}\PY{l+s+s1}{\PYZsq{}}\PY{l+s+s1}{dolphins.gml}\PY{l+s+s1}{\PYZsq{}}\PY{p}{)}
\end{Verbatim}


    \begin{Verbatim}[commandchars=\\\{\}]
{\color{incolor}In [{\color{incolor}3}]:} \PY{k}{def} \PY{n+nf}{ldata}\PY{p}{(}\PY{n}{archive}\PY{p}{)}\PY{p}{:}
            \PY{n}{f}\PY{o}{=}\PY{n+nb}{open}\PY{p}{(}\PY{n}{archive}\PY{p}{)}
            \PY{n}{data}\PY{o}{=}\PY{p}{[}\PY{p}{]}
            \PY{k}{for} \PY{n}{line} \PY{o+ow}{in} \PY{n}{f}\PY{p}{:}
                \PY{n}{line}\PY{o}{=}\PY{n}{line}\PY{o}{.}\PY{n}{strip}\PY{p}{(}\PY{p}{)}
                \PY{n}{col}\PY{o}{=}\PY{n}{line}\PY{o}{.}\PY{n}{split}\PY{p}{(}\PY{p}{)}
                \PY{n}{data}\PY{o}{.}\PY{n}{append}\PY{p}{(}\PY{n}{col}\PY{p}{)}
            \PY{k}{return} \PY{n}{data}
\end{Verbatim}


    \begin{Verbatim}[commandchars=\\\{\}]
{\color{incolor}In [{\color{incolor}32}]:} \PY{n}{G}
\end{Verbatim}


\begin{Verbatim}[commandchars=\\\{\}]
{\color{outcolor}Out[{\color{outcolor}32}]:} <networkx.classes.graph.Graph at 0x2af5b5f49e8>
\end{Verbatim}
            
    \begin{Verbatim}[commandchars=\\\{\}]
{\color{incolor}In [{\color{incolor}4}]:} \PY{n}{G}\PY{o}{.}\PY{n}{nodes}
\end{Verbatim}


\begin{Verbatim}[commandchars=\\\{\}]
{\color{outcolor}Out[{\color{outcolor}4}]:} NodeView(('Beak', 'Beescratch', 'Bumper', 'CCL', 'Cross', 'DN16', 'DN21', 'DN63', 'Double', 'Feather', 'Fish', 'Five', 'Fork', 'Gallatin', 'Grin', 'Haecksel', 'Hook', 'Jet', 'Jonah', 'Knit', 'Kringel', 'MN105', 'MN23', 'MN60', 'MN83', 'Mus', 'Notch', 'Number1', 'Oscar', 'Patchback', 'PL', 'Quasi', 'Ripplefluke', 'Scabs', 'Shmuddel', 'SMN5', 'SN100', 'SN4', 'SN63', 'SN89', 'SN9', 'SN90', 'SN96', 'Stripes', 'Thumper', 'Topless', 'TR120', 'TR77', 'TR82', 'TR88', 'TR99', 'Trigger', 'TSN103', 'TSN83', 'Upbang', 'Vau', 'Wave', 'Web', 'Whitetip', 'Zap', 'Zig', 'Zipfel'))
\end{Verbatim}
            
    \begin{Verbatim}[commandchars=\\\{\}]
{\color{incolor}In [{\color{incolor}5}]:} \PY{n}{dolphinsGender}\PY{o}{=}\PY{n}{ldata}\PY{p}{(}\PY{l+s+s1}{\PYZsq{}}\PY{l+s+s1}{dolphinsGender.txt}\PY{l+s+s1}{\PYZsq{}}\PY{p}{)}
        \PY{c+c1}{\PYZsh{}dolphinsGender}
\end{Verbatim}


    \begin{Verbatim}[commandchars=\\\{\}]
{\color{incolor}In [{\color{incolor}90}]:} \PY{n+nb}{type}\PY{p}{(}\PY{n}{dolphinsGender}\PY{p}{)}
\end{Verbatim}


\begin{Verbatim}[commandchars=\\\{\}]
{\color{outcolor}Out[{\color{outcolor}90}]:} list
\end{Verbatim}
            
    \begin{Verbatim}[commandchars=\\\{\}]
{\color{incolor}In [{\color{incolor}7}]:} \PY{n}{dict\PYZus{}gender} \PY{o}{=} \PY{p}{\PYZob{}}\PY{n}{dolphin\PYZus{}nombre} \PY{p}{:} \PY{n}{genero} \PY{k}{for} \PY{n}{dolphin\PYZus{}nombre}\PY{p}{,} \PY{n}{genero}  \PY{o+ow}{in} \PY{n}{dolphinsGender}\PY{p}{\PYZcb{}}
\end{Verbatim}


    \begin{Verbatim}[commandchars=\\\{\}]
{\color{incolor}In [{\color{incolor}8}]:} \PY{k}{def} \PY{n+nf}{AssignGender}\PY{p}{(}\PY{n}{G}\PY{p}{,}\PY{n}{dict\PYZus{}gender}\PY{p}{)}\PY{p}{:}
            \PY{k}{for} \PY{n}{n} \PY{o+ow}{in} \PY{n}{G}\PY{o}{.}\PY{n}{nodes}\PY{p}{:}
                \PY{n}{G}\PY{o}{.}\PY{n}{nodes}\PY{p}{[}\PY{n}{n}\PY{p}{]}\PY{p}{[}\PY{l+s+s2}{\PYZdq{}}\PY{l+s+s2}{gender}\PY{l+s+s2}{\PYZdq{}}\PY{p}{]} \PY{o}{=} \PY{n}{dict\PYZus{}gender}\PY{p}{[}\PY{n}{n}\PY{p}{]}
            \PY{k}{return}
\end{Verbatim}


    \begin{Verbatim}[commandchars=\\\{\}]
{\color{incolor}In [{\color{incolor}9}]:} \PY{n}{nx}\PY{o}{.}\PY{n}{get\PYZus{}node\PYZus{}attributes}\PY{p}{(}\PY{n}{G}\PY{p}{,}\PY{l+s+s1}{\PYZsq{}}\PY{l+s+s1}{gender}\PY{l+s+s1}{\PYZsq{}}\PY{p}{)}
\end{Verbatim}


\begin{Verbatim}[commandchars=\\\{\}]
{\color{outcolor}Out[{\color{outcolor}9}]:} \{\}
\end{Verbatim}
            
    \begin{Verbatim}[commandchars=\\\{\}]
{\color{incolor}In [{\color{incolor}10}]:} \PY{k}{def} \PY{n+nf}{color}\PY{p}{(}\PY{n}{g}\PY{p}{)}\PY{p}{:}
             \PY{k}{if} \PY{n}{g}\PY{o}{==}\PY{l+s+s1}{\PYZsq{}}\PY{l+s+s1}{m}\PY{l+s+s1}{\PYZsq{}}\PY{p}{:}
                 \PY{n}{col}\PY{o}{=}\PY{l+s+s1}{\PYZsq{}}\PY{l+s+s1}{blue}\PY{l+s+s1}{\PYZsq{}}
             \PY{k}{elif} \PY{n}{g}\PY{o}{==}\PY{l+s+s1}{\PYZsq{}}\PY{l+s+s1}{f}\PY{l+s+s1}{\PYZsq{}}\PY{p}{:}
                 \PY{n}{col}\PY{o}{=}\PY{l+s+s1}{\PYZsq{}}\PY{l+s+s1}{red}\PY{l+s+s1}{\PYZsq{}}
             \PY{k}{else}\PY{p}{:}
                 \PY{n}{col}\PY{o}{=}\PY{l+s+s1}{\PYZsq{}}\PY{l+s+s1}{green}\PY{l+s+s1}{\PYZsq{}}
             \PY{k}{return} \PY{n}{col}
\end{Verbatim}


    \section{ITEM A. HACEMOS UNA COMPARACION DE LOS
LAYOUTS}\label{item-a.-hacemos-una-comparacion-de-los-layouts}

    \begin{Verbatim}[commandchars=\\\{\}]
{\color{incolor}In [{\color{incolor}188}]:} \PY{n}{options} \PY{o}{=} \PY{p}{\PYZob{}}\PY{l+s+s1}{\PYZsq{}}\PY{l+s+s1}{node\PYZus{}color}\PY{l+s+s1}{\PYZsq{}}\PY{p}{:}\PY{p}{[}\PY{n}{color}\PY{p}{(}\PY{n}{g}\PY{p}{)} \PY{k}{for} \PY{n}{g} \PY{o+ow}{in} \PY{n}{nx}\PY{o}{.}\PY{n}{get\PYZus{}node\PYZus{}attributes}\PY{p}{(}\PY{n}{G}\PY{p}{,}\PY{l+s+s1}{\PYZsq{}}\PY{l+s+s1}{gender}\PY{l+s+s1}{\PYZsq{}}\PY{p}{)}\PY{o}{.}\PY{n}{values}\PY{p}{(}\PY{p}{)}\PY{p}{]}\PY{p}{,}\PY{l+s+s1}{\PYZsq{}}\PY{l+s+s1}{node\PYZus{}size}\PY{l+s+s1}{\PYZsq{}}\PY{p}{:}\PY{l+m+mi}{60}\PY{p}{,}\PY{l+s+s1}{\PYZsq{}}\PY{l+s+s1}{with\PYZus{}labels}\PY{l+s+s1}{\PYZsq{}}\PY{p}{:}\PY{k+kc}{True}\PY{p}{\PYZcb{}}
          \PY{n}{plt}\PY{o}{.}\PY{n}{figure}\PY{p}{(}\PY{n}{figsize}\PY{o}{=}\PY{p}{(}\PY{l+m+mi}{15}\PY{p}{,}\PY{l+m+mi}{10}\PY{p}{)}\PY{p}{)}
          \PY{n}{plt}\PY{o}{.}\PY{n}{subplot}\PY{p}{(}\PY{l+m+mi}{221}\PY{p}{)}
          \PY{n}{nx}\PY{o}{.}\PY{n}{draw\PYZus{}random}\PY{p}{(}\PY{n}{G}\PY{p}{,} \PY{o}{*}\PY{o}{*}\PY{n}{options}\PY{p}{)}
          \PY{n}{plt}\PY{o}{.}\PY{n}{subplot}\PY{p}{(}\PY{l+m+mi}{222}\PY{p}{)}
          \PY{n}{nx}\PY{o}{.}\PY{n}{draw\PYZus{}circular}\PY{p}{(}\PY{n}{G}\PY{p}{,} \PY{o}{*}\PY{o}{*}\PY{n}{options}\PY{p}{)}
          \PY{n}{plt}\PY{o}{.}\PY{n}{subplot}\PY{p}{(}\PY{l+m+mi}{223}\PY{p}{)}
          \PY{n}{nx}\PY{o}{.}\PY{n}{draw\PYZus{}spectral}\PY{p}{(}\PY{n}{G}\PY{p}{,} \PY{o}{*}\PY{o}{*}\PY{n}{options}\PY{p}{)}
          \PY{n}{plt}\PY{o}{.}\PY{n}{subplot}\PY{p}{(}\PY{l+m+mi}{224}\PY{p}{)}
          \PY{n}{nx}\PY{o}{.}\PY{n}{draw\PYZus{}spring}\PY{p}{(}\PY{n}{G}\PY{p}{,} \PY{o}{*}\PY{o}{*}\PY{n}{options}\PY{p}{)}
\end{Verbatim}


    \begin{Verbatim}[commandchars=\\\{\}]
C:\textbackslash{}Users\textbackslash{}Elizabeth\textbackslash{}Anaconda3\textbackslash{}lib\textbackslash{}site-packages\textbackslash{}matplotlib\textbackslash{}font\_manager.py:1328: UserWarning: findfont: Font family ['normal'] not found. Falling back to DejaVu Sans
  (prop.get\_family(), self.defaultFamily[fontext]))

    \end{Verbatim}

    \begin{center}
    \adjustimage{max size={0.9\linewidth}{0.9\paperheight}}{output_12_1.png}
    \end{center}
    { \hspace*{\fill} \\}
    
    \section{AHORA NOS QUEDAMOS CON EL MEJOR LAYOUT PORQUE PERMITE VER BIEN
UNA ESTRUCTURA. LO QUE HACE ESTE LAYOUT ES EL SIGUIENTE
ALGORITMO:}\label{ahora-nos-quedamos-con-el-mejor-layout-porque-permite-ver-bien-una-estructura.-lo-que-hace-este-layout-es-el-siguiente-algoritmo}

Fruchterman-Reingold force-directed

ref: https://en.wikipedia.org/wiki/Force-directed\_graph\_drawing

    \begin{Verbatim}[commandchars=\\\{\}]
{\color{incolor}In [{\color{incolor}12}]:} \PY{n}{options} \PY{o}{=} \PY{p}{\PYZob{}}\PY{l+s+s1}{\PYZsq{}}\PY{l+s+s1}{node\PYZus{}color}\PY{l+s+s1}{\PYZsq{}}\PY{p}{:}\PY{p}{[}\PY{n}{color}\PY{p}{(}\PY{n}{g}\PY{p}{)} \PY{k}{for} \PY{n}{g} \PY{o+ow}{in} \PY{n}{nx}\PY{o}{.}\PY{n}{get\PYZus{}node\PYZus{}attributes}\PY{p}{(}\PY{n}{G}\PY{p}{,}\PY{l+s+s1}{\PYZsq{}}\PY{l+s+s1}{gender}\PY{l+s+s1}{\PYZsq{}}\PY{p}{)}\PY{o}{.}\PY{n}{values}\PY{p}{(}\PY{p}{)}\PY{p}{]}\PY{p}{,}\PY{l+s+s1}{\PYZsq{}}\PY{l+s+s1}{node\PYZus{}size}\PY{l+s+s1}{\PYZsq{}}\PY{p}{:}\PY{l+m+mi}{200}\PY{p}{,}\PY{l+s+s1}{\PYZsq{}}\PY{l+s+s1}{with\PYZus{}labels}\PY{l+s+s1}{\PYZsq{}}\PY{p}{:}\PY{k+kc}{True}\PY{p}{\PYZcb{}}
         \PY{n}{plt}\PY{o}{.}\PY{n}{figure}\PY{p}{(}\PY{n}{figsize}\PY{o}{=}\PY{p}{(}\PY{l+m+mi}{15}\PY{p}{,}\PY{l+m+mi}{10}\PY{p}{)}\PY{p}{)}
         \PY{n}{nx}\PY{o}{.}\PY{n}{draw\PYZus{}spring}\PY{p}{(}\PY{n}{G}\PY{p}{,} \PY{o}{*}\PY{o}{*}\PY{n}{options}\PY{p}{)}
\end{Verbatim}


    \begin{center}
    \adjustimage{max size={0.9\linewidth}{0.9\paperheight}}{output_14_0.png}
    \end{center}
    { \hspace*{\fill} \\}
    
    \begin{Verbatim}[commandchars=\\\{\}]
{\color{incolor}In [{\color{incolor}13}]:} \PY{c+c1}{\PYZsh{}PARA VER SI ERA RED DIRIGIDA}
         
         \PY{n}{set1}\PY{o}{=}\PY{n+nb}{set}\PY{p}{(}\PY{n}{G}\PY{o}{.}\PY{n}{edges}\PY{p}{)}
         \PY{c+c1}{\PYZsh{}set1}
         \PY{n}{set2} \PY{o}{=} \PY{p}{\PYZob{}}\PY{p}{(}\PY{n}{nombre2}\PY{p}{,}\PY{n}{nombre1}\PY{p}{)} \PY{k}{for} \PY{n}{nombre1}\PY{p}{,} \PY{n}{nombre2}  \PY{o+ow}{in} \PY{n}{set1}\PY{p}{\PYZcb{}}
         \PY{c+c1}{\PYZsh{}set2}
         \PY{n}{set1}\PY{o}{.}\PY{n}{intersection}\PY{p}{(}\PY{n}{set2}\PY{p}{)}
\end{Verbatim}


\begin{Verbatim}[commandchars=\\\{\}]
{\color{outcolor}Out[{\color{outcolor}13}]:} set()
\end{Verbatim}
            
    \begin{Verbatim}[commandchars=\\\{\}]
{\color{incolor}In [{\color{incolor}14}]:} \PY{n}{pos}\PY{o}{=}\PY{n}{nx}\PY{o}{.}\PY{n}{spring\PYZus{}layout}\PY{p}{(}\PY{n}{G}\PY{p}{)}
         \PY{n}{pos}
\end{Verbatim}


\begin{Verbatim}[commandchars=\\\{\}]
{\color{outcolor}Out[{\color{outcolor}14}]:} \{'Beak': array([-0.19001891, -0.15976109]),
          'Beescratch': array([ 0.41141584, -0.03467136]),
          'Bumper': array([-0.32744294, -0.40711594]),
          'CCL': array([-0.12228902, -0.07953905]),
          'Cross': array([-0.41512293,  0.48517657]),
          'DN16': array([0.86832373, 0.24198342]),
          'DN21': array([0.79366825, 0.12713177]),
          'DN63': array([ 0.26346616, -0.08533439]),
          'Double': array([-0.11133487, -0.01580336]),
          'Feather': array([0.77064992, 0.22924626]),
          'Fish': array([-0.26027585, -0.2735212 ]),
          'Five': array([-0.5116831 ,  0.44403654]),
          'Fork': array([-0.70232443,  0.10822911]),
          'Gallatin': array([0.72914943, 0.19757529]),
          'Grin': array([-0.36734966, -0.08348162]),
          'Haecksel': array([-0.2133005 ,  0.11195443]),
          'Hook': array([-0.40138989, -0.12982592]),
          'Jet': array([0.73852445, 0.01206774]),
          'Jonah': array([-0.32840241,  0.09699877]),
          'Knit': array([ 0.3701245 , -0.13279082]),
          'Kringel': array([-0.21186667, -0.10793752]),
          'MN105': array([-0.43122871,  0.09644692]),
          'MN23': array([ 0.9799152 , -0.01949077]),
          'MN60': array([-0.1446924 ,  0.21666894]),
          'MN83': array([-0.37860877,  0.12790243]),
          'Mus': array([ 0.72227409, -0.16004395]),
          'Notch': array([ 0.59955951, -0.18233537]),
          'Number1': array([ 0.54723155, -0.10502475]),
          'Oscar': array([ 0.08331784, -0.12094605]),
          'Patchback': array([-0.42532216, -0.02331373]),
          'PL': array([ 0.13139349, -0.23079814]),
          'Quasi': array([ 0.94955957, -0.11177814]),
          'Ripplefluke': array([0.76644908, 0.43888649]),
          'Scabs': array([-0.46838228, -0.03326897]),
          'Shmuddel': array([-0.5697087 , -0.12133468]),
          'SMN5': array([-0.65053773, -0.17607871]),
          'SN100': array([0.03372287, 0.02682552]),
          'SN4': array([-0.32311058, -0.08240019]),
          'SN63': array([-0.48716545, -0.14075748]),
          'SN89': array([0.36029706, 0.14829078]),
          'SN9': array([-0.15614394, -0.05524667]),
          'SN90': array([0.62064027, 0.11789514]),
          'SN96': array([-0.14641312, -0.27568386]),
          'Stripes': array([-0.56056859, -0.18024336]),
          'Thumper': array([-0.42097255, -0.26580346]),
          'Topless': array([-0.27203417,  0.09552636]),
          'TR120': array([-0.82197975, -0.24589741]),
          'TR77': array([-0.06036422, -0.24409953]),
          'TR82': array([0.85825266, 0.35579195]),
          'TR88': array([-0.8418378 , -0.15225881]),
          'TR99': array([-0.30223091, -0.0261154 ]),
          'Trigger': array([-0.3625055 ,  0.23199716]),
          'TSN103': array([-0.34878944, -0.16591909]),
          'TSN83': array([-0.65670039, -0.38789986]),
          'Upbang': array([0.54961037, 0.04085644]),
          'Vau': array([-0.26593503,  0.34234914]),
          'Wave': array([1.        , 0.20824803]),
          'Web': array([0.67491052, 0.18527081]),
          'Whitetip': array([-0.7484596 , -0.06711407]),
          'Zap': array([-0.09171325,  0.08606688]),
          'Zig': array([0.75310455, 0.66583686]),
          'Zipfel': array([-0.47735464, -0.35562504])\}
\end{Verbatim}
            
    \begin{Verbatim}[commandchars=\\\{\}]
{\color{incolor}In [{\color{incolor}15}]:} \PY{n}{p}\PY{o}{=}\PY{l+m+mi}{0} \PY{c+c1}{\PYZsh{}female}
         \PY{n}{q}\PY{o}{=}\PY{l+m+mi}{0} \PY{c+c1}{\PYZsh{}male}
         \PY{n}{r}\PY{o}{=}\PY{l+m+mi}{0} \PY{c+c1}{\PYZsh{}na}
         \PY{k}{for} \PY{n}{i} \PY{o+ow}{in} \PY{n+nb}{range}\PY{p}{(}\PY{n}{np}\PY{o}{.}\PY{n}{shape}\PY{p}{(}\PY{n}{dolphinsGender}\PY{p}{)}\PY{p}{[}\PY{l+m+mi}{0}\PY{p}{]}\PY{p}{)}\PY{p}{:}
             \PY{k}{if} \PY{n}{dolphinsGender}\PY{p}{[}\PY{n}{i}\PY{p}{]}\PY{p}{[}\PY{l+m+mi}{1}\PY{p}{]}\PY{o}{==}\PY{l+s+s1}{\PYZsq{}}\PY{l+s+s1}{f}\PY{l+s+s1}{\PYZsq{}}\PY{p}{:}
                 \PY{n}{p}\PY{o}{=}\PY{n}{p}\PY{o}{+}\PY{l+m+mi}{1}
             \PY{k}{elif} \PY{n}{dolphinsGender}\PY{p}{[}\PY{n}{i}\PY{p}{]}\PY{p}{[}\PY{l+m+mi}{1}\PY{p}{]}\PY{o}{==}\PY{l+s+s1}{\PYZsq{}}\PY{l+s+s1}{m}\PY{l+s+s1}{\PYZsq{}}\PY{p}{:}
                 \PY{n}{q}\PY{o}{=}\PY{n}{q}\PY{o}{+}\PY{l+m+mi}{1}
             \PY{k}{else}\PY{p}{:}
                 \PY{n}{r}\PY{o}{=}\PY{n}{r}\PY{o}{+}\PY{l+m+mi}{1}
         \PY{p}{[}\PY{n}{p}\PY{p}{,}\PY{n}{q}\PY{p}{,}\PY{n}{r}\PY{p}{]}
\end{Verbatim}


\begin{Verbatim}[commandchars=\\\{\}]
{\color{outcolor}Out[{\color{outcolor}15}]:} [24, 34, 4]
\end{Verbatim}
            
    \section{PUNTO B-I}\label{punto-b-i}

    1ro: dividir el diccionario de dict\_gender por la key y por el value.

2do: La key la dejamos quita y vamo a aplicar shuffle al value.

3ro: Eso del shuffle lo hacemos en cada iteracion (1000 en total) y
generamos el diccionario devuelta uniendo ambas cosas para asignar
atributo gender a los nodos.

4to: Calculamos la modularidad \$ \frac{1}{2m} \sum\emph{\{ij\}
\delta(c}\{i\}, c\_\{j\}) (A\_\{ij\}-\frac{k_{i}k_{j}}{m})\$

\(k_{i}\) el grado del nodo i, \(k_{j}\) el grado del nodo j, \(c_{i}\)
la clase i, \(c_{j}\) la clase j, \(A_{ij} = 1\) si hay un enlace entre
el nodo i y el nodo j o sino \(A_{ij} = 0\) m la cantidad de enlaces
total

\(\delta(c_{i}, c_{j}) = \frac{c_{i}c_{j} + 1}{2}\), esto es solo para
dos clases.. aca tenemos 3

    \begin{Verbatim}[commandchars=\\\{\}]
{\color{incolor}In [{\color{incolor}70}]:} \PY{n}{A} \PY{o}{=} \PY{n}{nx}\PY{o}{.}\PY{n}{to\PYZus{}numpy\PYZus{}matrix}\PY{p}{(}\PY{n}{G}\PY{p}{)}                            \PY{c+c1}{\PYZsh{}hace la adjency matrix}
         \PY{n}{np}\PY{o}{.}\PY{n}{size}\PY{p}{(}\PY{n}{A}\PY{p}{[}\PY{l+m+mi}{0}\PY{p}{]}\PY{p}{)}\PY{p}{,} \PY{n}{np}\PY{o}{.}\PY{n}{size}\PY{p}{(}\PY{n}{nx}\PY{o}{.}\PY{n}{nodes}\PY{p}{(}\PY{n}{G}\PY{p}{)}\PY{p}{)}                  \PY{c+c1}{\PYZsh{}chequeo que el tamano este bien}
\end{Verbatim}


\begin{Verbatim}[commandchars=\\\{\}]
{\color{outcolor}Out[{\color{outcolor}70}]:} (62, 62, 0.0)
\end{Verbatim}
            
    \begin{Verbatim}[commandchars=\\\{\}]
{\color{incolor}In [{\color{incolor}73}]:} \PY{n}{G}\PY{o}{.}\PY{n}{nodes}\PY{p}{(}\PY{p}{)}
\end{Verbatim}


\begin{Verbatim}[commandchars=\\\{\}]
{\color{outcolor}Out[{\color{outcolor}73}]:} NodeView(('Beak', 'Beescratch', 'Bumper', 'CCL', 'Cross', 'DN16', 'DN21', 'DN63', 'Double', 'Feather', 'Fish', 'Five', 'Fork', 'Gallatin', 'Grin', 'Haecksel', 'Hook', 'Jet', 'Jonah', 'Knit', 'Kringel', 'MN105', 'MN23', 'MN60', 'MN83', 'Mus', 'Notch', 'Number1', 'Oscar', 'Patchback', 'PL', 'Quasi', 'Ripplefluke', 'Scabs', 'Shmuddel', 'SMN5', 'SN100', 'SN4', 'SN63', 'SN89', 'SN9', 'SN90', 'SN96', 'Stripes', 'Thumper', 'Topless', 'TR120', 'TR77', 'TR82', 'TR88', 'TR99', 'Trigger', 'TSN103', 'TSN83', 'Upbang', 'Vau', 'Wave', 'Web', 'Whitetip', 'Zap', 'Zig', 'Zipfel'))
\end{Verbatim}
            
    \begin{Verbatim}[commandchars=\\\{\}]
{\color{incolor}In [{\color{incolor}65}]:} \PY{n}{A} \PY{o}{=} \PY{n}{np}\PY{o}{.}\PY{n}{matrix}\PY{p}{(}\PY{p}{[}\PY{p}{[}\PY{l+m+mi}{1}\PY{p}{,} \PY{l+m+mi}{2}\PY{p}{,} \PY{l+m+mi}{2}\PY{p}{]} \PY{p}{,}\PY{p}{[}\PY{l+m+mi}{3} \PY{p}{,}\PY{l+m+mi}{4}\PY{p}{,}\PY{l+m+mi}{5}\PY{p}{]}\PY{p}{]}\PY{p}{)}
         \PY{n}{np}\PY{o}{.}\PY{n}{size}\PY{p}{(}\PY{n}{A}\PY{p}{[}\PY{p}{:}\PY{p}{,}\PY{l+m+mi}{1}\PY{p}{]}\PY{p}{)}   \PY{c+c1}{\PYZsh{}de la columna 1}
         \PY{n}{np}\PY{o}{.}\PY{n}{size}\PY{p}{(}\PY{n}{A}\PY{p}{[}\PY{p}{:}\PY{p}{,}\PY{l+m+mi}{1}\PY{p}{]}\PY{p}{)}   \PY{c+c1}{\PYZsh{}de la fila 1}
\end{Verbatim}


\begin{Verbatim}[commandchars=\\\{\}]
{\color{outcolor}Out[{\color{outcolor}65}]:} 2
\end{Verbatim}
            
    \begin{Verbatim}[commandchars=\\\{\}]
{\color{incolor}In [{\color{incolor}93}]:} \PY{n}{Degree} \PY{o}{=} \PY{n+nb}{list}\PY{p}{(}\PY{n}{G}\PY{o}{.}\PY{n}{degree}\PY{p}{)}                          \PY{c+c1}{\PYZsh{}hago una lista a partir de lo que devuelve G.degree }
         \PY{n}{degree} \PY{o}{=} \PY{p}{[}\PY{n}{degree} \PY{k}{for} \PY{n}{node}\PY{p}{,}\PY{n}{degree} \PY{o+ow}{in} \PY{n}{Degree}\PY{p}{]}      \PY{c+c1}{\PYZsh{}me quedo con un vector donde cada elemento es el grado para el nodo indexado como 0,1,2.. }
                                                          \PY{c+c1}{\PYZsh{} que se corresponde con el nombre de los delfines segun el orden que muestra matriz A }
                                                          \PY{c+c1}{\PYZsh{} que seria el que da G.nodes()}
         \PY{c+c1}{\PYZsh{}degree}
\end{Verbatim}


    \begin{Verbatim}[commandchars=\\\{\}]
{\color{incolor}In [{\color{incolor}97}]:} \PY{k}{def} \PY{n+nf}{delta}\PY{p}{(}\PY{n}{c\PYZus{}i}\PY{p}{,} \PY{n}{c\PYZus{}j}\PY{p}{)}\PY{p}{:}
             \PY{k}{if}  \PY{n}{c\PYZus{}i} \PY{o}{==} \PY{n}{c\PYZus{}j}\PY{p}{:}
                 \PY{k}{return} \PY{l+m+mi}{1} 
             \PY{k}{else}\PY{p}{:}
                 \PY{k}{return} \PY{l+m+mi}{0} 
\end{Verbatim}


    \begin{Verbatim}[commandchars=\\\{\}]
{\color{incolor}In [{\color{incolor}102}]:} \PY{n}{delta}\PY{p}{(}\PY{n}{A}\PY{p}{[}\PY{l+m+mi}{2}\PY{p}{]}\PY{p}{,} \PY{n}{A}\PY{p}{[}\PY{l+m+mi}{1}\PY{p}{]}\PY{p}{)}
\end{Verbatim}


\begin{Verbatim}[commandchars=\\\{\}]
{\color{outcolor}Out[{\color{outcolor}102}]:} 1
\end{Verbatim}
            
    \begin{Verbatim}[commandchars=\\\{\}]
{\color{incolor}In [{\color{incolor}123}]:} \PY{c+c1}{\PYZsh{}DEFINIMOSLA FUNCION MODULARIDAD a partir de un grafo G que ya tiene montado el campo escalar que le da una clase a cada nodo}
          
          \PY{k}{def} \PY{n+nf}{modularity}\PY{p}{(}\PY{n}{G}\PY{p}{,}\PY{n}{FieldString}\PY{p}{)}\PY{p}{:}
              \PY{n}{modularity} \PY{o}{=} \PY{l+m+mi}{0}
              \PY{n}{m} \PY{o}{=} \PY{n}{np}\PY{o}{.}\PY{n}{size}\PY{p}{(}\PY{n}{nx}\PY{o}{.}\PY{n}{edges}\PY{p}{(}\PY{n}{G}\PY{p}{)}\PY{p}{)}
              \PY{n}{A} \PY{o}{=} \PY{n}{nx}\PY{o}{.}\PY{n}{to\PYZus{}numpy\PYZus{}matrix}\PY{p}{(}\PY{n}{G}\PY{p}{)}     
              \PY{k}{for} \PY{n}{i} \PY{o+ow}{in} \PY{n+nb}{range}\PY{p}{(}\PY{n}{np}\PY{o}{.}\PY{n}{size}\PY{p}{(}\PY{n}{nx}\PY{o}{.}\PY{n}{nodes}\PY{p}{(}\PY{n}{G}\PY{p}{)}\PY{p}{)}\PY{p}{)}\PY{p}{:}
                  \PY{k}{for} \PY{n}{j} \PY{o+ow}{in} \PY{n+nb}{range}\PY{p}{(}\PY{n}{np}\PY{o}{.}\PY{n}{size}\PY{p}{(}\PY{n}{nx}\PY{o}{.}\PY{n}{nodes}\PY{p}{(}\PY{n}{G}\PY{p}{)}\PY{p}{)}\PY{p}{)}\PY{p}{:}
                      \PY{n}{A\PYZus{}ij} \PY{o}{=} \PY{n}{A}\PY{p}{[}\PY{n}{i}\PY{p}{,}\PY{n}{j}\PY{p}{]}
                      \PY{n}{Degree} \PY{o}{=} \PY{n+nb}{list}\PY{p}{(}\PY{n}{G}\PY{o}{.}\PY{n}{degree}\PY{p}{)}                           \PY{c+c1}{\PYZsh{}hago una lista a partir de lo que devuelve G.degree }
                      \PY{n}{degree} \PY{o}{=} \PY{p}{[}\PY{n}{degree} \PY{k}{for} \PY{n}{node}\PY{p}{,}\PY{n}{degree} \PY{o+ow}{in} \PY{n}{Degree}\PY{p}{]} 
                      \PY{n}{k\PYZus{}i} \PY{o}{=} \PY{n}{degree}\PY{p}{[}\PY{n}{i}\PY{p}{]}
                      \PY{n}{k\PYZus{}j} \PY{o}{=} \PY{n}{degree}\PY{p}{[}\PY{n}{j}\PY{p}{]}
                      \PY{n}{atrributes} \PY{o}{=} \PY{n+nb}{list}\PY{p}{(}\PY{n}{nx}\PY{o}{.}\PY{n}{get\PYZus{}node\PYZus{}attributes}\PY{p}{(}\PY{n}{G}\PY{p}{,}\PY{n}{FieldString}\PY{p}{)}\PY{o}{.}\PY{n}{values}\PY{p}{(}\PY{p}{)}\PY{p}{)}
                      \PY{n}{c\PYZus{}i} \PY{o}{=} \PY{n}{atrributes}\PY{p}{[}\PY{n}{i}\PY{p}{]}
                      \PY{n}{c\PYZus{}j} \PY{o}{=} \PY{n}{atrributes}\PY{p}{[}\PY{n}{j}\PY{p}{]}
                      \PY{n}{modularity} \PY{o}{=} \PY{p}{(}\PY{l+m+mi}{1}\PY{o}{/}\PY{p}{(}\PY{l+m+mi}{2} \PY{o}{*} \PY{n}{m}\PY{p}{)}\PY{p}{)} \PY{o}{*} \PY{n}{delta}\PY{p}{(}\PY{n}{c\PYZus{}i}\PY{p}{,} \PY{n}{c\PYZus{}j}\PY{p}{)} \PY{o}{*} \PY{p}{(}\PY{n}{A\PYZus{}ij} \PY{o}{+} \PY{p}{(}\PY{n}{k\PYZus{}i} \PY{o}{*} \PY{n}{k\PYZus{}j}\PY{p}{)}\PY{o}{/}\PY{n}{m}\PY{p}{)} \PY{o}{+} \PY{n}{modularity}
              \PY{k}{return} \PY{n}{modularity}
                      
                      
\end{Verbatim}


    \begin{Verbatim}[commandchars=\\\{\}]
{\color{incolor}In [{\color{incolor}127}]:} \PY{n}{modularity}\PY{p}{(}\PY{n}{G}\PY{p}{,} \PY{l+s+s1}{\PYZsq{}}\PY{l+s+s1}{gender}\PY{l+s+s1}{\PYZsq{}}\PY{p}{)}            \PY{c+c1}{\PYZsh{}da un valor entre \PYZhy{}1 y 1 entonces..creo que esta mas o menos bien }
\end{Verbatim}


\begin{Verbatim}[commandchars=\\\{\}]
{\color{outcolor}Out[{\color{outcolor}127}]:} 0.46883034690083286
\end{Verbatim}
            
    \begin{Verbatim}[commandchars=\\\{\}]
{\color{incolor}In [{\color{incolor}128}]:} \PY{c+c1}{\PYZsh{}genders=Array(dict\PYZus{}gender.values())}
          \PY{k}{def} \PY{n+nf}{BreakFieldAttribution}\PY{p}{(}\PY{n}{fieldAtributtion}\PY{p}{)}\PY{p}{:}
              \PY{n}{names} \PY{o}{=} \PY{p}{[}\PY{n}{nodes} \PY{k}{for} \PY{n}{nodes}\PY{p}{,}\PY{n}{field} \PY{o+ow}{in} \PY{n}{fieldAtributtion}\PY{p}{]}
              \PY{n}{field} \PY{o}{=} \PY{p}{[}\PY{n}{field} \PY{k}{for} \PY{n}{nodes}\PY{p}{,}\PY{n}{field} \PY{o+ow}{in} \PY{n}{fieldAtributtion}\PY{p}{]}
              \PY{k}{return} \PY{n}{names}\PY{p}{,}\PY{n}{field}
\end{Verbatim}


    \begin{Verbatim}[commandchars=\\\{\}]
{\color{incolor}In [{\color{incolor}115}]:} \PY{n}{names} \PY{o}{=} \PY{n}{BreakFieldAttribution}\PY{p}{(}\PY{n}{dolphinsGender}\PY{p}{)}\PY{p}{[}\PY{l+m+mi}{0}\PY{p}{]}
          \PY{n}{genders} \PY{o}{=} \PY{n}{BreakFieldAttribution}\PY{p}{(}\PY{n}{dolphinsGender}\PY{p}{)}\PY{p}{[}\PY{l+m+mi}{1}\PY{p}{]}
          \PY{n}{np}\PY{o}{.}\PY{n}{random}\PY{o}{.}\PY{n}{shuffle}\PY{p}{(}\PY{n}{genders}\PY{p}{)} 
          \PY{n}{dict\PYZus{}gender\PYZus{}together} \PY{o}{=} \PY{n+nb}{dict}\PY{p}{(}\PY{n+nb}{zip}\PY{p}{(}\PY{n}{names}\PY{p}{,}\PY{n}{genders}\PY{p}{)}\PY{p}{)}
          \PY{n}{H} \PY{o}{=} \PY{n}{nx}\PY{o}{.}\PY{n}{read\PYZus{}gml}\PY{p}{(}\PY{l+s+s1}{\PYZsq{}}\PY{l+s+s1}{dolphins.gml}\PY{l+s+s1}{\PYZsq{}}\PY{p}{)}                  \PY{c+c1}{\PYZsh{}trabajamos sobre H (grafo nuevo)}
\end{Verbatim}


    \begin{Verbatim}[commandchars=\\\{\}]
{\color{incolor}In [{\color{incolor}137}]:} \PY{n}{DistributionModularity} \PY{o}{=} \PY{p}{[}\PY{p}{]}                           \PY{c+c1}{\PYZsh{}vamos a ir appendando ahi en cada corrida }
          
          \PY{n}{names} \PY{o}{=} \PY{n}{BreakFieldAttribution}\PY{p}{(}\PY{n}{dolphinsGender}\PY{p}{)}\PY{p}{[}\PY{l+m+mi}{0}\PY{p}{]}
          \PY{n}{genders} \PY{o}{=} \PY{n}{BreakFieldAttribution}\PY{p}{(}\PY{n}{dolphinsGender}\PY{p}{)}\PY{p}{[}\PY{l+m+mi}{1}\PY{p}{]}
          \PY{n}{it} \PY{o}{=} \PY{l+m+mi}{1000}                                              \PY{c+c1}{\PYZsh{}numero de iteraciones }
          
          \PY{k}{for} \PY{n}{i} \PY{o+ow}{in} \PY{n+nb}{range}\PY{p}{(}\PY{n}{it}\PY{p}{)}\PY{p}{:}
              \PY{n}{np}\PY{o}{.}\PY{n}{random}\PY{o}{.}\PY{n}{shuffle}\PY{p}{(}\PY{n}{genders}\PY{p}{)}                       \PY{c+c1}{\PYZsh{}los mezcla randomly }
              \PY{n}{dict\PYZus{}gender\PYZus{}together} \PY{o}{=} \PY{n+nb}{dict}\PY{p}{(}\PY{n+nb}{zip}\PY{p}{(}\PY{n}{names}\PY{p}{,}\PY{n}{genders}\PY{p}{)}\PY{p}{)}  \PY{c+c1}{\PYZsh{}los vuelvo a poner como un diccionario }
              \PY{n}{H} \PY{o}{=} \PY{n}{nx}\PY{o}{.}\PY{n}{read\PYZus{}gml}\PY{p}{(}\PY{l+s+s1}{\PYZsq{}}\PY{l+s+s1}{dolphins.gml}\PY{l+s+s1}{\PYZsq{}}\PY{p}{)}                  \PY{c+c1}{\PYZsh{}trabajamos sobre H (grafo nuevo)}
              \PY{n}{AssignGender}\PY{p}{(}\PY{n}{H}\PY{p}{,}\PY{n}{dict\PYZus{}gender\PYZus{}together}\PY{p}{)}              
              \PY{c+c1}{\PYZsh{}modularity = modularity(H,\PYZsq{}gender\PYZsq{})}
              \PY{n}{DistributionModularity}\PY{o}{.}\PY{n}{append}\PY{p}{(}\PY{n}{modularity}\PY{p}{(}\PY{n}{H}\PY{p}{,}\PY{l+s+s1}{\PYZsq{}}\PY{l+s+s1}{gender}\PY{l+s+s1}{\PYZsq{}}\PY{p}{)}\PY{p}{)}
          
          \PY{c+c1}{\PYZsh{}ahora como output de este loop tenemos DistributionModularity que es la distribución de los valores de modularidad para la corrida }
          \PY{c+c1}{\PYZsh{}tantas veces }
          
          \PY{c+c1}{\PYZsh{}entonces nos quedaría hacer un histograma de esto y calcular el p value. }
\end{Verbatim}


    \begin{Verbatim}[commandchars=\\\{\}]
{\color{incolor}In [{\color{incolor}139}]:} \PY{n}{np}\PY{o}{.}\PY{n}{savetxt}\PY{p}{(}\PY{l+s+s1}{\PYZsq{}}\PY{l+s+s1}{DistributionModularity.txt}\PY{l+s+s1}{\PYZsq{}}\PY{p}{,} \PY{n}{DistributionModularity}\PY{p}{,} \PY{n}{delimiter}\PY{o}{=}\PY{l+s+s1}{\PYZsq{}}\PY{l+s+s1}{,}\PY{l+s+s1}{\PYZsq{}}\PY{p}{)} 
\end{Verbatim}


    EL HISTOGRAMA DE DistributionModularity

    \begin{Verbatim}[commandchars=\\\{\}]
{\color{incolor}In [{\color{incolor}145}]:} \PY{n}{font} \PY{o}{=} \PY{p}{\PYZob{}}\PY{l+s+s1}{\PYZsq{}}\PY{l+s+s1}{family}\PY{l+s+s1}{\PYZsq{}} \PY{p}{:} \PY{l+s+s1}{\PYZsq{}}\PY{l+s+s1}{normal}\PY{l+s+s1}{\PYZsq{}}\PY{p}{,}
                  \PY{l+s+s1}{\PYZsq{}}\PY{l+s+s1}{weight}\PY{l+s+s1}{\PYZsq{}} \PY{p}{:} \PY{l+s+s1}{\PYZsq{}}\PY{l+s+s1}{bold}\PY{l+s+s1}{\PYZsq{}}\PY{p}{,}
                  \PY{l+s+s1}{\PYZsq{}}\PY{l+s+s1}{size}\PY{l+s+s1}{\PYZsq{}}   \PY{p}{:} \PY{l+m+mi}{22}\PY{p}{\PYZcb{}}
          
          \PY{n}{plt}\PY{o}{.}\PY{n}{rc}\PY{p}{(}\PY{l+s+s1}{\PYZsq{}}\PY{l+s+s1}{font}\PY{l+s+s1}{\PYZsq{}}\PY{p}{,} \PY{o}{*}\PY{o}{*}\PY{n}{font}\PY{p}{)}
          
          \PY{n}{plt}\PY{o}{.}\PY{n}{figure}\PY{p}{(}\PY{n}{figsize}\PY{o}{=}\PY{p}{(}\PY{l+m+mi}{15}\PY{p}{,}\PY{l+m+mi}{10}\PY{p}{)}\PY{p}{)}
          \PY{n}{plt}\PY{o}{.}\PY{n}{hist}\PY{p}{(}\PY{n}{DistributionModularity}\PY{p}{)}
          \PY{n}{plt}\PY{o}{.}\PY{n}{title}\PY{p}{(}\PY{l+s+s2}{\PYZdq{}}\PY{l+s+s2}{Histograma de la distribución de modularidad para 1000 iteraciones de asignaciones aleatorias de sexo}\PY{l+s+s2}{\PYZdq{}}\PY{p}{)}
          \PY{n}{plt}\PY{o}{.}\PY{n}{xlabel}\PY{p}{(}\PY{l+s+s2}{\PYZdq{}}\PY{l+s+s2}{Modularidad}\PY{l+s+s2}{\PYZdq{}}\PY{p}{)}
          \PY{n}{plt}\PY{o}{.}\PY{n}{ylabel}\PY{p}{(}\PY{l+s+s2}{\PYZdq{}}\PY{l+s+s2}{Frecuencia}\PY{l+s+s2}{\PYZdq{}}\PY{p}{)}
\end{Verbatim}


\begin{Verbatim}[commandchars=\\\{\}]
{\color{outcolor}Out[{\color{outcolor}145}]:} Text(0,0.5,'Frecuencia')
\end{Verbatim}
            
    \begin{Verbatim}[commandchars=\\\{\}]
C:\textbackslash{}Users\textbackslash{}Elizabeth\textbackslash{}Anaconda3\textbackslash{}lib\textbackslash{}site-packages\textbackslash{}matplotlib\textbackslash{}font\_manager.py:1328: UserWarning: findfont: Font family ['normal'] not found. Falling back to DejaVu Sans
  (prop.get\_family(), self.defaultFamily[fontext]))

    \end{Verbatim}

    \begin{center}
    \adjustimage{max size={0.9\linewidth}{0.9\paperheight}}{output_33_2.png}
    \end{center}
    { \hspace*{\fill} \\}
    
    \section{PUNTO B-II}\label{punto-b-ii}

    \begin{Verbatim}[commandchars=\\\{\}]
{\color{incolor}In [{\color{incolor}149}]:} \PY{n}{np}\PY{o}{.}\PY{n}{mean}\PY{p}{(}\PY{n}{DistributionModularity}\PY{p}{)}\PY{p}{,} \PY{n}{modularity}\PY{p}{(}\PY{n}{G}\PY{p}{,} \PY{l+s+s1}{\PYZsq{}}\PY{l+s+s1}{gender}\PY{l+s+s1}{\PYZsq{}}\PY{p}{)}\PY{p}{,} \PY{n}{np}\PY{o}{.}\PY{n}{std}\PY{p}{(}\PY{n}{DistributionModularity}\PY{p}{)}
\end{Verbatim}


\begin{Verbatim}[commandchars=\\\{\}]
{\color{outcolor}Out[{\color{outcolor}149}]:} (0.4524053538230273, 0.46883034690083286, 0.027015608737991276)
\end{Verbatim}
            
    Entonces, el valor que sale aleatorio es: \(0.452 \pm 0.027\) en
unidades arbitrarias Y el valor real es 0.469 que se encuentra dentro
del intervalo del valor medio de la distribucion aleatoria

    \section{PUNTO B-III: CALCULO DEL P
VALUE}\label{punto-b-iii-calculo-del-p-value}

    \begin{Verbatim}[commandchars=\\\{\}]
{\color{incolor}In [{\color{incolor}178}]:} \PY{c+c1}{\PYZsh{}CALCULO LAS APARICIONES DE VALORES POR ENCIMA DEL QUE PRESENTA NUESTRO GRAFO CON LA ATRIBUCION DE GENERO }
          
          \PY{n}{frecuencia} \PY{o}{=} \PY{l+m+mi}{0}
          \PY{k}{for} \PY{n}{i} \PY{o+ow}{in} \PY{n+nb}{range}\PY{p}{(}\PY{n}{np}\PY{o}{.}\PY{n}{size}\PY{p}{(}\PY{n}{DistributionModularity}\PY{p}{)}\PY{p}{)}\PY{p}{:}
              \PY{k}{if} \PY{n}{DistributionModularity}\PY{p}{[}\PY{n}{i}\PY{p}{]} \PY{o}{\PYZgt{}}\PY{o}{=}  \PY{l+m+mf}{0.469}\PY{p}{:}                  \PY{c+c1}{\PYZsh{}como nuestro valor real se encuentra a derecha buscamos los eventos mas extremos a derecha}
                  \PY{n}{frecuencia} \PY{o}{=} \PY{n}{frecuencia} \PY{o}{+} \PY{l+m+mi}{1} 
          \PY{n}{frecuencia}
\end{Verbatim}


\begin{Verbatim}[commandchars=\\\{\}]
{\color{outcolor}Out[{\color{outcolor}178}]:} 271
\end{Verbatim}
            
    \begin{Verbatim}[commandchars=\\\{\}]
{\color{incolor}In [{\color{incolor}187}]:} \PY{n}{pvalue} \PY{o}{=} \PY{p}{(}\PY{n+nb}{max}\PY{p}{(}\PY{n}{DistributionModularity}\PY{p}{)} \PY{o}{\PYZhy{}} \PY{l+m+mf}{0.469}\PY{p}{)} \PY{o}{*} \PY{p}{(}\PY{n}{frecuencia}\PY{o}{/}\PY{n}{it}\PY{p}{)}                \PY{c+c1}{\PYZsh{}es el area a derecha del valor real en el histograma}
          \PY{k}{if} \PY{n}{pvalue} \PY{o}{\PYZlt{}}\PY{o}{=} \PY{p}{(}\PY{l+m+mi}{1}\PY{o}{/}\PY{l+m+mi}{1000}\PY{p}{)}\PY{p}{:}                                                          \PY{c+c1}{\PYZsh{}en general se toma ese valor }
              \PY{n+nb}{print}\PY{p}{(}\PY{l+s+s2}{\PYZdq{}}\PY{l+s+s2}{fuera de la hipotesis nula}\PY{l+s+s2}{\PYZdq{}}\PY{p}{)}
          \PY{k}{else}\PY{p}{:}
              \PY{n+nb}{print} \PY{p}{(}\PY{l+s+s2}{\PYZdq{}}\PY{l+s+s2}{en la hipotesis nula}\PY{l+s+s2}{\PYZdq{}}\PY{p}{)}
\end{Verbatim}


    \begin{Verbatim}[commandchars=\\\{\}]
en la hipotesis nula

    \end{Verbatim}

    como tenemos un pvalue grande que el tipico treashold entonces decimos
que nuestro resultado se condice con lo de la hipotesis nula entonces
que no hay homofilia fuera de la homofilia que se genera de forma
aleatoria.


    % Add a bibliography block to the postdoc
    
    
    
    \end{document}
