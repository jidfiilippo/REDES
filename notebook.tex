
% Default to the notebook output style

    


% Inherit from the specified cell style.




    
\documentclass[11pt]{article}

    
    
    \usepackage[T1]{fontenc}
    % Nicer default font (+ math font) than Computer Modern for most use cases
    \usepackage{mathpazo}

    % Basic figure setup, for now with no caption control since it's done
    % automatically by Pandoc (which extracts ![](path) syntax from Markdown).
    \usepackage{graphicx}
    % We will generate all images so they have a width \maxwidth. This means
    % that they will get their normal width if they fit onto the page, but
    % are scaled down if they would overflow the margins.
    \makeatletter
    \def\maxwidth{\ifdim\Gin@nat@width>\linewidth\linewidth
    \else\Gin@nat@width\fi}
    \makeatother
    \let\Oldincludegraphics\includegraphics
    % Set max figure width to be 80% of text width, for now hardcoded.
    \renewcommand{\includegraphics}[1]{\Oldincludegraphics[width=.8\maxwidth]{#1}}
    % Ensure that by default, figures have no caption (until we provide a
    % proper Figure object with a Caption API and a way to capture that
    % in the conversion process - todo).
    \usepackage{caption}
    \DeclareCaptionLabelFormat{nolabel}{}
    \captionsetup{labelformat=nolabel}

    \usepackage{adjustbox} % Used to constrain images to a maximum size 
    \usepackage{xcolor} % Allow colors to be defined
    \usepackage{enumerate} % Needed for markdown enumerations to work
    \usepackage{geometry} % Used to adjust the document margins
    \usepackage{amsmath} % Equations
    \usepackage{amssymb} % Equations
    \usepackage{textcomp} % defines textquotesingle
    % Hack from http://tex.stackexchange.com/a/47451/13684:
    \AtBeginDocument{%
        \def\PYZsq{\textquotesingle}% Upright quotes in Pygmentized code
    }
    \usepackage{upquote} % Upright quotes for verbatim code
    \usepackage{eurosym} % defines \euro
    \usepackage[mathletters]{ucs} % Extended unicode (utf-8) support
    \usepackage[utf8x]{inputenc} % Allow utf-8 characters in the tex document
    \usepackage{fancyvrb} % verbatim replacement that allows latex
    \usepackage{grffile} % extends the file name processing of package graphics 
                         % to support a larger range 
    % The hyperref package gives us a pdf with properly built
    % internal navigation ('pdf bookmarks' for the table of contents,
    % internal cross-reference links, web links for URLs, etc.)
    \usepackage{hyperref}
    \usepackage{longtable} % longtable support required by pandoc >1.10
    \usepackage{booktabs}  % table support for pandoc > 1.12.2
    \usepackage[inline]{enumitem} % IRkernel/repr support (it uses the enumerate* environment)
    \usepackage[normalem]{ulem} % ulem is needed to support strikethroughs (\sout)
                                % normalem makes italics be italics, not underlines
    

    
    
    % Colors for the hyperref package
    \definecolor{urlcolor}{rgb}{0,.145,.698}
    \definecolor{linkcolor}{rgb}{.71,0.21,0.01}
    \definecolor{citecolor}{rgb}{.12,.54,.11}

    % ANSI colors
    \definecolor{ansi-black}{HTML}{3E424D}
    \definecolor{ansi-black-intense}{HTML}{282C36}
    \definecolor{ansi-red}{HTML}{E75C58}
    \definecolor{ansi-red-intense}{HTML}{B22B31}
    \definecolor{ansi-green}{HTML}{00A250}
    \definecolor{ansi-green-intense}{HTML}{007427}
    \definecolor{ansi-yellow}{HTML}{DDB62B}
    \definecolor{ansi-yellow-intense}{HTML}{B27D12}
    \definecolor{ansi-blue}{HTML}{208FFB}
    \definecolor{ansi-blue-intense}{HTML}{0065CA}
    \definecolor{ansi-magenta}{HTML}{D160C4}
    \definecolor{ansi-magenta-intense}{HTML}{A03196}
    \definecolor{ansi-cyan}{HTML}{60C6C8}
    \definecolor{ansi-cyan-intense}{HTML}{258F8F}
    \definecolor{ansi-white}{HTML}{C5C1B4}
    \definecolor{ansi-white-intense}{HTML}{A1A6B2}

    % commands and environments needed by pandoc snippets
    % extracted from the output of `pandoc -s`
    \providecommand{\tightlist}{%
      \setlength{\itemsep}{0pt}\setlength{\parskip}{0pt}}
    \DefineVerbatimEnvironment{Highlighting}{Verbatim}{commandchars=\\\{\}}
    % Add ',fontsize=\small' for more characters per line
    \newenvironment{Shaded}{}{}
    \newcommand{\KeywordTok}[1]{\textcolor[rgb]{0.00,0.44,0.13}{\textbf{{#1}}}}
    \newcommand{\DataTypeTok}[1]{\textcolor[rgb]{0.56,0.13,0.00}{{#1}}}
    \newcommand{\DecValTok}[1]{\textcolor[rgb]{0.25,0.63,0.44}{{#1}}}
    \newcommand{\BaseNTok}[1]{\textcolor[rgb]{0.25,0.63,0.44}{{#1}}}
    \newcommand{\FloatTok}[1]{\textcolor[rgb]{0.25,0.63,0.44}{{#1}}}
    \newcommand{\CharTok}[1]{\textcolor[rgb]{0.25,0.44,0.63}{{#1}}}
    \newcommand{\StringTok}[1]{\textcolor[rgb]{0.25,0.44,0.63}{{#1}}}
    \newcommand{\CommentTok}[1]{\textcolor[rgb]{0.38,0.63,0.69}{\textit{{#1}}}}
    \newcommand{\OtherTok}[1]{\textcolor[rgb]{0.00,0.44,0.13}{{#1}}}
    \newcommand{\AlertTok}[1]{\textcolor[rgb]{1.00,0.00,0.00}{\textbf{{#1}}}}
    \newcommand{\FunctionTok}[1]{\textcolor[rgb]{0.02,0.16,0.49}{{#1}}}
    \newcommand{\RegionMarkerTok}[1]{{#1}}
    \newcommand{\ErrorTok}[1]{\textcolor[rgb]{1.00,0.00,0.00}{\textbf{{#1}}}}
    \newcommand{\NormalTok}[1]{{#1}}
    
    % Additional commands for more recent versions of Pandoc
    \newcommand{\ConstantTok}[1]{\textcolor[rgb]{0.53,0.00,0.00}{{#1}}}
    \newcommand{\SpecialCharTok}[1]{\textcolor[rgb]{0.25,0.44,0.63}{{#1}}}
    \newcommand{\VerbatimStringTok}[1]{\textcolor[rgb]{0.25,0.44,0.63}{{#1}}}
    \newcommand{\SpecialStringTok}[1]{\textcolor[rgb]{0.73,0.40,0.53}{{#1}}}
    \newcommand{\ImportTok}[1]{{#1}}
    \newcommand{\DocumentationTok}[1]{\textcolor[rgb]{0.73,0.13,0.13}{\textit{{#1}}}}
    \newcommand{\AnnotationTok}[1]{\textcolor[rgb]{0.38,0.63,0.69}{\textbf{\textit{{#1}}}}}
    \newcommand{\CommentVarTok}[1]{\textcolor[rgb]{0.38,0.63,0.69}{\textbf{\textit{{#1}}}}}
    \newcommand{\VariableTok}[1]{\textcolor[rgb]{0.10,0.09,0.49}{{#1}}}
    \newcommand{\ControlFlowTok}[1]{\textcolor[rgb]{0.00,0.44,0.13}{\textbf{{#1}}}}
    \newcommand{\OperatorTok}[1]{\textcolor[rgb]{0.40,0.40,0.40}{{#1}}}
    \newcommand{\BuiltInTok}[1]{{#1}}
    \newcommand{\ExtensionTok}[1]{{#1}}
    \newcommand{\PreprocessorTok}[1]{\textcolor[rgb]{0.74,0.48,0.00}{{#1}}}
    \newcommand{\AttributeTok}[1]{\textcolor[rgb]{0.49,0.56,0.16}{{#1}}}
    \newcommand{\InformationTok}[1]{\textcolor[rgb]{0.38,0.63,0.69}{\textbf{\textit{{#1}}}}}
    \newcommand{\WarningTok}[1]{\textcolor[rgb]{0.38,0.63,0.69}{\textbf{\textit{{#1}}}}}
    
    
    % Define a nice break command that doesn't care if a line doesn't already
    % exist.
    \def\br{\hspace*{\fill} \\* }
    % Math Jax compatability definitions
    \def\gt{>}
    \def\lt{<}
    % Document parameters
    \title{TP1\_EJ3}
    
    
    

    % Pygments definitions
    
\makeatletter
\def\PY@reset{\let\PY@it=\relax \let\PY@bf=\relax%
    \let\PY@ul=\relax \let\PY@tc=\relax%
    \let\PY@bc=\relax \let\PY@ff=\relax}
\def\PY@tok#1{\csname PY@tok@#1\endcsname}
\def\PY@toks#1+{\ifx\relax#1\empty\else%
    \PY@tok{#1}\expandafter\PY@toks\fi}
\def\PY@do#1{\PY@bc{\PY@tc{\PY@ul{%
    \PY@it{\PY@bf{\PY@ff{#1}}}}}}}
\def\PY#1#2{\PY@reset\PY@toks#1+\relax+\PY@do{#2}}

\expandafter\def\csname PY@tok@w\endcsname{\def\PY@tc##1{\textcolor[rgb]{0.73,0.73,0.73}{##1}}}
\expandafter\def\csname PY@tok@c\endcsname{\let\PY@it=\textit\def\PY@tc##1{\textcolor[rgb]{0.25,0.50,0.50}{##1}}}
\expandafter\def\csname PY@tok@cp\endcsname{\def\PY@tc##1{\textcolor[rgb]{0.74,0.48,0.00}{##1}}}
\expandafter\def\csname PY@tok@k\endcsname{\let\PY@bf=\textbf\def\PY@tc##1{\textcolor[rgb]{0.00,0.50,0.00}{##1}}}
\expandafter\def\csname PY@tok@kp\endcsname{\def\PY@tc##1{\textcolor[rgb]{0.00,0.50,0.00}{##1}}}
\expandafter\def\csname PY@tok@kt\endcsname{\def\PY@tc##1{\textcolor[rgb]{0.69,0.00,0.25}{##1}}}
\expandafter\def\csname PY@tok@o\endcsname{\def\PY@tc##1{\textcolor[rgb]{0.40,0.40,0.40}{##1}}}
\expandafter\def\csname PY@tok@ow\endcsname{\let\PY@bf=\textbf\def\PY@tc##1{\textcolor[rgb]{0.67,0.13,1.00}{##1}}}
\expandafter\def\csname PY@tok@nb\endcsname{\def\PY@tc##1{\textcolor[rgb]{0.00,0.50,0.00}{##1}}}
\expandafter\def\csname PY@tok@nf\endcsname{\def\PY@tc##1{\textcolor[rgb]{0.00,0.00,1.00}{##1}}}
\expandafter\def\csname PY@tok@nc\endcsname{\let\PY@bf=\textbf\def\PY@tc##1{\textcolor[rgb]{0.00,0.00,1.00}{##1}}}
\expandafter\def\csname PY@tok@nn\endcsname{\let\PY@bf=\textbf\def\PY@tc##1{\textcolor[rgb]{0.00,0.00,1.00}{##1}}}
\expandafter\def\csname PY@tok@ne\endcsname{\let\PY@bf=\textbf\def\PY@tc##1{\textcolor[rgb]{0.82,0.25,0.23}{##1}}}
\expandafter\def\csname PY@tok@nv\endcsname{\def\PY@tc##1{\textcolor[rgb]{0.10,0.09,0.49}{##1}}}
\expandafter\def\csname PY@tok@no\endcsname{\def\PY@tc##1{\textcolor[rgb]{0.53,0.00,0.00}{##1}}}
\expandafter\def\csname PY@tok@nl\endcsname{\def\PY@tc##1{\textcolor[rgb]{0.63,0.63,0.00}{##1}}}
\expandafter\def\csname PY@tok@ni\endcsname{\let\PY@bf=\textbf\def\PY@tc##1{\textcolor[rgb]{0.60,0.60,0.60}{##1}}}
\expandafter\def\csname PY@tok@na\endcsname{\def\PY@tc##1{\textcolor[rgb]{0.49,0.56,0.16}{##1}}}
\expandafter\def\csname PY@tok@nt\endcsname{\let\PY@bf=\textbf\def\PY@tc##1{\textcolor[rgb]{0.00,0.50,0.00}{##1}}}
\expandafter\def\csname PY@tok@nd\endcsname{\def\PY@tc##1{\textcolor[rgb]{0.67,0.13,1.00}{##1}}}
\expandafter\def\csname PY@tok@s\endcsname{\def\PY@tc##1{\textcolor[rgb]{0.73,0.13,0.13}{##1}}}
\expandafter\def\csname PY@tok@sd\endcsname{\let\PY@it=\textit\def\PY@tc##1{\textcolor[rgb]{0.73,0.13,0.13}{##1}}}
\expandafter\def\csname PY@tok@si\endcsname{\let\PY@bf=\textbf\def\PY@tc##1{\textcolor[rgb]{0.73,0.40,0.53}{##1}}}
\expandafter\def\csname PY@tok@se\endcsname{\let\PY@bf=\textbf\def\PY@tc##1{\textcolor[rgb]{0.73,0.40,0.13}{##1}}}
\expandafter\def\csname PY@tok@sr\endcsname{\def\PY@tc##1{\textcolor[rgb]{0.73,0.40,0.53}{##1}}}
\expandafter\def\csname PY@tok@ss\endcsname{\def\PY@tc##1{\textcolor[rgb]{0.10,0.09,0.49}{##1}}}
\expandafter\def\csname PY@tok@sx\endcsname{\def\PY@tc##1{\textcolor[rgb]{0.00,0.50,0.00}{##1}}}
\expandafter\def\csname PY@tok@m\endcsname{\def\PY@tc##1{\textcolor[rgb]{0.40,0.40,0.40}{##1}}}
\expandafter\def\csname PY@tok@gh\endcsname{\let\PY@bf=\textbf\def\PY@tc##1{\textcolor[rgb]{0.00,0.00,0.50}{##1}}}
\expandafter\def\csname PY@tok@gu\endcsname{\let\PY@bf=\textbf\def\PY@tc##1{\textcolor[rgb]{0.50,0.00,0.50}{##1}}}
\expandafter\def\csname PY@tok@gd\endcsname{\def\PY@tc##1{\textcolor[rgb]{0.63,0.00,0.00}{##1}}}
\expandafter\def\csname PY@tok@gi\endcsname{\def\PY@tc##1{\textcolor[rgb]{0.00,0.63,0.00}{##1}}}
\expandafter\def\csname PY@tok@gr\endcsname{\def\PY@tc##1{\textcolor[rgb]{1.00,0.00,0.00}{##1}}}
\expandafter\def\csname PY@tok@ge\endcsname{\let\PY@it=\textit}
\expandafter\def\csname PY@tok@gs\endcsname{\let\PY@bf=\textbf}
\expandafter\def\csname PY@tok@gp\endcsname{\let\PY@bf=\textbf\def\PY@tc##1{\textcolor[rgb]{0.00,0.00,0.50}{##1}}}
\expandafter\def\csname PY@tok@go\endcsname{\def\PY@tc##1{\textcolor[rgb]{0.53,0.53,0.53}{##1}}}
\expandafter\def\csname PY@tok@gt\endcsname{\def\PY@tc##1{\textcolor[rgb]{0.00,0.27,0.87}{##1}}}
\expandafter\def\csname PY@tok@err\endcsname{\def\PY@bc##1{\setlength{\fboxsep}{0pt}\fcolorbox[rgb]{1.00,0.00,0.00}{1,1,1}{\strut ##1}}}
\expandafter\def\csname PY@tok@kc\endcsname{\let\PY@bf=\textbf\def\PY@tc##1{\textcolor[rgb]{0.00,0.50,0.00}{##1}}}
\expandafter\def\csname PY@tok@kd\endcsname{\let\PY@bf=\textbf\def\PY@tc##1{\textcolor[rgb]{0.00,0.50,0.00}{##1}}}
\expandafter\def\csname PY@tok@kn\endcsname{\let\PY@bf=\textbf\def\PY@tc##1{\textcolor[rgb]{0.00,0.50,0.00}{##1}}}
\expandafter\def\csname PY@tok@kr\endcsname{\let\PY@bf=\textbf\def\PY@tc##1{\textcolor[rgb]{0.00,0.50,0.00}{##1}}}
\expandafter\def\csname PY@tok@bp\endcsname{\def\PY@tc##1{\textcolor[rgb]{0.00,0.50,0.00}{##1}}}
\expandafter\def\csname PY@tok@fm\endcsname{\def\PY@tc##1{\textcolor[rgb]{0.00,0.00,1.00}{##1}}}
\expandafter\def\csname PY@tok@vc\endcsname{\def\PY@tc##1{\textcolor[rgb]{0.10,0.09,0.49}{##1}}}
\expandafter\def\csname PY@tok@vg\endcsname{\def\PY@tc##1{\textcolor[rgb]{0.10,0.09,0.49}{##1}}}
\expandafter\def\csname PY@tok@vi\endcsname{\def\PY@tc##1{\textcolor[rgb]{0.10,0.09,0.49}{##1}}}
\expandafter\def\csname PY@tok@vm\endcsname{\def\PY@tc##1{\textcolor[rgb]{0.10,0.09,0.49}{##1}}}
\expandafter\def\csname PY@tok@sa\endcsname{\def\PY@tc##1{\textcolor[rgb]{0.73,0.13,0.13}{##1}}}
\expandafter\def\csname PY@tok@sb\endcsname{\def\PY@tc##1{\textcolor[rgb]{0.73,0.13,0.13}{##1}}}
\expandafter\def\csname PY@tok@sc\endcsname{\def\PY@tc##1{\textcolor[rgb]{0.73,0.13,0.13}{##1}}}
\expandafter\def\csname PY@tok@dl\endcsname{\def\PY@tc##1{\textcolor[rgb]{0.73,0.13,0.13}{##1}}}
\expandafter\def\csname PY@tok@s2\endcsname{\def\PY@tc##1{\textcolor[rgb]{0.73,0.13,0.13}{##1}}}
\expandafter\def\csname PY@tok@sh\endcsname{\def\PY@tc##1{\textcolor[rgb]{0.73,0.13,0.13}{##1}}}
\expandafter\def\csname PY@tok@s1\endcsname{\def\PY@tc##1{\textcolor[rgb]{0.73,0.13,0.13}{##1}}}
\expandafter\def\csname PY@tok@mb\endcsname{\def\PY@tc##1{\textcolor[rgb]{0.40,0.40,0.40}{##1}}}
\expandafter\def\csname PY@tok@mf\endcsname{\def\PY@tc##1{\textcolor[rgb]{0.40,0.40,0.40}{##1}}}
\expandafter\def\csname PY@tok@mh\endcsname{\def\PY@tc##1{\textcolor[rgb]{0.40,0.40,0.40}{##1}}}
\expandafter\def\csname PY@tok@mi\endcsname{\def\PY@tc##1{\textcolor[rgb]{0.40,0.40,0.40}{##1}}}
\expandafter\def\csname PY@tok@il\endcsname{\def\PY@tc##1{\textcolor[rgb]{0.40,0.40,0.40}{##1}}}
\expandafter\def\csname PY@tok@mo\endcsname{\def\PY@tc##1{\textcolor[rgb]{0.40,0.40,0.40}{##1}}}
\expandafter\def\csname PY@tok@ch\endcsname{\let\PY@it=\textit\def\PY@tc##1{\textcolor[rgb]{0.25,0.50,0.50}{##1}}}
\expandafter\def\csname PY@tok@cm\endcsname{\let\PY@it=\textit\def\PY@tc##1{\textcolor[rgb]{0.25,0.50,0.50}{##1}}}
\expandafter\def\csname PY@tok@cpf\endcsname{\let\PY@it=\textit\def\PY@tc##1{\textcolor[rgb]{0.25,0.50,0.50}{##1}}}
\expandafter\def\csname PY@tok@c1\endcsname{\let\PY@it=\textit\def\PY@tc##1{\textcolor[rgb]{0.25,0.50,0.50}{##1}}}
\expandafter\def\csname PY@tok@cs\endcsname{\let\PY@it=\textit\def\PY@tc##1{\textcolor[rgb]{0.25,0.50,0.50}{##1}}}

\def\PYZbs{\char`\\}
\def\PYZus{\char`\_}
\def\PYZob{\char`\{}
\def\PYZcb{\char`\}}
\def\PYZca{\char`\^}
\def\PYZam{\char`\&}
\def\PYZlt{\char`\<}
\def\PYZgt{\char`\>}
\def\PYZsh{\char`\#}
\def\PYZpc{\char`\%}
\def\PYZdl{\char`\$}
\def\PYZhy{\char`\-}
\def\PYZsq{\char`\'}
\def\PYZdq{\char`\"}
\def\PYZti{\char`\~}
% for compatibility with earlier versions
\def\PYZat{@}
\def\PYZlb{[}
\def\PYZrb{]}
\makeatother


    % Exact colors from NB
    \definecolor{incolor}{rgb}{0.0, 0.0, 0.5}
    \definecolor{outcolor}{rgb}{0.545, 0.0, 0.0}



    
    % Prevent overflowing lines due to hard-to-break entities
    \sloppy 
    % Setup hyperref package
    \hypersetup{
      breaklinks=true,  % so long urls are correctly broken across lines
      colorlinks=true,
      urlcolor=urlcolor,
      linkcolor=linkcolor,
      citecolor=citecolor,
      }
    % Slightly bigger margins than the latex defaults
    
    \geometry{verbose,tmargin=1in,bmargin=1in,lmargin=1in,rmargin=1in}
    
    

    \begin{document}
    
    
    \maketitle
    
    

    
    \section{EJERCICIO 3:}\label{ejercicio-3}

Consideramos la red as-22july06.gml creada por Mark Newman que contiene
la estructura de los sistemas autónomos de internet relevada a mediados
de 2006.

    \begin{Verbatim}[commandchars=\\\{\}]
{\color{incolor}In [{\color{incolor}4}]:} \PY{c+c1}{\PYZsh{}paquetes }
        \PY{k+kn}{import} \PY{n+nn}{numpy} \PY{k}{as} \PY{n+nn}{np}
        \PY{k+kn}{import} \PY{n+nn}{networkx} \PY{k}{as} \PY{n+nn}{nx}
        \PY{k+kn}{import} \PY{n+nn}{matplotlib}\PY{n+nn}{.}\PY{n+nn}{pylab} \PY{k}{as} \PY{n+nn}{plt}
        \PY{o}{\PYZpc{}}\PY{k}{matplotlib} inline
        \PY{k+kn}{import} \PY{n+nn}{os}
        \PY{c+c1}{\PYZsh{}from random import shuffle}
        \PY{k+kn}{import} \PY{n+nn}{math}
        \PY{k+kn}{import} \PY{n+nn}{pandas} \PY{k}{as} \PY{n+nn}{pd}
        \PY{k+kn}{from} \PY{n+nn}{scipy} \PY{k}{import} \PY{n}{optimize}
\end{Verbatim}


    Generamos el grafo

    \begin{Verbatim}[commandchars=\\\{\}]
{\color{incolor}In [{\color{incolor}5}]:} \PY{n}{G} \PY{o}{=} \PY{n}{nx}\PY{o}{.}\PY{n}{read\PYZus{}gml}\PY{p}{(}\PY{l+s+s1}{\PYZsq{}}\PY{l+s+s1}{as\PYZhy{}22july06.gml}\PY{l+s+s1}{\PYZsq{}}\PY{p}{)}
\end{Verbatim}


    Definimos algunos objetos de interes

    \begin{Verbatim}[commandchars=\\\{\}]
{\color{incolor}In [{\color{incolor}6}]:} \PY{n}{N} \PY{o}{=} \PY{n+nb}{len}\PY{p}{(}\PY{n}{G}\PY{o}{.}\PY{n}{nodes}\PY{p}{(}\PY{p}{)}\PY{p}{)}
        \PY{n}{degree} \PY{o}{=} \PY{p}{[}\PY{n}{degree} \PY{k}{for} \PY{n}{node}\PY{p}{,}\PY{n}{degree} \PY{o+ow}{in} \PY{n}{G}\PY{o}{.}\PY{n}{degree}\PY{p}{(}\PY{p}{)}\PY{p}{]}  \PY{c+c1}{\PYZsh{}a partir de un diccionario nos quedamos con los values en una lista }
        \PY{n}{set\PYZus{}degree} \PY{o}{=} \PY{n+nb}{set}\PY{p}{(}\PY{n}{degree}\PY{p}{)}                         \PY{c+c1}{\PYZsh{}lo hacemos un set para quitarle las repeticiones }
        \PY{n+nb}{len}\PY{p}{(}\PY{n}{degree}\PY{p}{)}\PY{p}{,} \PY{n+nb}{len}\PY{p}{(}\PY{n}{set\PYZus{}degree}\PY{p}{)}                     \PY{c+c1}{\PYZsh{}para tener una idea de cuantas repeticiones hay }
\end{Verbatim}


\begin{Verbatim}[commandchars=\\\{\}]
{\color{outcolor}Out[{\color{outcolor}6}]:} (22963, 161)
\end{Verbatim}
            
    \section{a. BINEADO LINEAL}\label{a.-bineado-lineal}

    \begin{Verbatim}[commandchars=\\\{\}]
{\color{incolor}In [{\color{incolor}7}]:} \PY{c+c1}{\PYZsh{}vamos a recorrer todo el set con k. Por ej: Para k = 1, P1 es un vector donde cada elemento es el grado k = 1 al recorrer degree }
        \PY{c+c1}{\PYZsh{}que tiene las repeticiones. Entonces, el tamaño de ese vector es la cantidad de nodos que tienen grado k = 1. }
        \PY{c+c1}{\PYZsh{}asi es que P es un vector donde cada elemento es la cantidad de nodos con grado k normalizado por la cantidad de nodos total.}
        
        \PY{n}{P} \PY{o}{=} \PY{p}{[}\PY{p}{]}
        
        \PY{k}{for} \PY{n}{k} \PY{o+ow}{in} \PY{n}{set\PYZus{}degree}\PY{p}{:}
            \PY{n}{Pk} \PY{o}{=} \PY{p}{[}\PY{n}{d} \PY{k}{for} \PY{n}{d} \PY{o+ow}{in} \PY{n}{degree} \PY{k}{if} \PY{n}{d}\PY{o}{==}\PY{n}{k}\PY{p}{]}
            \PY{n}{P}\PY{o}{.}\PY{n}{append}\PY{p}{(}\PY{n+nb}{len}\PY{p}{(}\PY{n}{Pk}\PY{p}{)}\PY{o}{/}\PY{n}{N}\PY{p}{)}
\end{Verbatim}


    \begin{Verbatim}[commandchars=\\\{\}]
{\color{incolor}In [{\color{incolor}8}]:} \PY{c+c1}{\PYZsh{}chequeo que si sumo todos los elementos obtengo 1}
        \PY{n+nb}{sum}\PY{p}{(}\PY{n}{P}\PY{p}{)}
\end{Verbatim}


\begin{Verbatim}[commandchars=\\\{\}]
{\color{outcolor}Out[{\color{outcolor}8}]:} 1.0000000000000036
\end{Verbatim}
            
    \begin{Verbatim}[commandchars=\\\{\}]
{\color{incolor}In [{\color{incolor}9}]:} \PY{n}{ls\PYZus{}set\PYZus{}degree} \PY{o}{=} \PY{n+nb}{list}\PY{p}{(}\PY{n}{set\PYZus{}degree}\PY{p}{)}                             \PY{c+c1}{\PYZsh{}lo ponemos como una lista para poder orderlo para que el elemento i\PYZhy{}esimo}
                                                                     \PY{c+c1}{\PYZsh{}se corresponda con el i\PYZhy{}esimo del vector P para plotear }
            
        \PY{n}{dict\PYZus{}distdg} \PY{o}{=} \PY{p}{\PYZob{}}\PY{n}{ls\PYZus{}set\PYZus{}degree}\PY{p}{[}\PY{n}{i}\PY{p}{]}\PY{p}{:}\PY{n}{P}\PY{p}{[}\PY{n}{i}\PY{p}{]} \PY{k}{for} \PY{n}{i} \PY{o+ow}{in} \PY{n+nb}{range}\PY{p}{(}\PY{n+nb}{len}\PY{p}{(}\PY{n}{P}\PY{p}{)}\PY{p}{)}\PY{p}{\PYZcb{}} \PY{c+c1}{\PYZsh{}armamos un diccionario donde la key es el grado y el value la probabilidad }
\end{Verbatim}


    \begin{Verbatim}[commandchars=\\\{\}]
{\color{incolor}In [{\color{incolor}10}]:} \PY{n}{font} \PY{o}{=} \PY{p}{\PYZob{}}\PY{l+s+s1}{\PYZsq{}}\PY{l+s+s1}{family}\PY{l+s+s1}{\PYZsq{}} \PY{p}{:} \PY{l+s+s1}{\PYZsq{}}\PY{l+s+s1}{Helvetica}\PY{l+s+s1}{\PYZsq{}}\PY{p}{,}
                 \PY{l+s+s1}{\PYZsq{}}\PY{l+s+s1}{weight}\PY{l+s+s1}{\PYZsq{}} \PY{p}{:} \PY{l+s+s1}{\PYZsq{}}\PY{l+s+s1}{bold}\PY{l+s+s1}{\PYZsq{}}\PY{p}{,}
                 \PY{l+s+s1}{\PYZsq{}}\PY{l+s+s1}{size}\PY{l+s+s1}{\PYZsq{}}   \PY{p}{:} \PY{l+m+mi}{20}\PY{p}{\PYZcb{}}
         
         \PY{n}{plt}\PY{o}{.}\PY{n}{rc}\PY{p}{(}\PY{l+s+s1}{\PYZsq{}}\PY{l+s+s1}{font}\PY{l+s+s1}{\PYZsq{}}\PY{p}{,} \PY{o}{*}\PY{o}{*}\PY{n}{font}\PY{p}{)}
         
         \PY{n}{f} \PY{o}{=} \PY{n}{plt}\PY{o}{.}\PY{n}{figure}\PY{p}{(}\PY{n}{figsize} \PY{o}{=} \PY{p}{(}\PY{l+m+mi}{10}\PY{p}{,}\PY{l+m+mi}{12}\PY{p}{)}\PY{p}{)}
         \PY{n}{f}\PY{o}{.}\PY{n}{subplots\PYZus{}adjust}\PY{p}{(}\PY{n}{hspace}\PY{o}{=}\PY{l+m+mf}{0.4}\PY{p}{,} \PY{n}{wspace}\PY{o}{=}\PY{l+m+mf}{0.4}\PY{p}{)}
         \PY{n}{f}\PY{o}{.}\PY{n}{suptitle}\PY{p}{(}\PY{l+s+s1}{\PYZsq{}}\PY{l+s+s1}{Bineados lineales}\PY{l+s+s1}{\PYZsq{}}\PY{p}{,}\PY{n}{fontweight} \PY{o}{=} \PY{l+s+s2}{\PYZdq{}}\PY{l+s+s2}{bold}\PY{l+s+s2}{\PYZdq{}}\PY{p}{,} \PY{n}{size} \PY{o}{=} \PY{l+m+mi}{30}\PY{p}{)} 
         \PY{n}{sub1} \PY{o}{=} \PY{n}{f}\PY{o}{.}\PY{n}{add\PYZus{}subplot}\PY{p}{(}\PY{l+m+mi}{221}\PY{p}{)}
         \PY{n}{plt}\PY{o}{.}\PY{n}{plot}\PY{p}{(}\PY{n}{ls\PYZus{}set\PYZus{}degree}\PY{p}{,}\PY{n}{P}\PY{p}{,}\PY{l+s+s1}{\PYZsq{}}\PY{l+s+s1}{.}\PY{l+s+s1}{\PYZsq{}}\PY{p}{)}
         \PY{c+c1}{\PYZsh{}sub1.set\PYZus{}title(\PYZsq{}Ambas escalas lineales\PYZsq{}, size = 20)}
         \PY{c+c1}{\PYZsh{}sub1.set\PYZus{}xlabel(\PYZdq{}k\PYZdq{})}
         \PY{c+c1}{\PYZsh{}sub1.set\PYZus{}ylabel(\PYZdq{}\PYZdl{}P\PYZus{}k\PYZdl{}\PYZdq{})}
         \PY{n}{sub2} \PY{o}{=} \PY{n}{f}\PY{o}{.}\PY{n}{add\PYZus{}subplot}\PY{p}{(}\PY{l+m+mi}{222}\PY{p}{)}
         \PY{n}{plt}\PY{o}{.}\PY{n}{loglog}\PY{p}{(}\PY{n}{ls\PYZus{}set\PYZus{}degree}\PY{p}{,}\PY{n}{P}\PY{p}{,}\PY{l+s+s1}{\PYZsq{}}\PY{l+s+s1}{.}\PY{l+s+s1}{\PYZsq{}}\PY{p}{)}
         \PY{c+c1}{\PYZsh{}sub2.set\PYZus{}title(\PYZsq{}Ambas escalas logaritmicas\PYZsq{}, size = 20)}
         \PY{c+c1}{\PYZsh{}sub2.set\PYZus{}xlabel(\PYZdq{}k\PYZdq{})}
         \PY{c+c1}{\PYZsh{}sub2.set\PYZus{}ylabel(\PYZdq{}\PYZdl{}P\PYZus{}k\PYZdl{}\PYZdq{})}
         \PY{n}{sub3} \PY{o}{=} \PY{n}{f}\PY{o}{.}\PY{n}{add\PYZus{}subplot}\PY{p}{(}\PY{l+m+mi}{223}\PY{p}{)}
         \PY{n}{plt}\PY{o}{.}\PY{n}{semilogy}\PY{p}{(}\PY{n}{ls\PYZus{}set\PYZus{}degree}\PY{p}{,}\PY{n}{P}\PY{p}{,}\PY{l+s+s1}{\PYZsq{}}\PY{l+s+s1}{.}\PY{l+s+s1}{\PYZsq{}}\PY{p}{)}
         \PY{c+c1}{\PYZsh{}sub3.set\PYZus{}title(\PYZsq{}Escala lineal en x y logaritmica en y\PYZsq{}, size = 20)}
         \PY{n}{sub4} \PY{o}{=} \PY{n}{f}\PY{o}{.}\PY{n}{add\PYZus{}subplot}\PY{p}{(}\PY{l+m+mi}{224}\PY{p}{)}
         \PY{n}{plt}\PY{o}{.}\PY{n}{semilogx}\PY{p}{(}\PY{n}{ls\PYZus{}set\PYZus{}degree}\PY{p}{,}\PY{n}{P}\PY{p}{,}\PY{l+s+s1}{\PYZsq{}}\PY{l+s+s1}{.}\PY{l+s+s1}{\PYZsq{}}\PY{p}{)}
         \PY{c+c1}{\PYZsh{}sub3.set\PYZus{}xlabel(\PYZdq{}k\PYZdq{})}
         \PY{c+c1}{\PYZsh{}sub3.set\PYZus{}ylabel(\PYZdq{}\PYZdl{}P\PYZus{}k\PYZdl{}\PYZdq{})}
         \PY{c+c1}{\PYZsh{}\PYZsh{}\PYZsh{}\PYZsh{}\PYZsh{}\PYZsh{}\PYZsh{}\PYZsh{}\PYZsh{}\PYZsh{}\PYZsh{}\PYZsh{}\PYZsh{}\PYZsh{}\PYZsh{}\PYZsh{}\PYZsh{}\PYZsh{}\PYZsh{}\PYZsh{}\PYZsh{}\PYZsh{}\PYZsh{}\PYZsh{}\PYZsh{}\PYZsh{}\PYZsh{}\PYZsh{}\PYZsh{}\PYZsh{}\PYZsh{}\PYZsh{}\PYZsh{}\PYZsh{}\PYZsh{}}
         \PY{n}{sub1}\PY{o}{.}\PY{n}{set\PYZus{}title}\PY{p}{(}\PY{l+s+s1}{\PYZsq{}}\PY{l+s+s1}{Ambas escalas lineales}\PY{l+s+s1}{\PYZsq{}}\PY{p}{,} \PY{n}{size} \PY{o}{=} \PY{l+m+mi}{20}\PY{p}{)}
         \PY{n}{sub2}\PY{o}{.}\PY{n}{set\PYZus{}title}\PY{p}{(}\PY{l+s+s1}{\PYZsq{}}\PY{l+s+s1}{Ambas escalas log}\PY{l+s+s1}{\PYZsq{}}\PY{p}{,} \PY{n}{size} \PY{o}{=} \PY{l+m+mi}{20}\PY{p}{)}
         \PY{n}{sub3}\PY{o}{.}\PY{n}{set\PYZus{}title}\PY{p}{(}\PY{l+s+s1}{\PYZsq{}}\PY{l+s+s1}{Escala lineal en x y log en y}\PY{l+s+s1}{\PYZsq{}}\PY{p}{,} \PY{n}{size} \PY{o}{=} \PY{l+m+mi}{20}\PY{p}{)}
         \PY{n}{sub4}\PY{o}{.}\PY{n}{set\PYZus{}title}\PY{p}{(}\PY{l+s+s1}{\PYZsq{}}\PY{l+s+s1}{Escala log en x y lineal en y}\PY{l+s+s1}{\PYZsq{}}\PY{p}{,} \PY{n}{size} \PY{o}{=} \PY{l+m+mi}{20}\PY{p}{)}
         \PY{n}{sub1}\PY{o}{.}\PY{n}{set\PYZus{}xlabel}\PY{p}{(}\PY{l+s+s2}{\PYZdq{}}\PY{l+s+s2}{k}\PY{l+s+s2}{\PYZdq{}}\PY{p}{)}
         \PY{n}{sub1}\PY{o}{.}\PY{n}{set\PYZus{}ylabel}\PY{p}{(}\PY{l+s+s2}{\PYZdq{}}\PY{l+s+s2}{\PYZdl{}P\PYZus{}k\PYZdl{}}\PY{l+s+s2}{\PYZdq{}}\PY{p}{)}
         \PY{n}{sub2}\PY{o}{.}\PY{n}{set\PYZus{}xlabel}\PY{p}{(}\PY{l+s+s2}{\PYZdq{}}\PY{l+s+s2}{k}\PY{l+s+s2}{\PYZdq{}}\PY{p}{)}
         \PY{n}{sub2}\PY{o}{.}\PY{n}{set\PYZus{}ylabel}\PY{p}{(}\PY{l+s+s2}{\PYZdq{}}\PY{l+s+s2}{\PYZdl{}P\PYZus{}k\PYZdl{}}\PY{l+s+s2}{\PYZdq{}}\PY{p}{)}
         \PY{n}{sub3}\PY{o}{.}\PY{n}{set\PYZus{}xlabel}\PY{p}{(}\PY{l+s+s2}{\PYZdq{}}\PY{l+s+s2}{k}\PY{l+s+s2}{\PYZdq{}}\PY{p}{)}
         \PY{n}{sub3}\PY{o}{.}\PY{n}{set\PYZus{}ylabel}\PY{p}{(}\PY{l+s+s2}{\PYZdq{}}\PY{l+s+s2}{\PYZdl{}P\PYZus{}k\PYZdl{}}\PY{l+s+s2}{\PYZdq{}}\PY{p}{)}
         \PY{n}{sub4}\PY{o}{.}\PY{n}{set\PYZus{}xlabel}\PY{p}{(}\PY{l+s+s2}{\PYZdq{}}\PY{l+s+s2}{k}\PY{l+s+s2}{\PYZdq{}}\PY{p}{)}
         \PY{n}{sub4}\PY{o}{.}\PY{n}{set\PYZus{}ylabel}\PY{p}{(}\PY{l+s+s2}{\PYZdq{}}\PY{l+s+s2}{\PYZdl{}P\PYZus{}k\PYZdl{}}\PY{l+s+s2}{\PYZdq{}}\PY{p}{)}
\end{Verbatim}


\begin{Verbatim}[commandchars=\\\{\}]
{\color{outcolor}Out[{\color{outcolor}10}]:} Text(0,0.5,'\$P\_k\$')
\end{Verbatim}
            
    \begin{Verbatim}[commandchars=\\\{\}]
C:\textbackslash{}Users\textbackslash{}Elizabeth\textbackslash{}Anaconda3\textbackslash{}lib\textbackslash{}site-packages\textbackslash{}matplotlib\textbackslash{}font\_manager.py:1328: UserWarning: findfont: Font family ['Helvetica'] not found. Falling back to DejaVu Sans
  (prop.get\_family(), self.defaultFamily[fontext]))

    \end{Verbatim}

    \begin{center}
    \adjustimage{max size={0.9\linewidth}{0.9\paperheight}}{output_10_2.png}
    \end{center}
    { \hspace*{\fill} \\}
    
    Dentro de los graficos hechos para el bineado lineal, en el que ambas
escalas son logaritmicas se puede ver notoriamente que existe un rango
de k tal que la distribucion es lineal decreciente. Sin embargo, existe
un problema. A medida que va creciendo k se ve que existen varios k para
el mismo valor de \(P_k\) (apilamiento) produciendo que los puntos se
organicen como lineas horizontales.

    \section{a. BINEADO LOGARITMICO EN BASE
2}\label{a.-bineado-logaritmico-en-base-2}

\(P_{k_i} = \frac{N_i}{\Delta b_i} = \frac{N_i}{2^i}\)

\(b = [0, 2^0, 2^1, 2^2, ... ,2^{i_{max}}], b_i = 2^i\)

    \begin{Verbatim}[commandchars=\\\{\}]
{\color{incolor}In [{\color{incolor}11}]:} \PY{c+c1}{\PYZsh{}buscamos el i\PYZus{}max que tiene nuestra distribucion para hacer la serie }
         \PY{n}{imax} \PY{o}{=} \PY{n+nb}{int}\PY{p}{(}\PY{n}{math}\PY{o}{.}\PY{n}{log}\PY{p}{(}\PY{n}{ls\PYZus{}set\PYZus{}degree}\PY{p}{[}\PY{o}{\PYZhy{}}\PY{l+m+mi}{1}\PY{p}{]}\PY{p}{,}\PY{l+m+mi}{2}\PY{p}{)}\PY{p}{)}\PY{o}{+}\PY{l+m+mi}{1}
         \PY{n+nb}{print}\PY{p}{(}\PY{n}{imax}\PY{p}{)}
         \PY{n+nb}{print}\PY{p}{(}\PY{l+m+mi}{2}\PY{o}{*}\PY{o}{*}\PY{l+m+mi}{9}\PY{p}{)}
\end{Verbatim}


    \begin{Verbatim}[commandchars=\\\{\}]
9
512

    \end{Verbatim}

    \begin{Verbatim}[commandchars=\\\{\}]
{\color{incolor}In [{\color{incolor}12}]:} \PY{c+c1}{\PYZsh{}armamos la serie b }
         \PY{n}{serie} \PY{o}{=} \PY{p}{[}\PY{p}{]}
         \PY{n}{serie}\PY{o}{.}\PY{n}{append}\PY{p}{(}\PY{l+m+mi}{0}\PY{p}{)}
         
         \PY{k}{for} \PY{n}{k} \PY{o+ow}{in} \PY{n+nb}{range}\PY{p}{(}\PY{n}{imax}\PY{p}{)}\PY{p}{:}
             \PY{n}{serie}\PY{o}{.}\PY{n}{append}\PY{p}{(}\PY{n}{serie}\PY{p}{[}\PY{n}{k}\PY{p}{]}\PY{o}{+}\PY{l+m+mi}{2}\PY{o}{*}\PY{o}{*}\PY{n}{k}\PY{p}{)}
             
         \PY{n}{serie}
\end{Verbatim}


\begin{Verbatim}[commandchars=\\\{\}]
{\color{outcolor}Out[{\color{outcolor}12}]:} [0, 1, 3, 7, 15, 31, 63, 127, 255, 511]
\end{Verbatim}
            
    \begin{Verbatim}[commandchars=\\\{\}]
{\color{incolor}In [{\color{incolor}13}]:} \PY{c+c1}{\PYZsh{}Iteramos sobre cada elemento de la serie y para plog iteramos sobre la lista de los grados y vamos armando un vector donde }
         \PY{c+c1}{\PYZsh{}cada elemento es el grado. Va a haber un vector para cada elemento de la serie donde sus elementos van a ser el grado para }
         \PY{c+c1}{\PYZsh{}cada uno de los nodos que cumpla que el grado este contenido entre dos elementos sucesivos de la serie. }
         \PY{c+c1}{\PYZsh{}Entonces, Plog va guardando el tamano de cada uno de esos vectores (cantidad de nodos con grado comprendido en ese bin) normalizado}
         \PY{c+c1}{\PYZsh{}por la cantidad total de nodos. }
         
         \PY{n}{Plog} \PY{o}{=} \PY{p}{[}\PY{p}{]}
         
         \PY{k}{for} \PY{n}{i} \PY{o+ow}{in} \PY{n+nb}{range}\PY{p}{(}\PY{n+nb}{len}\PY{p}{(}\PY{n}{serie}\PY{p}{)}\PY{p}{)}\PY{p}{:}
             \PY{n}{plog} \PY{o}{=} \PY{p}{[}\PY{n}{d} \PY{k}{for} \PY{n}{d} \PY{o+ow}{in} \PY{n}{degree} \PY{k}{if} \PY{n}{serie}\PY{p}{[}\PY{n}{i}\PY{p}{]} \PY{o}{\PYZlt{}} \PY{n}{d} \PY{o}{\PYZlt{}}\PY{o}{=} \PY{n}{serie}\PY{p}{[}\PY{n}{i}\PY{p}{]} \PY{o}{+} \PY{l+m+mi}{2}\PY{o}{*}\PY{o}{*}\PY{n}{i}\PY{p}{]} 
             \PY{n}{Plog}\PY{o}{.}\PY{n}{append}\PY{p}{(}\PY{n+nb}{len}\PY{p}{(}\PY{n}{plog}\PY{p}{)}\PY{o}{/}\PY{n}{N}\PY{p}{)} 
             
         \PY{n}{Plog}
\end{Verbatim}


\begin{Verbatim}[commandchars=\\\{\}]
{\color{outcolor}Out[{\color{outcolor}13}]:} [0.34141880416321907,
          0.5190523886251797,
          0.08134825588990985,
          0.03165962635544136,
          0.014458041196707747,
          0.006314505944345251,
          0.003048382180028742,
          0.0012193528720114968,
          0.0008274180202935157,
          0.0003919348517179811]
\end{Verbatim}
            
    \begin{Verbatim}[commandchars=\\\{\}]
{\color{incolor}In [{\color{incolor}14}]:} \PY{c+c1}{\PYZsh{}queremos ver que esta normalizada la distribucion.}
         \PY{n+nb}{sum}\PY{p}{(}\PY{n}{Plog}\PY{p}{)}
\end{Verbatim}


\begin{Verbatim}[commandchars=\\\{\}]
{\color{outcolor}Out[{\color{outcolor}14}]:} 0.9997387100988548
\end{Verbatim}
            
    \begin{Verbatim}[commandchars=\\\{\}]
{\color{incolor}In [{\color{incolor}15}]:} \PY{n}{font} \PY{o}{=} \PY{p}{\PYZob{}}\PY{l+s+s1}{\PYZsq{}}\PY{l+s+s1}{family}\PY{l+s+s1}{\PYZsq{}} \PY{p}{:} \PY{l+s+s1}{\PYZsq{}}\PY{l+s+s1}{Helvetica}\PY{l+s+s1}{\PYZsq{}}\PY{p}{,}
                 \PY{l+s+s1}{\PYZsq{}}\PY{l+s+s1}{weight}\PY{l+s+s1}{\PYZsq{}} \PY{p}{:} \PY{l+s+s1}{\PYZsq{}}\PY{l+s+s1}{bold}\PY{l+s+s1}{\PYZsq{}}\PY{p}{,}
                 \PY{l+s+s1}{\PYZsq{}}\PY{l+s+s1}{size}\PY{l+s+s1}{\PYZsq{}}   \PY{p}{:} \PY{l+m+mi}{20}\PY{p}{\PYZcb{}}
         
         \PY{n}{plt}\PY{o}{.}\PY{n}{rc}\PY{p}{(}\PY{l+s+s1}{\PYZsq{}}\PY{l+s+s1}{font}\PY{l+s+s1}{\PYZsq{}}\PY{p}{,} \PY{o}{*}\PY{o}{*}\PY{n}{font}\PY{p}{)}
         
         \PY{n}{f} \PY{o}{=} \PY{n}{plt}\PY{o}{.}\PY{n}{figure}\PY{p}{(}\PY{n}{figsize} \PY{o}{=} \PY{p}{(}\PY{l+m+mi}{10}\PY{p}{,}\PY{l+m+mi}{12}\PY{p}{)}\PY{p}{)}
         \PY{n}{f}\PY{o}{.}\PY{n}{subplots\PYZus{}adjust}\PY{p}{(}\PY{n}{hspace}\PY{o}{=}\PY{l+m+mf}{0.4}\PY{p}{,} \PY{n}{wspace}\PY{o}{=}\PY{l+m+mf}{0.4}\PY{p}{)}
         \PY{n}{f}\PY{o}{.}\PY{n}{suptitle}\PY{p}{(}\PY{l+s+s1}{\PYZsq{}}\PY{l+s+s1}{Bineados logaritmicos base 2}\PY{l+s+s1}{\PYZsq{}}\PY{p}{,}\PY{n}{fontweight} \PY{o}{=} \PY{l+s+s2}{\PYZdq{}}\PY{l+s+s2}{bold}\PY{l+s+s2}{\PYZdq{}}\PY{p}{,} \PY{n}{size} \PY{o}{=} \PY{l+m+mi}{30}\PY{p}{)} 
         \PY{n}{sub1} \PY{o}{=} \PY{n}{f}\PY{o}{.}\PY{n}{add\PYZus{}subplot}\PY{p}{(}\PY{l+m+mi}{221}\PY{p}{)}
         \PY{n}{plt}\PY{o}{.}\PY{n}{plot}\PY{p}{(}\PY{n}{serie}\PY{p}{,}\PY{n}{Plog}\PY{p}{,}\PY{l+s+s1}{\PYZsq{}}\PY{l+s+s1}{.}\PY{l+s+s1}{\PYZsq{}}\PY{p}{)}
         \PY{n}{sub2} \PY{o}{=} \PY{n}{f}\PY{o}{.}\PY{n}{add\PYZus{}subplot}\PY{p}{(}\PY{l+m+mi}{222}\PY{p}{)}
         \PY{n}{plt}\PY{o}{.}\PY{n}{loglog}\PY{p}{(}\PY{n}{serie}\PY{p}{,}\PY{n}{Plog}\PY{p}{,}\PY{l+s+s1}{\PYZsq{}}\PY{l+s+s1}{.}\PY{l+s+s1}{\PYZsq{}}\PY{p}{)}
         \PY{n}{sub3} \PY{o}{=} \PY{n}{f}\PY{o}{.}\PY{n}{add\PYZus{}subplot}\PY{p}{(}\PY{l+m+mi}{223}\PY{p}{)}
         \PY{n}{plt}\PY{o}{.}\PY{n}{semilogy}\PY{p}{(}\PY{n}{serie}\PY{p}{,}\PY{n}{Plog}\PY{p}{,}\PY{l+s+s1}{\PYZsq{}}\PY{l+s+s1}{.}\PY{l+s+s1}{\PYZsq{}}\PY{p}{)}
         \PY{n}{sub4} \PY{o}{=} \PY{n}{f}\PY{o}{.}\PY{n}{add\PYZus{}subplot}\PY{p}{(}\PY{l+m+mi}{224}\PY{p}{)}
         \PY{n}{plt}\PY{o}{.}\PY{n}{semilogx}\PY{p}{(}\PY{n}{serie}\PY{p}{,}\PY{n}{Plog}\PY{p}{,}\PY{l+s+s1}{\PYZsq{}}\PY{l+s+s1}{.}\PY{l+s+s1}{\PYZsq{}}\PY{p}{)}
         \PY{c+c1}{\PYZsh{}\PYZsh{}\PYZsh{}\PYZsh{}\PYZsh{}\PYZsh{}\PYZsh{}\PYZsh{}\PYZsh{}\PYZsh{}\PYZsh{}\PYZsh{}\PYZsh{}\PYZsh{}\PYZsh{}\PYZsh{}\PYZsh{}\PYZsh{}\PYZsh{}\PYZsh{}\PYZsh{}\PYZsh{}\PYZsh{}\PYZsh{}\PYZsh{}\PYZsh{}\PYZsh{}\PYZsh{}\PYZsh{}\PYZsh{}\PYZsh{}\PYZsh{}\PYZsh{}\PYZsh{}\PYZsh{}}
         \PY{n}{sub1}\PY{o}{.}\PY{n}{set\PYZus{}title}\PY{p}{(}\PY{l+s+s1}{\PYZsq{}}\PY{l+s+s1}{Ambas escalas lineales}\PY{l+s+s1}{\PYZsq{}}\PY{p}{,} \PY{n}{size} \PY{o}{=} \PY{l+m+mi}{20}\PY{p}{)}
         \PY{n}{sub2}\PY{o}{.}\PY{n}{set\PYZus{}title}\PY{p}{(}\PY{l+s+s1}{\PYZsq{}}\PY{l+s+s1}{Ambas escalas log}\PY{l+s+s1}{\PYZsq{}}\PY{p}{,} \PY{n}{size} \PY{o}{=} \PY{l+m+mi}{20}\PY{p}{)}
         \PY{n}{sub3}\PY{o}{.}\PY{n}{set\PYZus{}title}\PY{p}{(}\PY{l+s+s1}{\PYZsq{}}\PY{l+s+s1}{Escala lineal en x y log en y}\PY{l+s+s1}{\PYZsq{}}\PY{p}{,} \PY{n}{size} \PY{o}{=} \PY{l+m+mi}{20}\PY{p}{)}
         \PY{n}{sub4}\PY{o}{.}\PY{n}{set\PYZus{}title}\PY{p}{(}\PY{l+s+s1}{\PYZsq{}}\PY{l+s+s1}{Escala log en x y lineal en y}\PY{l+s+s1}{\PYZsq{}}\PY{p}{,} \PY{n}{size} \PY{o}{=} \PY{l+m+mi}{20}\PY{p}{)}
         \PY{n}{sub1}\PY{o}{.}\PY{n}{set\PYZus{}xlabel}\PY{p}{(}\PY{l+s+s2}{\PYZdq{}}\PY{l+s+s2}{k}\PY{l+s+s2}{\PYZdq{}}\PY{p}{)}
         \PY{n}{sub1}\PY{o}{.}\PY{n}{set\PYZus{}ylabel}\PY{p}{(}\PY{l+s+s2}{\PYZdq{}}\PY{l+s+s2}{\PYZdl{}P\PYZus{}k\PYZdl{}}\PY{l+s+s2}{\PYZdq{}}\PY{p}{)}
         \PY{n}{sub2}\PY{o}{.}\PY{n}{set\PYZus{}xlabel}\PY{p}{(}\PY{l+s+s2}{\PYZdq{}}\PY{l+s+s2}{k}\PY{l+s+s2}{\PYZdq{}}\PY{p}{)}
         \PY{n}{sub2}\PY{o}{.}\PY{n}{set\PYZus{}ylabel}\PY{p}{(}\PY{l+s+s2}{\PYZdq{}}\PY{l+s+s2}{\PYZdl{}P\PYZus{}k\PYZdl{}}\PY{l+s+s2}{\PYZdq{}}\PY{p}{)}
         \PY{n}{sub3}\PY{o}{.}\PY{n}{set\PYZus{}xlabel}\PY{p}{(}\PY{l+s+s2}{\PYZdq{}}\PY{l+s+s2}{k}\PY{l+s+s2}{\PYZdq{}}\PY{p}{)}
         \PY{n}{sub3}\PY{o}{.}\PY{n}{set\PYZus{}ylabel}\PY{p}{(}\PY{l+s+s2}{\PYZdq{}}\PY{l+s+s2}{\PYZdl{}P\PYZus{}k\PYZdl{}}\PY{l+s+s2}{\PYZdq{}}\PY{p}{)}
         \PY{n}{sub4}\PY{o}{.}\PY{n}{set\PYZus{}xlabel}\PY{p}{(}\PY{l+s+s2}{\PYZdq{}}\PY{l+s+s2}{k}\PY{l+s+s2}{\PYZdq{}}\PY{p}{)}
         \PY{n}{sub4}\PY{o}{.}\PY{n}{set\PYZus{}ylabel}\PY{p}{(}\PY{l+s+s2}{\PYZdq{}}\PY{l+s+s2}{\PYZdl{}P\PYZus{}k\PYZdl{}}\PY{l+s+s2}{\PYZdq{}}\PY{p}{)}
\end{Verbatim}


\begin{Verbatim}[commandchars=\\\{\}]
{\color{outcolor}Out[{\color{outcolor}15}]:} Text(0,0.5,'\$P\_k\$')
\end{Verbatim}
            
    \begin{Verbatim}[commandchars=\\\{\}]
C:\textbackslash{}Users\textbackslash{}Elizabeth\textbackslash{}Anaconda3\textbackslash{}lib\textbackslash{}site-packages\textbackslash{}matplotlib\textbackslash{}font\_manager.py:1328: UserWarning: findfont: Font family ['Helvetica'] not found. Falling back to DejaVu Sans
  (prop.get\_family(), self.defaultFamily[fontext]))

    \end{Verbatim}

    \begin{center}
    \adjustimage{max size={0.9\linewidth}{0.9\paperheight}}{output_17_2.png}
    \end{center}
    { \hspace*{\fill} \\}
    
    Dentro de los graficos hechos para el bineado logaritmico de base 2, en
el que ambas escalas son logaritmicas se puede ver notoriamente que
existe un rango de k tal que la distribucion es lineal.

    \begin{Verbatim}[commandchars=\\\{\}]
{\color{incolor}In [{\color{incolor}16}]:} \PY{n}{font} \PY{o}{=} \PY{p}{\PYZob{}}\PY{l+s+s1}{\PYZsq{}}\PY{l+s+s1}{family}\PY{l+s+s1}{\PYZsq{}} \PY{p}{:} \PY{l+s+s1}{\PYZsq{}}\PY{l+s+s1}{Helvetica}\PY{l+s+s1}{\PYZsq{}}\PY{p}{,}
                 \PY{l+s+s1}{\PYZsq{}}\PY{l+s+s1}{weight}\PY{l+s+s1}{\PYZsq{}} \PY{p}{:} \PY{l+s+s1}{\PYZsq{}}\PY{l+s+s1}{bold}\PY{l+s+s1}{\PYZsq{}}\PY{p}{,}
                 \PY{l+s+s1}{\PYZsq{}}\PY{l+s+s1}{size}\PY{l+s+s1}{\PYZsq{}}   \PY{p}{:} \PY{l+m+mi}{20}\PY{p}{\PYZcb{}}
         
         \PY{n}{plt}\PY{o}{.}\PY{n}{rc}\PY{p}{(}\PY{l+s+s1}{\PYZsq{}}\PY{l+s+s1}{font}\PY{l+s+s1}{\PYZsq{}}\PY{p}{,} \PY{o}{*}\PY{o}{*}\PY{n}{font}\PY{p}{)}
         
         \PY{n}{f} \PY{o}{=} \PY{n}{plt}\PY{o}{.}\PY{n}{figure}\PY{p}{(}\PY{n}{figsize} \PY{o}{=} \PY{p}{(}\PY{l+m+mi}{10}\PY{p}{,}\PY{l+m+mi}{12}\PY{p}{)}\PY{p}{)}
         \PY{n}{f}\PY{o}{.}\PY{n}{subplots\PYZus{}adjust}\PY{p}{(}\PY{n}{hspace}\PY{o}{=}\PY{l+m+mf}{0.4}\PY{p}{,} \PY{n}{wspace}\PY{o}{=}\PY{l+m+mf}{0.4}\PY{p}{)}
         \PY{n}{f}\PY{o}{.}\PY{n}{suptitle}\PY{p}{(}\PY{l+s+s1}{\PYZsq{}}\PY{l+s+s1}{Comparación de bineados en ambas escalas log}\PY{l+s+s1}{\PYZsq{}}\PY{p}{,}\PY{n}{fontweight} \PY{o}{=} \PY{l+s+s2}{\PYZdq{}}\PY{l+s+s2}{bold}\PY{l+s+s2}{\PYZdq{}}\PY{p}{,} \PY{n}{size} \PY{o}{=} \PY{l+m+mi}{30}\PY{p}{)} 
         \PY{n}{sub1} \PY{o}{=} \PY{n}{f}\PY{o}{.}\PY{n}{add\PYZus{}subplot}\PY{p}{(}\PY{l+m+mi}{221}\PY{p}{)}
         \PY{n}{plt}\PY{o}{.}\PY{n}{loglog}\PY{p}{(}\PY{n}{serie}\PY{p}{,}\PY{n}{Plog}\PY{p}{,}\PY{l+s+s1}{\PYZsq{}}\PY{l+s+s1}{.}\PY{l+s+s1}{\PYZsq{}}\PY{p}{)}
         \PY{n}{sub2} \PY{o}{=} \PY{n}{f}\PY{o}{.}\PY{n}{add\PYZus{}subplot}\PY{p}{(}\PY{l+m+mi}{222}\PY{p}{)}
         \PY{n}{plt}\PY{o}{.}\PY{n}{loglog}\PY{p}{(}\PY{n}{ls\PYZus{}set\PYZus{}degree}\PY{p}{,}\PY{n}{P}\PY{p}{,}\PY{l+s+s1}{\PYZsq{}}\PY{l+s+s1}{.}\PY{l+s+s1}{\PYZsq{}}\PY{p}{)}
         \PY{n}{sub1}\PY{o}{.}\PY{n}{set\PYZus{}title}\PY{p}{(}\PY{l+s+s1}{\PYZsq{}}\PY{l+s+s1}{Bineado logarítmico base 2}\PY{l+s+s1}{\PYZsq{}}\PY{p}{,} \PY{n}{size} \PY{o}{=} \PY{l+m+mi}{20}\PY{p}{)}
         \PY{n}{sub2}\PY{o}{.}\PY{n}{set\PYZus{}title}\PY{p}{(}\PY{l+s+s1}{\PYZsq{}}\PY{l+s+s1}{Bineado lineal}\PY{l+s+s1}{\PYZsq{}}\PY{p}{,} \PY{n}{size} \PY{o}{=} \PY{l+m+mi}{20}\PY{p}{)}
         \PY{n}{sub1}\PY{o}{.}\PY{n}{set\PYZus{}xlabel}\PY{p}{(}\PY{l+s+s2}{\PYZdq{}}\PY{l+s+s2}{k}\PY{l+s+s2}{\PYZdq{}}\PY{p}{)}
         \PY{n}{sub1}\PY{o}{.}\PY{n}{set\PYZus{}ylabel}\PY{p}{(}\PY{l+s+s2}{\PYZdq{}}\PY{l+s+s2}{\PYZdl{}P\PYZus{}k\PYZdl{}}\PY{l+s+s2}{\PYZdq{}}\PY{p}{)}
         \PY{n}{sub2}\PY{o}{.}\PY{n}{set\PYZus{}xlabel}\PY{p}{(}\PY{l+s+s2}{\PYZdq{}}\PY{l+s+s2}{k}\PY{l+s+s2}{\PYZdq{}}\PY{p}{)}
         \PY{n}{sub2}\PY{o}{.}\PY{n}{set\PYZus{}ylabel}\PY{p}{(}\PY{l+s+s2}{\PYZdq{}}\PY{l+s+s2}{\PYZdl{}P\PYZus{}k\PYZdl{}}\PY{l+s+s2}{\PYZdq{}}\PY{p}{)}
\end{Verbatim}


\begin{Verbatim}[commandchars=\\\{\}]
{\color{outcolor}Out[{\color{outcolor}16}]:} Text(0,0.5,'\$P\_k\$')
\end{Verbatim}
            
    \begin{Verbatim}[commandchars=\\\{\}]
C:\textbackslash{}Users\textbackslash{}Elizabeth\textbackslash{}Anaconda3\textbackslash{}lib\textbackslash{}site-packages\textbackslash{}matplotlib\textbackslash{}font\_manager.py:1328: UserWarning: findfont: Font family ['Helvetica'] not found. Falling back to DejaVu Sans
  (prop.get\_family(), self.defaultFamily[fontext]))

    \end{Verbatim}

    \begin{center}
    \adjustimage{max size={0.9\linewidth}{0.9\paperheight}}{output_19_2.png}
    \end{center}
    { \hspace*{\fill} \\}
    
    Poniendo ambas escalas log, es claro que en un ploteo de bineado
logaritmico en base 2 se aprecia de forma mas clara el comportamiento de
la red libre escala porque en el caso lineal se ve que para un mismo
valor de \(P_k\) le corresponden distintos valores k lo cual no permite
visualizar correctamente el comportamiento lineal. Al tomar un bineado
logarítimico se agrupan los k en la serie
\(b = [0, 2^0, 2^1, 2^2, ... ,2^{i_{max}}], b_i = 2^i\) y se tiene un
único valor del \(P_i\) que corresponde con ese elemento de la serie
\(b_i\) tal que hace un promedio entonces es claro visualizar la
relación lineal y hacer un ajuste sobre ese grafico en vez de en el
bineado lineal.

    \section{b. EL AJUSTE PARA ENCONTRAR EL PARAMETRO DE
ESCALA}\label{b.-el-ajuste-para-encontrar-el-parametro-de-escala}

    \begin{Verbatim}[commandchars=\\\{\}]
{\color{incolor}In [{\color{incolor}17}]:} \PY{c+c1}{\PYZsh{}\PYZsh{}\PYZsh{}\PYZsh{}\PYZsh{}\PYZsh{}\PYZsh{}\PYZsh{}\PYZsh{}\PYZsh{}}
         \PY{c+c1}{\PYZsh{} Fitting the data \PYZhy{}\PYZhy{} Least Squares Method}
         \PY{c+c1}{\PYZsh{}\PYZsh{}\PYZsh{}\PYZsh{}\PYZsh{}\PYZsh{}\PYZsh{}\PYZsh{}\PYZsh{}\PYZsh{}}
         
         \PY{c+c1}{\PYZsh{} Power\PYZhy{}law fitting is best done by first converting}
         \PY{c+c1}{\PYZsh{} to a linear equation and then fitting to a straight line.}
         \PY{c+c1}{\PYZsh{} Note that the `logyerr` term here is ignoring a constant prefactor.}
         \PY{c+c1}{\PYZsh{}}
         \PY{c+c1}{\PYZsh{}  y = a * x\PYZca{}b}
         \PY{c+c1}{\PYZsh{}  log(y) = log(a) + b*log(x)}
         \PY{c+c1}{\PYZsh{}}
         
         \PY{c+c1}{\PYZsh{}\PYZsh{}\PYZsh{}\PYZsh{}\PYZsh{}\PYZsh{}\PYZsh{}\PYZsh{}\PYZsh{}\PYZsh{}\PYZsh{}\PYZsh{}\PYZsh{}\PYZsh{}\PYZsh{}\PYZsh{}\PYZsh{}\PYZsh{}\PYZsh{}\PYZsh{}\PYZsh{}}
         \PY{c+c1}{\PYZsh{}\PYZsh{}\PYZsh{}\PYZsh{}\PYZsh{}\PYZsh{}\PYZsh{}\PYZsh{}\PYZsh{}\PYZsh{}\PYZsh{}\PYZsh{}\PYZsh{}\PYZsh{}\PYZsh{}\PYZsh{}\PYZsh{}\PYZsh{}\PYZsh{}\PYZsh{}\PYZsh{}serie tiene su primer elemento que es cero.. entonces se lo sacamos porque no podemos hacer el log sino. }
         \PY{k}{def} \PY{n+nf}{recortar}\PY{p}{(}\PY{n}{x}\PY{p}{)}\PY{p}{:}
             \PY{n}{xf} \PY{o}{=} \PY{p}{[}\PY{p}{]}
             \PY{k}{for} \PY{n}{i} \PY{o+ow}{in} \PY{n+nb}{range}\PY{p}{(}\PY{n+nb}{len}\PY{p}{(}\PY{n}{x}\PY{p}{)}\PY{o}{\PYZhy{}}\PY{l+m+mi}{1}\PY{p}{)}\PY{p}{:}
                 \PY{n}{xf}\PY{o}{.}\PY{n}{append}\PY{p}{(}\PY{n}{x}\PY{p}{[}\PY{n}{i}\PY{o}{+}\PY{l+m+mi}{1}\PY{p}{]}\PY{p}{)}
             \PY{k}{return} \PY{n}{xf}
          
         \PY{n}{xdata} \PY{o}{=} \PY{n}{recortar}\PY{p}{(}\PY{n}{serie}\PY{p}{)}
         \PY{n}{ydata} \PY{o}{=} \PY{n}{recortar}\PY{p}{(}\PY{n}{Plog}\PY{p}{)}
         
         \PY{c+c1}{\PYZsh{}\PYZsh{}\PYZsh{}\PYZsh{}\PYZsh{}\PYZsh{}\PYZsh{}\PYZsh{}\PYZsh{}\PYZsh{}\PYZsh{}\PYZsh{}\PYZsh{}\PYZsh{}\PYZsh{}\PYZsh{}\PYZsh{}\PYZsh{}\PYZsh{}\PYZsh{}}
         \PY{c+c1}{\PYZsh{}esta es la funcion que queremos fitear }
         \PY{n}{powerlaw} \PY{o}{=} \PY{k}{lambda} \PY{n}{x}\PY{p}{,} \PY{n}{amp}\PY{p}{,} \PY{n}{index}\PY{p}{:} \PY{n}{amp} \PY{o}{*} \PY{p}{(}\PY{n}{x}\PY{o}{*}\PY{o}{*}\PY{n}{index}\PY{p}{)}
         
         \PY{c+c1}{\PYZsh{}\PYZsh{}\PYZsh{}\PYZsh{}\PYZsh{}\PYZsh{}\PYZsh{}\PYZsh{}\PYZsh{}\PYZsh{}\PYZsh{}\PYZsh{}\PYZsh{}\PYZsh{}\PYZsh{}\PYZsh{}\PYZsh{}\PYZsh{}\PYZsh{}}
         \PY{c+c1}{\PYZsh{}es mejor fitear una lineal y despues volver a la powerlaw con los parametros que obtenemos}
         \PY{k+kn}{from} \PY{n+nn}{scipy} \PY{k}{import} \PY{n}{optimize}
         
         \PY{n}{logx} \PY{o}{=} \PY{n}{np}\PY{o}{.}\PY{n}{log10}\PY{p}{(}\PY{n}{xdata}\PY{p}{)}
         \PY{n}{logy} \PY{o}{=} \PY{n}{np}\PY{o}{.}\PY{n}{log10}\PY{p}{(}\PY{n}{ydata}\PY{p}{)}
         
         
         \PY{n}{fitfunc} \PY{o}{=} \PY{k}{lambda} \PY{n}{p}\PY{p}{,} \PY{n}{x}\PY{p}{:} \PY{n}{p}\PY{p}{[}\PY{l+m+mi}{0}\PY{p}{]} \PY{o}{+} \PY{n}{p}\PY{p}{[}\PY{l+m+mi}{1}\PY{p}{]} \PY{o}{*} \PY{n}{x}
         \PY{n}{errfunc} \PY{o}{=} \PY{k}{lambda} \PY{n}{p}\PY{p}{,} \PY{n}{x}\PY{p}{,} \PY{n}{y}\PY{p}{:} \PY{p}{(}\PY{n}{y} \PY{o}{\PYZhy{}} \PY{n}{fitfunc}\PY{p}{(}\PY{n}{p}\PY{p}{,} \PY{n}{x}\PY{p}{)}\PY{p}{)}                                                \PY{c+c1}{\PYZsh{}me lo que vamos a minimizar}
         
         
         \PY{n}{pinit} \PY{o}{=} \PY{p}{[}\PY{l+m+mf}{1.0}\PY{p}{,} \PY{o}{\PYZhy{}}\PY{l+m+mf}{1.0}\PY{p}{]}                                                                
         \PY{c+c1}{\PYZsh{}elegimos unos valores ad hoc y vemos que funcionaron sino probariamos distintos                                             }
         
         \PY{n}{out} \PY{o}{=} \PY{n}{optimize}\PY{o}{.}\PY{n}{leastsq}\PY{p}{(}\PY{n}{errfunc}\PY{p}{,} \PY{n}{pinit}\PY{p}{,}\PY{n}{args}\PY{o}{=}\PY{p}{(}\PY{n}{logx}\PY{p}{,} \PY{n}{logy}\PY{p}{)}\PY{p}{,} \PY{n}{full\PYZus{}output}\PY{o}{=}\PY{l+m+mi}{1}\PY{p}{)}                     
         
         
         \PY{n}{pfinal} \PY{o}{=} \PY{n}{out}\PY{p}{[}\PY{l+m+mi}{0}\PY{p}{]}
         \PY{n}{covar} \PY{o}{=} \PY{n}{out}\PY{p}{[}\PY{l+m+mi}{1}\PY{p}{]}
         \PY{n+nb}{print}\PY{p}{(}\PY{n}{pfinal}\PY{p}{)}
         \PY{n+nb}{print}\PY{p}{(}\PY{n}{covar}\PY{p}{)}
         
         \PY{n}{index} \PY{o}{=} \PY{n}{pfinal}\PY{p}{[}\PY{l+m+mi}{1}\PY{p}{]}
         \PY{n}{amp} \PY{o}{=} \PY{l+m+mf}{10.0}\PY{o}{*}\PY{o}{*}\PY{n}{pfinal}\PY{p}{[}\PY{l+m+mi}{0}\PY{p}{]}                                                                  
\end{Verbatim}


    \begin{Verbatim}[commandchars=\\\{\}]
[-0.47583394 -1.11854948]
[[ 0.43054041 -0.22101113]
 [-0.22101113  0.15291621]]

    \end{Verbatim}

    \begin{Verbatim}[commandchars=\\\{\}]
{\color{incolor}In [{\color{incolor}18}]:} \PY{n}{font} \PY{o}{=} \PY{p}{\PYZob{}} \PY{l+s+s1}{\PYZsq{}}\PY{l+s+s1}{family}\PY{l+s+s1}{\PYZsq{}} \PY{p}{:} \PY{l+s+s1}{\PYZsq{}}\PY{l+s+s1}{\PYZsq{}}\PY{p}{,}
                 \PY{l+s+s1}{\PYZsq{}}\PY{l+s+s1}{weight}\PY{l+s+s1}{\PYZsq{}} \PY{p}{:} \PY{l+s+s1}{\PYZsq{}}\PY{l+s+s1}{bold}\PY{l+s+s1}{\PYZsq{}}\PY{p}{,}
                 \PY{l+s+s1}{\PYZsq{}}\PY{l+s+s1}{size}\PY{l+s+s1}{\PYZsq{}}   \PY{p}{:} \PY{l+m+mi}{20}\PY{p}{\PYZcb{}}
         
         \PY{n}{plt}\PY{o}{.}\PY{n}{rc}\PY{p}{(}\PY{l+s+s1}{\PYZsq{}}\PY{l+s+s1}{font}\PY{l+s+s1}{\PYZsq{}}\PY{p}{,} \PY{o}{*}\PY{o}{*}\PY{n}{font}\PY{p}{)}
         
         \PY{n}{f} \PY{o}{=} \PY{n}{plt}\PY{o}{.}\PY{n}{figure}\PY{p}{(}\PY{n}{figsize} \PY{o}{=} \PY{p}{(}\PY{l+m+mi}{12}\PY{p}{,}\PY{l+m+mi}{14}\PY{p}{)}\PY{p}{)}
         \PY{n}{f}\PY{o}{.}\PY{n}{subplots\PYZus{}adjust}\PY{p}{(}\PY{n}{hspace}\PY{o}{=}\PY{l+m+mf}{0.6}\PY{p}{,} \PY{n}{wspace}\PY{o}{=}\PY{l+m+mf}{0.6}\PY{p}{)}
         \PY{n}{sub1} \PY{o}{=} \PY{n}{f}\PY{o}{.}\PY{n}{add\PYZus{}subplot}\PY{p}{(}\PY{l+m+mi}{221}\PY{p}{)}
         \PY{n}{plt}\PY{o}{.}\PY{n}{plot}\PY{p}{(}\PY{n}{xdata}\PY{p}{,} \PY{n}{powerlaw}\PY{p}{(}\PY{n}{xdata}\PY{p}{,} \PY{n}{amp}\PY{p}{,} \PY{n}{index}\PY{p}{)}\PY{p}{)}                               \PY{c+c1}{\PYZsh{} Fit}
         \PY{n}{plt}\PY{o}{.}\PY{n}{plot}\PY{p}{(}\PY{n}{xdata}\PY{p}{,} \PY{n}{ydata}\PY{p}{,} \PY{l+s+s1}{\PYZsq{}}\PY{l+s+s1}{k.}\PY{l+s+s1}{\PYZsq{}}\PY{p}{)}                                               \PY{c+c1}{\PYZsh{} Data}
         \PY{n}{plt}\PY{o}{.}\PY{n}{title}\PY{p}{(}\PY{l+s+s1}{\PYZsq{}}\PY{l+s+s1}{Mejor fiteo Power Law}\PY{l+s+s1}{\PYZsq{}}\PY{p}{)}
         \PY{n}{plt}\PY{o}{.}\PY{n}{xlabel}\PY{p}{(}\PY{l+s+s1}{\PYZsq{}}\PY{l+s+s1}{k}\PY{l+s+s1}{\PYZsq{}}\PY{p}{)}
         \PY{n}{plt}\PY{o}{.}\PY{n}{ylabel}\PY{p}{(}\PY{l+s+s1}{\PYZsq{}}\PY{l+s+s1}{\PYZdl{}P\PYZus{}k\PYZdl{}}\PY{l+s+s1}{\PYZsq{}}\PY{p}{)}
         \PY{n}{sub2} \PY{o}{=} \PY{n}{f}\PY{o}{.}\PY{n}{add\PYZus{}subplot}\PY{p}{(}\PY{l+m+mi}{222}\PY{p}{)}
         \PY{n}{plt}\PY{o}{.}\PY{n}{loglog}\PY{p}{(}\PY{n}{xdata}\PY{p}{,} \PY{n}{powerlaw}\PY{p}{(}\PY{n}{xdata}\PY{p}{,} \PY{n}{amp}\PY{p}{,} \PY{n}{index}\PY{p}{)}\PY{p}{)}
         \PY{n}{plt}\PY{o}{.}\PY{n}{loglog}\PY{p}{(}\PY{n}{xdata}\PY{p}{,} \PY{n}{ydata}\PY{p}{,} \PY{l+s+s1}{\PYZsq{}}\PY{l+s+s1}{k.}\PY{l+s+s1}{\PYZsq{}}\PY{p}{)}  \PY{c+c1}{\PYZsh{} Data}
         \PY{n}{plt}\PY{o}{.}\PY{n}{title}\PY{p}{(}\PY{l+s+s1}{\PYZsq{}}\PY{l+s+s1}{Mejor fiteo lineal}\PY{l+s+s1}{\PYZsq{}}\PY{p}{)}
         \PY{n}{plt}\PY{o}{.}\PY{n}{xlabel}\PY{p}{(}\PY{l+s+s1}{\PYZsq{}}\PY{l+s+s1}{k (log scale)}\PY{l+s+s1}{\PYZsq{}}\PY{p}{)}
         \PY{n}{plt}\PY{o}{.}\PY{n}{ylabel}\PY{p}{(}\PY{l+s+s1}{\PYZsq{}}\PY{l+s+s1}{\PYZdl{}P\PYZus{}k\PYZdl{} (log scale)}\PY{l+s+s1}{\PYZsq{}}\PY{p}{)}
         
         
         \PY{n+nb}{print}\PY{p}{(}\PY{l+s+s1}{\PYZsq{}}\PY{l+s+s1}{parametro de orden de Power Law =}\PY{l+s+s1}{\PYZsq{}}\PY{p}{,} \PY{o}{\PYZhy{}}\PY{n}{index}\PY{p}{)}
\end{Verbatim}


    \begin{Verbatim}[commandchars=\\\{\}]
parametro de orden de Power Law = 1.1185494761412604

    \end{Verbatim}

    \begin{Verbatim}[commandchars=\\\{\}]
C:\textbackslash{}Users\textbackslash{}Elizabeth\textbackslash{}Anaconda3\textbackslash{}lib\textbackslash{}site-packages\textbackslash{}matplotlib\textbackslash{}font\_manager.py:1328: UserWarning: findfont: Font family [] not found. Falling back to DejaVu Sans
  (prop.get\_family(), self.defaultFamily[fontext]))

    \end{Verbatim}

    \begin{center}
    \adjustimage{max size={0.9\linewidth}{0.9\paperheight}}{output_23_2.png}
    \end{center}
    { \hspace*{\fill} \\}
    
    Para corroborar el valor del parametro de orden hacemos un fiteo lineal
de matplotlib a los datos linealizados (haciendoles log10) pues vimos
que en esa escala el comportamiento es lineal. Notese que no fue
necesario hacer un cutoff a partir de un k o hasta un determinado k para
considerar un comportamiento lineal.

    \begin{Verbatim}[commandchars=\\\{\}]
{\color{incolor}In [{\color{incolor}19}]:} \PY{k+kn}{from} \PY{n+nn}{scipy} \PY{k}{import} \PY{n}{stats}
         
         \PY{c+c1}{\PYZsh{} Generated linear fit}
         \PY{n}{slope}\PY{p}{,} \PY{n}{intercept}\PY{p}{,} \PY{n}{r\PYZus{}value}\PY{p}{,} \PY{n}{p\PYZus{}value}\PY{p}{,} \PY{n}{std\PYZus{}err} \PY{o}{=} \PY{n}{stats}\PY{o}{.}\PY{n}{linregress}\PY{p}{(}\PY{n}{logx}\PY{p}{,}\PY{n}{logy}\PY{p}{)}
         \PY{n}{line} \PY{o}{=} \PY{p}{[}\PY{p}{(}\PY{n}{slope} \PY{o}{*} \PY{n}{i} \PY{o}{+} \PY{n}{intercept}\PY{p}{)} \PY{k}{for} \PY{n}{i} \PY{o+ow}{in} \PY{n}{logx}\PY{p}{]}
         
         \PY{n}{plt}\PY{o}{.}\PY{n}{plot}\PY{p}{(}\PY{n}{logx}\PY{p}{,}\PY{n}{logy}\PY{p}{,}\PY{l+s+s1}{\PYZsq{}}\PY{l+s+s1}{o}\PY{l+s+s1}{\PYZsq{}}\PY{p}{,} \PY{n}{logx}\PY{p}{,} \PY{n}{line}\PY{p}{)}
         \PY{n}{plt}\PY{o}{.}\PY{n}{title}\PY{p}{(}\PY{l+s+s1}{\PYZsq{}}\PY{l+s+s1}{Fiteo lineal con Matplotlib}\PY{l+s+s1}{\PYZsq{}}\PY{p}{)}
         \PY{n}{plt}\PY{o}{.}\PY{n}{xlabel}\PY{p}{(}\PY{l+s+s1}{\PYZsq{}}\PY{l+s+s1}{\PYZdl{}log\PYZus{}}\PY{l+s+si}{\PYZob{}10\PYZcb{}}\PY{l+s+s1}{(k)\PYZdl{}}\PY{l+s+s1}{\PYZsq{}}\PY{p}{)}
         \PY{n}{plt}\PY{o}{.}\PY{n}{ylabel}\PY{p}{(}\PY{l+s+s1}{\PYZsq{}}\PY{l+s+s1}{\PYZdl{}log\PYZus{}}\PY{l+s+si}{\PYZob{}10\PYZcb{}}\PY{l+s+s1}{(P\PYZus{}k)\PYZdl{}}\PY{l+s+s1}{\PYZsq{}}\PY{p}{)}
         \PY{n+nb}{print}\PY{p}{(}\PY{l+s+s1}{\PYZsq{}}\PY{l+s+s1}{parametro de orden de ajuste lineal}\PY{l+s+s1}{\PYZsq{}}\PY{p}{,} \PY{o}{\PYZhy{}}\PY{n}{slope}\PY{p}{)}
\end{Verbatim}


    \begin{Verbatim}[commandchars=\\\{\}]
parametro de orden de ajuste lineal 1.1185494765731623

    \end{Verbatim}

    \begin{Verbatim}[commandchars=\\\{\}]
C:\textbackslash{}Users\textbackslash{}Elizabeth\textbackslash{}Anaconda3\textbackslash{}lib\textbackslash{}site-packages\textbackslash{}matplotlib\textbackslash{}font\_manager.py:1328: UserWarning: findfont: Font family [] not found. Falling back to DejaVu Sans
  (prop.get\_family(), self.defaultFamily[fontext]))

    \end{Verbatim}

    \begin{center}
    \adjustimage{max size={0.9\linewidth}{0.9\paperheight}}{output_25_2.png}
    \end{center}
    { \hspace*{\fill} \\}
    
    Entonces, vemos que por ambos ajustes hallamos el mismo valor del
parametro de orden \(\gamma = 1.1\). El hecho de poder hallar dicho
parametro nos esta diciendo que en efecto es una power law el
comportamiento. Y que la mejor forma de apreciarlo es con el bindeado
logaritico en base 2 en ambas escalas logaritmicas.


    % Add a bibliography block to the postdoc
    
    
    
    \end{document}
