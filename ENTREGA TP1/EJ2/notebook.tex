
% Default to the notebook output style

    


% Inherit from the specified cell style.




    
\documentclass[11pt]{article}

    
    
    \usepackage[T1]{fontenc}
    % Nicer default font (+ math font) than Computer Modern for most use cases
    \usepackage{mathpazo}

    % Basic figure setup, for now with no caption control since it's done
    % automatically by Pandoc (which extracts ![](path) syntax from Markdown).
    \usepackage{graphicx}
    % We will generate all images so they have a width \maxwidth. This means
    % that they will get their normal width if they fit onto the page, but
    % are scaled down if they would overflow the margins.
    \makeatletter
    \def\maxwidth{\ifdim\Gin@nat@width>\linewidth\linewidth
    \else\Gin@nat@width\fi}
    \makeatother
    \let\Oldincludegraphics\includegraphics
    % Set max figure width to be 80% of text width, for now hardcoded.
    \renewcommand{\includegraphics}[1]{\Oldincludegraphics[width=.8\maxwidth]{#1}}
    % Ensure that by default, figures have no caption (until we provide a
    % proper Figure object with a Caption API and a way to capture that
    % in the conversion process - todo).
    \usepackage{caption}
    \DeclareCaptionLabelFormat{nolabel}{}
    \captionsetup{labelformat=nolabel}

    \usepackage{adjustbox} % Used to constrain images to a maximum size 
    \usepackage{xcolor} % Allow colors to be defined
    \usepackage{enumerate} % Needed for markdown enumerations to work
    \usepackage{geometry} % Used to adjust the document margins
    \usepackage{amsmath} % Equations
    \usepackage{amssymb} % Equations
    \usepackage{textcomp} % defines textquotesingle
    % Hack from http://tex.stackexchange.com/a/47451/13684:
    \AtBeginDocument{%
        \def\PYZsq{\textquotesingle}% Upright quotes in Pygmentized code
    }
    \usepackage{upquote} % Upright quotes for verbatim code
    \usepackage{eurosym} % defines \euro
    \usepackage[mathletters]{ucs} % Extended unicode (utf-8) support
    \usepackage[utf8x]{inputenc} % Allow utf-8 characters in the tex document
    \usepackage{fancyvrb} % verbatim replacement that allows latex
    \usepackage{grffile} % extends the file name processing of package graphics 
                         % to support a larger range 
    % The hyperref package gives us a pdf with properly built
    % internal navigation ('pdf bookmarks' for the table of contents,
    % internal cross-reference links, web links for URLs, etc.)
    \usepackage{hyperref}
    \usepackage{longtable} % longtable support required by pandoc >1.10
    \usepackage{booktabs}  % table support for pandoc > 1.12.2
    \usepackage[inline]{enumitem} % IRkernel/repr support (it uses the enumerate* environment)
    \usepackage[normalem]{ulem} % ulem is needed to support strikethroughs (\sout)
                                % normalem makes italics be italics, not underlines
    

    
    
    % Colors for the hyperref package
    \definecolor{urlcolor}{rgb}{0,.145,.698}
    \definecolor{linkcolor}{rgb}{.71,0.21,0.01}
    \definecolor{citecolor}{rgb}{.12,.54,.11}

    % ANSI colors
    \definecolor{ansi-black}{HTML}{3E424D}
    \definecolor{ansi-black-intense}{HTML}{282C36}
    \definecolor{ansi-red}{HTML}{E75C58}
    \definecolor{ansi-red-intense}{HTML}{B22B31}
    \definecolor{ansi-green}{HTML}{00A250}
    \definecolor{ansi-green-intense}{HTML}{007427}
    \definecolor{ansi-yellow}{HTML}{DDB62B}
    \definecolor{ansi-yellow-intense}{HTML}{B27D12}
    \definecolor{ansi-blue}{HTML}{208FFB}
    \definecolor{ansi-blue-intense}{HTML}{0065CA}
    \definecolor{ansi-magenta}{HTML}{D160C4}
    \definecolor{ansi-magenta-intense}{HTML}{A03196}
    \definecolor{ansi-cyan}{HTML}{60C6C8}
    \definecolor{ansi-cyan-intense}{HTML}{258F8F}
    \definecolor{ansi-white}{HTML}{C5C1B4}
    \definecolor{ansi-white-intense}{HTML}{A1A6B2}

    % commands and environments needed by pandoc snippets
    % extracted from the output of `pandoc -s`
    \providecommand{\tightlist}{%
      \setlength{\itemsep}{0pt}\setlength{\parskip}{0pt}}
    \DefineVerbatimEnvironment{Highlighting}{Verbatim}{commandchars=\\\{\}}
    % Add ',fontsize=\small' for more characters per line
    \newenvironment{Shaded}{}{}
    \newcommand{\KeywordTok}[1]{\textcolor[rgb]{0.00,0.44,0.13}{\textbf{{#1}}}}
    \newcommand{\DataTypeTok}[1]{\textcolor[rgb]{0.56,0.13,0.00}{{#1}}}
    \newcommand{\DecValTok}[1]{\textcolor[rgb]{0.25,0.63,0.44}{{#1}}}
    \newcommand{\BaseNTok}[1]{\textcolor[rgb]{0.25,0.63,0.44}{{#1}}}
    \newcommand{\FloatTok}[1]{\textcolor[rgb]{0.25,0.63,0.44}{{#1}}}
    \newcommand{\CharTok}[1]{\textcolor[rgb]{0.25,0.44,0.63}{{#1}}}
    \newcommand{\StringTok}[1]{\textcolor[rgb]{0.25,0.44,0.63}{{#1}}}
    \newcommand{\CommentTok}[1]{\textcolor[rgb]{0.38,0.63,0.69}{\textit{{#1}}}}
    \newcommand{\OtherTok}[1]{\textcolor[rgb]{0.00,0.44,0.13}{{#1}}}
    \newcommand{\AlertTok}[1]{\textcolor[rgb]{1.00,0.00,0.00}{\textbf{{#1}}}}
    \newcommand{\FunctionTok}[1]{\textcolor[rgb]{0.02,0.16,0.49}{{#1}}}
    \newcommand{\RegionMarkerTok}[1]{{#1}}
    \newcommand{\ErrorTok}[1]{\textcolor[rgb]{1.00,0.00,0.00}{\textbf{{#1}}}}
    \newcommand{\NormalTok}[1]{{#1}}
    
    % Additional commands for more recent versions of Pandoc
    \newcommand{\ConstantTok}[1]{\textcolor[rgb]{0.53,0.00,0.00}{{#1}}}
    \newcommand{\SpecialCharTok}[1]{\textcolor[rgb]{0.25,0.44,0.63}{{#1}}}
    \newcommand{\VerbatimStringTok}[1]{\textcolor[rgb]{0.25,0.44,0.63}{{#1}}}
    \newcommand{\SpecialStringTok}[1]{\textcolor[rgb]{0.73,0.40,0.53}{{#1}}}
    \newcommand{\ImportTok}[1]{{#1}}
    \newcommand{\DocumentationTok}[1]{\textcolor[rgb]{0.73,0.13,0.13}{\textit{{#1}}}}
    \newcommand{\AnnotationTok}[1]{\textcolor[rgb]{0.38,0.63,0.69}{\textbf{\textit{{#1}}}}}
    \newcommand{\CommentVarTok}[1]{\textcolor[rgb]{0.38,0.63,0.69}{\textbf{\textit{{#1}}}}}
    \newcommand{\VariableTok}[1]{\textcolor[rgb]{0.10,0.09,0.49}{{#1}}}
    \newcommand{\ControlFlowTok}[1]{\textcolor[rgb]{0.00,0.44,0.13}{\textbf{{#1}}}}
    \newcommand{\OperatorTok}[1]{\textcolor[rgb]{0.40,0.40,0.40}{{#1}}}
    \newcommand{\BuiltInTok}[1]{{#1}}
    \newcommand{\ExtensionTok}[1]{{#1}}
    \newcommand{\PreprocessorTok}[1]{\textcolor[rgb]{0.74,0.48,0.00}{{#1}}}
    \newcommand{\AttributeTok}[1]{\textcolor[rgb]{0.49,0.56,0.16}{{#1}}}
    \newcommand{\InformationTok}[1]{\textcolor[rgb]{0.38,0.63,0.69}{\textbf{\textit{{#1}}}}}
    \newcommand{\WarningTok}[1]{\textcolor[rgb]{0.38,0.63,0.69}{\textbf{\textit{{#1}}}}}
    
    
    % Define a nice break command that doesn't care if a line doesn't already
    % exist.
    \def\br{\hspace*{\fill} \\* }
    % Math Jax compatability definitions
    \def\gt{>}
    \def\lt{<}
    % Document parameters
    \title{EJ2}
    
    
    

    % Pygments definitions
    
\makeatletter
\def\PY@reset{\let\PY@it=\relax \let\PY@bf=\relax%
    \let\PY@ul=\relax \let\PY@tc=\relax%
    \let\PY@bc=\relax \let\PY@ff=\relax}
\def\PY@tok#1{\csname PY@tok@#1\endcsname}
\def\PY@toks#1+{\ifx\relax#1\empty\else%
    \PY@tok{#1}\expandafter\PY@toks\fi}
\def\PY@do#1{\PY@bc{\PY@tc{\PY@ul{%
    \PY@it{\PY@bf{\PY@ff{#1}}}}}}}
\def\PY#1#2{\PY@reset\PY@toks#1+\relax+\PY@do{#2}}

\expandafter\def\csname PY@tok@w\endcsname{\def\PY@tc##1{\textcolor[rgb]{0.73,0.73,0.73}{##1}}}
\expandafter\def\csname PY@tok@c\endcsname{\let\PY@it=\textit\def\PY@tc##1{\textcolor[rgb]{0.25,0.50,0.50}{##1}}}
\expandafter\def\csname PY@tok@cp\endcsname{\def\PY@tc##1{\textcolor[rgb]{0.74,0.48,0.00}{##1}}}
\expandafter\def\csname PY@tok@k\endcsname{\let\PY@bf=\textbf\def\PY@tc##1{\textcolor[rgb]{0.00,0.50,0.00}{##1}}}
\expandafter\def\csname PY@tok@kp\endcsname{\def\PY@tc##1{\textcolor[rgb]{0.00,0.50,0.00}{##1}}}
\expandafter\def\csname PY@tok@kt\endcsname{\def\PY@tc##1{\textcolor[rgb]{0.69,0.00,0.25}{##1}}}
\expandafter\def\csname PY@tok@o\endcsname{\def\PY@tc##1{\textcolor[rgb]{0.40,0.40,0.40}{##1}}}
\expandafter\def\csname PY@tok@ow\endcsname{\let\PY@bf=\textbf\def\PY@tc##1{\textcolor[rgb]{0.67,0.13,1.00}{##1}}}
\expandafter\def\csname PY@tok@nb\endcsname{\def\PY@tc##1{\textcolor[rgb]{0.00,0.50,0.00}{##1}}}
\expandafter\def\csname PY@tok@nf\endcsname{\def\PY@tc##1{\textcolor[rgb]{0.00,0.00,1.00}{##1}}}
\expandafter\def\csname PY@tok@nc\endcsname{\let\PY@bf=\textbf\def\PY@tc##1{\textcolor[rgb]{0.00,0.00,1.00}{##1}}}
\expandafter\def\csname PY@tok@nn\endcsname{\let\PY@bf=\textbf\def\PY@tc##1{\textcolor[rgb]{0.00,0.00,1.00}{##1}}}
\expandafter\def\csname PY@tok@ne\endcsname{\let\PY@bf=\textbf\def\PY@tc##1{\textcolor[rgb]{0.82,0.25,0.23}{##1}}}
\expandafter\def\csname PY@tok@nv\endcsname{\def\PY@tc##1{\textcolor[rgb]{0.10,0.09,0.49}{##1}}}
\expandafter\def\csname PY@tok@no\endcsname{\def\PY@tc##1{\textcolor[rgb]{0.53,0.00,0.00}{##1}}}
\expandafter\def\csname PY@tok@nl\endcsname{\def\PY@tc##1{\textcolor[rgb]{0.63,0.63,0.00}{##1}}}
\expandafter\def\csname PY@tok@ni\endcsname{\let\PY@bf=\textbf\def\PY@tc##1{\textcolor[rgb]{0.60,0.60,0.60}{##1}}}
\expandafter\def\csname PY@tok@na\endcsname{\def\PY@tc##1{\textcolor[rgb]{0.49,0.56,0.16}{##1}}}
\expandafter\def\csname PY@tok@nt\endcsname{\let\PY@bf=\textbf\def\PY@tc##1{\textcolor[rgb]{0.00,0.50,0.00}{##1}}}
\expandafter\def\csname PY@tok@nd\endcsname{\def\PY@tc##1{\textcolor[rgb]{0.67,0.13,1.00}{##1}}}
\expandafter\def\csname PY@tok@s\endcsname{\def\PY@tc##1{\textcolor[rgb]{0.73,0.13,0.13}{##1}}}
\expandafter\def\csname PY@tok@sd\endcsname{\let\PY@it=\textit\def\PY@tc##1{\textcolor[rgb]{0.73,0.13,0.13}{##1}}}
\expandafter\def\csname PY@tok@si\endcsname{\let\PY@bf=\textbf\def\PY@tc##1{\textcolor[rgb]{0.73,0.40,0.53}{##1}}}
\expandafter\def\csname PY@tok@se\endcsname{\let\PY@bf=\textbf\def\PY@tc##1{\textcolor[rgb]{0.73,0.40,0.13}{##1}}}
\expandafter\def\csname PY@tok@sr\endcsname{\def\PY@tc##1{\textcolor[rgb]{0.73,0.40,0.53}{##1}}}
\expandafter\def\csname PY@tok@ss\endcsname{\def\PY@tc##1{\textcolor[rgb]{0.10,0.09,0.49}{##1}}}
\expandafter\def\csname PY@tok@sx\endcsname{\def\PY@tc##1{\textcolor[rgb]{0.00,0.50,0.00}{##1}}}
\expandafter\def\csname PY@tok@m\endcsname{\def\PY@tc##1{\textcolor[rgb]{0.40,0.40,0.40}{##1}}}
\expandafter\def\csname PY@tok@gh\endcsname{\let\PY@bf=\textbf\def\PY@tc##1{\textcolor[rgb]{0.00,0.00,0.50}{##1}}}
\expandafter\def\csname PY@tok@gu\endcsname{\let\PY@bf=\textbf\def\PY@tc##1{\textcolor[rgb]{0.50,0.00,0.50}{##1}}}
\expandafter\def\csname PY@tok@gd\endcsname{\def\PY@tc##1{\textcolor[rgb]{0.63,0.00,0.00}{##1}}}
\expandafter\def\csname PY@tok@gi\endcsname{\def\PY@tc##1{\textcolor[rgb]{0.00,0.63,0.00}{##1}}}
\expandafter\def\csname PY@tok@gr\endcsname{\def\PY@tc##1{\textcolor[rgb]{1.00,0.00,0.00}{##1}}}
\expandafter\def\csname PY@tok@ge\endcsname{\let\PY@it=\textit}
\expandafter\def\csname PY@tok@gs\endcsname{\let\PY@bf=\textbf}
\expandafter\def\csname PY@tok@gp\endcsname{\let\PY@bf=\textbf\def\PY@tc##1{\textcolor[rgb]{0.00,0.00,0.50}{##1}}}
\expandafter\def\csname PY@tok@go\endcsname{\def\PY@tc##1{\textcolor[rgb]{0.53,0.53,0.53}{##1}}}
\expandafter\def\csname PY@tok@gt\endcsname{\def\PY@tc##1{\textcolor[rgb]{0.00,0.27,0.87}{##1}}}
\expandafter\def\csname PY@tok@err\endcsname{\def\PY@bc##1{\setlength{\fboxsep}{0pt}\fcolorbox[rgb]{1.00,0.00,0.00}{1,1,1}{\strut ##1}}}
\expandafter\def\csname PY@tok@kc\endcsname{\let\PY@bf=\textbf\def\PY@tc##1{\textcolor[rgb]{0.00,0.50,0.00}{##1}}}
\expandafter\def\csname PY@tok@kd\endcsname{\let\PY@bf=\textbf\def\PY@tc##1{\textcolor[rgb]{0.00,0.50,0.00}{##1}}}
\expandafter\def\csname PY@tok@kn\endcsname{\let\PY@bf=\textbf\def\PY@tc##1{\textcolor[rgb]{0.00,0.50,0.00}{##1}}}
\expandafter\def\csname PY@tok@kr\endcsname{\let\PY@bf=\textbf\def\PY@tc##1{\textcolor[rgb]{0.00,0.50,0.00}{##1}}}
\expandafter\def\csname PY@tok@bp\endcsname{\def\PY@tc##1{\textcolor[rgb]{0.00,0.50,0.00}{##1}}}
\expandafter\def\csname PY@tok@fm\endcsname{\def\PY@tc##1{\textcolor[rgb]{0.00,0.00,1.00}{##1}}}
\expandafter\def\csname PY@tok@vc\endcsname{\def\PY@tc##1{\textcolor[rgb]{0.10,0.09,0.49}{##1}}}
\expandafter\def\csname PY@tok@vg\endcsname{\def\PY@tc##1{\textcolor[rgb]{0.10,0.09,0.49}{##1}}}
\expandafter\def\csname PY@tok@vi\endcsname{\def\PY@tc##1{\textcolor[rgb]{0.10,0.09,0.49}{##1}}}
\expandafter\def\csname PY@tok@vm\endcsname{\def\PY@tc##1{\textcolor[rgb]{0.10,0.09,0.49}{##1}}}
\expandafter\def\csname PY@tok@sa\endcsname{\def\PY@tc##1{\textcolor[rgb]{0.73,0.13,0.13}{##1}}}
\expandafter\def\csname PY@tok@sb\endcsname{\def\PY@tc##1{\textcolor[rgb]{0.73,0.13,0.13}{##1}}}
\expandafter\def\csname PY@tok@sc\endcsname{\def\PY@tc##1{\textcolor[rgb]{0.73,0.13,0.13}{##1}}}
\expandafter\def\csname PY@tok@dl\endcsname{\def\PY@tc##1{\textcolor[rgb]{0.73,0.13,0.13}{##1}}}
\expandafter\def\csname PY@tok@s2\endcsname{\def\PY@tc##1{\textcolor[rgb]{0.73,0.13,0.13}{##1}}}
\expandafter\def\csname PY@tok@sh\endcsname{\def\PY@tc##1{\textcolor[rgb]{0.73,0.13,0.13}{##1}}}
\expandafter\def\csname PY@tok@s1\endcsname{\def\PY@tc##1{\textcolor[rgb]{0.73,0.13,0.13}{##1}}}
\expandafter\def\csname PY@tok@mb\endcsname{\def\PY@tc##1{\textcolor[rgb]{0.40,0.40,0.40}{##1}}}
\expandafter\def\csname PY@tok@mf\endcsname{\def\PY@tc##1{\textcolor[rgb]{0.40,0.40,0.40}{##1}}}
\expandafter\def\csname PY@tok@mh\endcsname{\def\PY@tc##1{\textcolor[rgb]{0.40,0.40,0.40}{##1}}}
\expandafter\def\csname PY@tok@mi\endcsname{\def\PY@tc##1{\textcolor[rgb]{0.40,0.40,0.40}{##1}}}
\expandafter\def\csname PY@tok@il\endcsname{\def\PY@tc##1{\textcolor[rgb]{0.40,0.40,0.40}{##1}}}
\expandafter\def\csname PY@tok@mo\endcsname{\def\PY@tc##1{\textcolor[rgb]{0.40,0.40,0.40}{##1}}}
\expandafter\def\csname PY@tok@ch\endcsname{\let\PY@it=\textit\def\PY@tc##1{\textcolor[rgb]{0.25,0.50,0.50}{##1}}}
\expandafter\def\csname PY@tok@cm\endcsname{\let\PY@it=\textit\def\PY@tc##1{\textcolor[rgb]{0.25,0.50,0.50}{##1}}}
\expandafter\def\csname PY@tok@cpf\endcsname{\let\PY@it=\textit\def\PY@tc##1{\textcolor[rgb]{0.25,0.50,0.50}{##1}}}
\expandafter\def\csname PY@tok@c1\endcsname{\let\PY@it=\textit\def\PY@tc##1{\textcolor[rgb]{0.25,0.50,0.50}{##1}}}
\expandafter\def\csname PY@tok@cs\endcsname{\let\PY@it=\textit\def\PY@tc##1{\textcolor[rgb]{0.25,0.50,0.50}{##1}}}

\def\PYZbs{\char`\\}
\def\PYZus{\char`\_}
\def\PYZob{\char`\{}
\def\PYZcb{\char`\}}
\def\PYZca{\char`\^}
\def\PYZam{\char`\&}
\def\PYZlt{\char`\<}
\def\PYZgt{\char`\>}
\def\PYZsh{\char`\#}
\def\PYZpc{\char`\%}
\def\PYZdl{\char`\$}
\def\PYZhy{\char`\-}
\def\PYZsq{\char`\'}
\def\PYZdq{\char`\"}
\def\PYZti{\char`\~}
% for compatibility with earlier versions
\def\PYZat{@}
\def\PYZlb{[}
\def\PYZrb{]}
\makeatother


    % Exact colors from NB
    \definecolor{incolor}{rgb}{0.0, 0.0, 0.5}
    \definecolor{outcolor}{rgb}{0.545, 0.0, 0.0}



    
    % Prevent overflowing lines due to hard-to-break entities
    \sloppy 
    % Setup hyperref package
    \hypersetup{
      breaklinks=true,  % so long urls are correctly broken across lines
      colorlinks=true,
      urlcolor=urlcolor,
      linkcolor=linkcolor,
      citecolor=citecolor,
      }
    % Slightly bigger margins than the latex defaults
    
    \geometry{verbose,tmargin=1in,bmargin=1in,lmargin=1in,rmargin=1in}
    
    

    \begin{document}
    
    
    \maketitle
    
    

    
    \section{EJERCICIO 2 : DELFINES DE NUEVA
ZELANDA}\label{ejercicio-2-delfines-de-nueva-zelanda}

    \begin{Verbatim}[commandchars=\\\{\}]
{\color{incolor}In [{\color{incolor}1}]:} \PY{k+kn}{import} \PY{n+nn}{numpy} \PY{k}{as} \PY{n+nn}{np}
        \PY{k+kn}{import} \PY{n+nn}{networkx} \PY{k}{as} \PY{n+nn}{nx}
        \PY{k+kn}{import} \PY{n+nn}{os}
        \PY{k+kn}{from} \PY{n+nn}{random} \PY{k}{import} \PY{n}{shuffle}          \PY{c+c1}{\PYZsh{} importamos las librerias necesarias}
        \PY{k+kn}{import} \PY{n+nn}{matplotlib}\PY{n+nn}{.}\PY{n+nn}{pylab} \PY{k}{as} \PY{n+nn}{plt}       
        \PY{o}{\PYZpc{}}\PY{k}{matplotlib} inline
        \PY{k+kn}{from} \PY{n+nn}{matplotlib}\PY{n+nn}{.}\PY{n+nn}{pyplot} \PY{k}{import} \PY{n}{title}\PY{p}{,}\PY{n}{xlabel}\PY{p}{,}\PY{n}{ylabel}\PY{p}{,}\PY{n}{show}
\end{Verbatim}


    \begin{Verbatim}[commandchars=\\\{\}]
{\color{incolor}In [{\color{incolor}2}]:} \PY{n}{G} \PY{o}{=} \PY{n}{nx}\PY{o}{.}\PY{n}{read\PYZus{}gml}\PY{p}{(}\PY{l+s+s1}{\PYZsq{}}\PY{l+s+s1}{dolphins.gml}\PY{l+s+s1}{\PYZsq{}}\PY{p}{)} \PY{c+c1}{\PYZsh{} definimos la red:cada nodo es un delfin y cada conexion representa una interaccion}
\end{Verbatim}


    \begin{Verbatim}[commandchars=\\\{\}]
{\color{incolor}In [{\color{incolor}3}]:} \PY{k}{def} \PY{n+nf}{ldata}\PY{p}{(}\PY{n}{archive}\PY{p}{)}\PY{p}{:}
            \PY{n}{f}\PY{o}{=}\PY{n+nb}{open}\PY{p}{(}\PY{n}{archive}\PY{p}{)}
            \PY{n}{data}\PY{o}{=}\PY{p}{[}\PY{p}{]}
            \PY{k}{for} \PY{n}{line} \PY{o+ow}{in} \PY{n}{f}\PY{p}{:}
                \PY{n}{line}\PY{o}{=}\PY{n}{line}\PY{o}{.}\PY{n}{strip}\PY{p}{(}\PY{p}{)}
                \PY{n}{col}\PY{o}{=}\PY{n}{line}\PY{o}{.}\PY{n}{split}\PY{p}{(}\PY{p}{)}
                \PY{n}{data}\PY{o}{.}\PY{n}{append}\PY{p}{(}\PY{n}{col}\PY{p}{)}
            \PY{k}{return} \PY{n}{data}
        
        \PY{n}{dolphinsGender}\PY{o}{=}\PY{n}{ldata}\PY{p}{(}\PY{l+s+s1}{\PYZsq{}}\PY{l+s+s1}{dolphinsGender.txt}\PY{l+s+s1}{\PYZsq{}}\PY{p}{)} \PY{c+c1}{\PYZsh{} usamos la funcion \PYZdq{}ldata\PYZdq{} para abrir el archivo .txt }
                                                   \PY{c+c1}{\PYZsh{} que contiene el genero de cada delfin}
\end{Verbatim}


    \begin{Verbatim}[commandchars=\\\{\}]
{\color{incolor}In [{\color{incolor}4}]:} \PY{n}{dict\PYZus{}gender} \PY{o}{=} \PY{p}{\PYZob{}}\PY{n}{dolphin\PYZus{}nombre} \PY{p}{:} \PY{n}{genero} \PY{k}{for} \PY{n}{dolphin\PYZus{}nombre}\PY{p}{,} \PY{n}{genero}  \PY{o+ow}{in} \PY{n}{dolphinsGender}\PY{p}{\PYZcb{}} \PY{c+c1}{\PYZsh{}lista a diccionario}
\end{Verbatim}


    \begin{Verbatim}[commandchars=\\\{\}]
{\color{incolor}In [{\color{incolor}5}]:} \PY{k}{def} \PY{n+nf}{AssignGender}\PY{p}{(}\PY{n}{G}\PY{p}{,}\PY{n}{dict\PYZus{}gender}\PY{p}{)}\PY{p}{:}
            \PY{k}{for} \PY{n}{n} \PY{o+ow}{in} \PY{n}{G}\PY{o}{.}\PY{n}{nodes}\PY{p}{:}
                \PY{n}{G}\PY{o}{.}\PY{n}{nodes}\PY{p}{[}\PY{n}{n}\PY{p}{]}\PY{p}{[}\PY{l+s+s2}{\PYZdq{}}\PY{l+s+s2}{gender}\PY{l+s+s2}{\PYZdq{}}\PY{p}{]} \PY{o}{=} \PY{n}{dict\PYZus{}gender}\PY{p}{[}\PY{n}{n}\PY{p}{]}
            \PY{k}{return}
        
        \PY{k}{for} \PY{n}{n} \PY{o+ow}{in} \PY{n}{G}\PY{o}{.}\PY{n}{nodes}\PY{p}{:}                                   \PY{c+c1}{\PYZsh{} usamos la funcion \PYZdq{}AssignGender\PYZdq{} para asignar}
            \PY{n}{G}\PY{o}{.}\PY{n}{nodes}\PY{p}{[}\PY{n}{n}\PY{p}{]}\PY{p}{[}\PY{l+s+s2}{\PYZdq{}}\PY{l+s+s2}{gender}\PY{l+s+s2}{\PYZdq{}}\PY{p}{]} \PY{o}{=} \PY{n}{dict\PYZus{}gender}\PY{p}{[}\PY{n}{n}\PY{p}{]}           \PY{c+c1}{\PYZsh{} un atributo (genero) a los nodos de la red (delfines)}
\end{Verbatim}


    \begin{Verbatim}[commandchars=\\\{\}]
{\color{incolor}In [{\color{incolor}6}]:} \PY{n+nb}{list}\PY{p}{(}\PY{n}{nx}\PY{o}{.}\PY{n}{get\PYZus{}node\PYZus{}attributes}\PY{p}{(}\PY{n}{G}\PY{p}{,}\PY{l+s+s1}{\PYZsq{}}\PY{l+s+s1}{gender}\PY{l+s+s1}{\PYZsq{}}\PY{p}{)}\PY{o}{.}\PY{n}{items}\PY{p}{(}\PY{p}{)}\PY{p}{)}\PY{p}{[}\PY{l+m+mi}{0}\PY{p}{:}\PY{l+m+mi}{9}\PY{p}{]}  \PY{c+c1}{\PYZsh{}chequeamos algunos casos}
\end{Verbatim}


\begin{Verbatim}[commandchars=\\\{\}]
{\color{outcolor}Out[{\color{outcolor}6}]:} [('Beak', 'm'),
         ('Beescratch', 'm'),
         ('Bumper', 'm'),
         ('CCL', 'f'),
         ('Cross', 'm'),
         ('DN16', 'f'),
         ('DN21', 'm'),
         ('DN63', 'm'),
         ('Double', 'f')]
\end{Verbatim}
            
    \section{(A) Examine diferentes opciones de layout para este grafo e
identifique la que le resulte más informativa. Justifique su elección
detallando las características estructurales de la red que su elección
pone en evidencia. Incluya en la representación gráfica de la red
información sobre el sexo de los
delfines.}\label{a-examine-diferentes-opciones-de-layout-para-este-grafo-e-identifique-la-que-le-resulte-muxe1s-informativa.-justifique-su-elecciuxf3n-detallando-las-caracteruxedsticas-estructurales-de-la-red-que-su-elecciuxf3n-pone-en-evidencia.-incluya-en-la-representaciuxf3n-gruxe1fica-de-la-red-informaciuxf3n-sobre-el-sexo-de-los-delfines.}

    \begin{Verbatim}[commandchars=\\\{\}]
{\color{incolor}In [{\color{incolor}7}]:} \PY{k}{def} \PY{n+nf}{color}\PY{p}{(}\PY{n}{g}\PY{p}{)}\PY{p}{:}                  
            \PY{k}{if} \PY{n}{g}\PY{o}{==}\PY{l+s+s1}{\PYZsq{}}\PY{l+s+s1}{m}\PY{l+s+s1}{\PYZsq{}}\PY{p}{:}
                \PY{n}{col}\PY{o}{=}\PY{l+s+s1}{\PYZsq{}}\PY{l+s+s1}{blue}\PY{l+s+s1}{\PYZsq{}}
            \PY{k}{elif} \PY{n}{g}\PY{o}{==}\PY{l+s+s1}{\PYZsq{}}\PY{l+s+s1}{f}\PY{l+s+s1}{\PYZsq{}}\PY{p}{:}                   \PY{c+c1}{\PYZsh{} definimos una funcion para asignar un color distinto a cada genero}
                \PY{n}{col}\PY{o}{=}\PY{l+s+s1}{\PYZsq{}}\PY{l+s+s1}{red}\PY{l+s+s1}{\PYZsq{}}
            \PY{k}{else}\PY{p}{:}
                \PY{n}{col}\PY{o}{=}\PY{l+s+s1}{\PYZsq{}}\PY{l+s+s1}{green}\PY{l+s+s1}{\PYZsq{}}
            \PY{k}{return} \PY{n}{col}
        
        \PY{n}{options} \PY{o}{=} \PY{p}{\PYZob{}}\PY{l+s+s1}{\PYZsq{}}\PY{l+s+s1}{node\PYZus{}color}\PY{l+s+s1}{\PYZsq{}}\PY{p}{:}\PY{p}{[}\PY{n}{color}\PY{p}{(}\PY{n}{g}\PY{p}{)} \PY{k}{for} \PY{n}{g} \PY{o+ow}{in} \PY{n}{nx}\PY{o}{.}\PY{n}{get\PYZus{}node\PYZus{}attributes}\PY{p}{(}\PY{n}{G}\PY{p}{,}\PY{l+s+s1}{\PYZsq{}}\PY{l+s+s1}{gender}\PY{l+s+s1}{\PYZsq{}}\PY{p}{)}\PY{o}{.}\PY{n}{values}\PY{p}{(}\PY{p}{)}\PY{p}{]}\PY{p}{,}
                   \PY{l+s+s1}{\PYZsq{}}\PY{l+s+s1}{node\PYZus{}size}\PY{l+s+s1}{\PYZsq{}}\PY{p}{:}\PY{l+m+mi}{60}\PY{p}{,}\PY{l+s+s1}{\PYZsq{}}\PY{l+s+s1}{with\PYZus{}labels}\PY{l+s+s1}{\PYZsq{}}\PY{p}{:}\PY{k+kc}{True}\PY{p}{\PYZcb{}}
        
        \PY{n}{plt}\PY{o}{.}\PY{n}{figure}\PY{p}{(}\PY{n}{figsize}\PY{o}{=}\PY{p}{(}\PY{l+m+mi}{15}\PY{p}{,}\PY{l+m+mi}{10}\PY{p}{)}\PY{p}{)}
        \PY{n}{plt}\PY{o}{.}\PY{n}{subplot}\PY{p}{(}\PY{l+m+mi}{221}\PY{p}{)}
        \PY{n}{nx}\PY{o}{.}\PY{n}{draw\PYZus{}random}\PY{p}{(}\PY{n}{G}\PY{p}{,} \PY{o}{*}\PY{o}{*}\PY{n}{options}\PY{p}{)}
        \PY{n}{plt}\PY{o}{.}\PY{n}{subplot}\PY{p}{(}\PY{l+m+mi}{222}\PY{p}{)}
        \PY{n}{nx}\PY{o}{.}\PY{n}{draw\PYZus{}circular}\PY{p}{(}\PY{n}{G}\PY{p}{,} \PY{o}{*}\PY{o}{*}\PY{n}{options}\PY{p}{)}           \PY{c+c1}{\PYZsh{} graficamos la red con distintos layouts}
        \PY{n}{plt}\PY{o}{.}\PY{n}{subplot}\PY{p}{(}\PY{l+m+mi}{223}\PY{p}{)}
        \PY{n}{nx}\PY{o}{.}\PY{n}{draw\PYZus{}spectral}\PY{p}{(}\PY{n}{G}\PY{p}{,} \PY{o}{*}\PY{o}{*}\PY{n}{options}\PY{p}{)}
        \PY{n}{plt}\PY{o}{.}\PY{n}{subplot}\PY{p}{(}\PY{l+m+mi}{224}\PY{p}{)}
        \PY{n}{nx}\PY{o}{.}\PY{n}{draw\PYZus{}spring}\PY{p}{(}\PY{n}{G}\PY{p}{,} \PY{o}{*}\PY{o}{*}\PY{n}{options}\PY{p}{)}
\end{Verbatim}


    \begin{center}
    \adjustimage{max size={0.9\linewidth}{0.9\paperheight}}{output_8_0.png}
    \end{center}
    { \hspace*{\fill} \\}
    
    De los distintos Layouts analizados, el método "Fruchterman-Reingold
Force-Directed Graph" es el que nos permite visualizar mejor la
estructura de la red.
(https://en.wikipedia.org/wiki/Force-directed\_graph\_drawing)

    \begin{Verbatim}[commandchars=\\\{\}]
{\color{incolor}In [{\color{incolor}8}]:} \PY{n}{options} \PY{o}{=} \PY{p}{\PYZob{}}\PY{l+s+s1}{\PYZsq{}}\PY{l+s+s1}{node\PYZus{}color}\PY{l+s+s1}{\PYZsq{}}\PY{p}{:}\PY{p}{[}\PY{n}{color}\PY{p}{(}\PY{n}{g}\PY{p}{)} \PY{k}{for} \PY{n}{g} \PY{o+ow}{in} \PY{n}{nx}\PY{o}{.}\PY{n}{get\PYZus{}node\PYZus{}attributes}\PY{p}{(}\PY{n}{G}\PY{p}{,}\PY{l+s+s1}{\PYZsq{}}\PY{l+s+s1}{gender}\PY{l+s+s1}{\PYZsq{}}\PY{p}{)}\PY{o}{.}\PY{n}{values}\PY{p}{(}\PY{p}{)}\PY{p}{]}\PY{p}{,}
                   \PY{l+s+s1}{\PYZsq{}}\PY{l+s+s1}{node\PYZus{}size}\PY{l+s+s1}{\PYZsq{}}\PY{p}{:}\PY{l+m+mi}{200}\PY{p}{,}\PY{l+s+s1}{\PYZsq{}}\PY{l+s+s1}{with\PYZus{}labels}\PY{l+s+s1}{\PYZsq{}}\PY{p}{:}\PY{k+kc}{True}\PY{p}{\PYZcb{}}
        
        \PY{n}{plt}\PY{o}{.}\PY{n}{figure}\PY{p}{(}\PY{n}{figsize}\PY{o}{=}\PY{p}{(}\PY{l+m+mi}{15}\PY{p}{,}\PY{l+m+mi}{10}\PY{p}{)}\PY{p}{)}
        \PY{n}{nx}\PY{o}{.}\PY{n}{draw\PYZus{}spring}\PY{p}{(}\PY{n}{G}\PY{p}{,} \PY{o}{*}\PY{o}{*}\PY{n}{options}\PY{p}{)}
\end{Verbatim}


    \begin{center}
    \adjustimage{max size={0.9\linewidth}{0.9\paperheight}}{output_10_0.png}
    \end{center}
    { \hspace*{\fill} \\}
    
    \section{(B) ¿Se trata una red donde prevalece la homofilia en la
variable
género?}\label{b-se-trata-una-red-donde-prevalece-la-homofilia-en-la-variable-guxe9nero}

    \section{(i) Considere la distribución nula para la fracción de enlaces
que vinculan géneros diferentes, generada a partir de al menos 1000
asignaciones aleatorias de
género.}\label{i-considere-la-distribuciuxf3n-nula-para-la-fracciuxf3n-de-enlaces-que-vinculan-guxe9neros-diferentes-generada-a-partir-de-al-menos-1000-asignaciones-aleatorias-de-guxe9nero.}

    La modularidad es una medida de la estructura de las redes.
Particularmente mide la fuerza de la división de una red en
módulos/comunidades. Las redes con alta modularidad tienen conexiones
sólidas entre los nodos dentro de los módulos, pero escasas conexiones
entre nodos en diferentes módulos, con lo cual resulta un parámetro
óptimo para observar la presencia de homofilia en una red. La
modularidad se defino como:

\[Q=\frac{1}{2m} \sum_{ij} \delta(c_{i}, c_{j}) (A_{ij}-\frac{k_{i}k_{j}}{2m})\]

 \(k_{i}:\) grado del nodo i \(k_{j}:\) grado del nodo j \(c_{i}:\)
clase i \(c_{j}:\) clase j \(m:\) cantidad de enlaces total
\(A_{ij} = 1\) si hay un enlace entre el nodo i y el nodo j
\(A_{ij} = 0\) en otro caso. 

    \begin{Verbatim}[commandchars=\\\{\}]
{\color{incolor}In [{\color{incolor}9}]:} \PY{k}{def} \PY{n+nf}{delta}\PY{p}{(}\PY{n}{c\PYZus{}i}\PY{p}{,} \PY{n}{c\PYZus{}j}\PY{p}{)}\PY{p}{:}
            \PY{k}{if}  \PY{n}{c\PYZus{}i} \PY{o}{==} \PY{n}{c\PYZus{}j}\PY{p}{:}
                \PY{k}{return} \PY{l+m+mi}{1} 
            \PY{k}{else}\PY{p}{:}
                \PY{k}{return} \PY{l+m+mi}{0} 
        
        \PY{c+c1}{\PYZsh{}definimos la funcion modularidad a partir de un grafo \PYZsq{}G\PYZsq{} que ya tiene montado el campo escalar \PYZsq{}FieldString\PYZsq{} que}
        \PY{c+c1}{\PYZsh{}le da una clase a cada nodo}
        
        \PY{k}{def} \PY{n+nf}{modularity}\PY{p}{(}\PY{n}{G}\PY{p}{,}\PY{n}{FieldString}\PY{p}{)}\PY{p}{:}
            \PY{n}{modularity} \PY{o}{=} \PY{l+m+mi}{0}
            \PY{n}{m} \PY{o}{=} \PY{n}{np}\PY{o}{.}\PY{n}{shape}\PY{p}{(}\PY{n}{nx}\PY{o}{.}\PY{n}{edges}\PY{p}{(}\PY{n}{G}\PY{p}{)}\PY{p}{)}\PY{p}{[}\PY{l+m+mi}{0}\PY{p}{]} \PY{c+c1}{\PYZsh{}cantidad de enlaces}
            \PY{n}{A} \PY{o}{=} \PY{n}{nx}\PY{o}{.}\PY{n}{to\PYZus{}numpy\PYZus{}matrix}\PY{p}{(}\PY{n}{G}\PY{p}{)}    \PY{c+c1}{\PYZsh{}adjency matrix (por definicion cumple las condiciones de A\PYZus{}ij mencionadas arriba)}
            
            \PY{n}{Degree} \PY{o}{=} \PY{n+nb}{list}\PY{p}{(}\PY{n}{G}\PY{o}{.}\PY{n}{degree}\PY{p}{)}                     \PY{c+c1}{\PYZsh{} lista de tuplas (nodo,grado) con un orden determinado}
            \PY{n}{degree} \PY{o}{=} \PY{p}{[}\PY{n}{degree} \PY{k}{for} \PY{n}{node}\PY{p}{,}\PY{n}{degree} \PY{o+ow}{in} \PY{n}{Degree}\PY{p}{]} \PY{c+c1}{\PYZsh{} genero un vector donde cada elemento es el grado para el nodo}
                                                        \PY{c+c1}{\PYZsh{} indexado como 0,1,2... , que se corresponde con el orden de }
                                                        \PY{c+c1}{\PYZsh{} la lista Degree}
            
            \PY{n}{atrributes} \PY{o}{=} \PY{n+nb}{list}\PY{p}{(}\PY{n}{nx}\PY{o}{.}\PY{n}{get\PYZus{}node\PYZus{}attributes}\PY{p}{(}\PY{n}{G}\PY{p}{,}\PY{n}{FieldString}\PY{p}{)}\PY{o}{.}\PY{n}{values}\PY{p}{(}\PY{p}{)}\PY{p}{)}  \PY{c+c1}{\PYZsh{}lista con los generos ordenados}
            
            \PY{k}{for} \PY{n}{i} \PY{o+ow}{in} \PY{n+nb}{range}\PY{p}{(}\PY{n}{np}\PY{o}{.}\PY{n}{size}\PY{p}{(}\PY{n}{nx}\PY{o}{.}\PY{n}{nodes}\PY{p}{(}\PY{n}{G}\PY{p}{)}\PY{p}{)}\PY{p}{)}\PY{p}{:}
                \PY{k}{for} \PY{n}{j} \PY{o+ow}{in} \PY{n+nb}{range}\PY{p}{(}\PY{n}{np}\PY{o}{.}\PY{n}{size}\PY{p}{(}\PY{n}{nx}\PY{o}{.}\PY{n}{nodes}\PY{p}{(}\PY{n}{G}\PY{p}{)}\PY{p}{)}\PY{p}{)}\PY{p}{:}
                    \PY{n}{modularity} \PY{o}{=} \PY{p}{(}\PY{p}{(}\PY{l+m+mi}{1}\PY{o}{/}\PY{p}{(}\PY{l+m+mi}{2} \PY{o}{*} \PY{n}{m}\PY{p}{)}\PY{p}{)} \PY{o}{*} \PY{n}{delta}\PY{p}{(}\PY{n}{atrributes}\PY{p}{[}\PY{n}{i}\PY{p}{]}\PY{p}{,} \PY{n}{atrributes}\PY{p}{[}\PY{n}{j}\PY{p}{]}\PY{p}{)}\PY{o}{*}     \PY{c+c1}{\PYZsh{} calculo la modularidad}
                                  \PY{p}{(}\PY{n}{A}\PY{p}{[}\PY{n}{i}\PY{p}{,}\PY{n}{j}\PY{p}{]} \PY{o}{\PYZhy{}} \PY{p}{(}\PY{n}{degree}\PY{p}{[}\PY{n}{i}\PY{p}{]} \PY{o}{*} \PY{n}{degree}\PY{p}{[}\PY{n}{j}\PY{p}{]}\PY{p}{)}\PY{o}{/}\PY{p}{(}\PY{l+m+mi}{2}\PY{o}{*}\PY{n}{m}\PY{p}{)}\PY{p}{)}\PY{o}{+}\PY{n}{modularity}\PY{p}{)}
            
            \PY{k}{return} \PY{n}{modularity}
\end{Verbatim}


    \begin{Verbatim}[commandchars=\\\{\}]
{\color{incolor}In [{\color{incolor}10}]:} \PY{n}{modularity}\PY{p}{(}\PY{n}{G}\PY{p}{,} \PY{l+s+s1}{\PYZsq{}}\PY{l+s+s1}{gender}\PY{l+s+s1}{\PYZsq{}}\PY{p}{)} \PY{c+c1}{\PYZsh{} modularidad de la red real}
\end{Verbatim}


\begin{Verbatim}[commandchars=\\\{\}]
{\color{outcolor}Out[{\color{outcolor}10}]:} 0.12370950516197912
\end{Verbatim}
            
    Modularidad en el caso aleatorio

    \begin{Verbatim}[commandchars=\\\{\}]
{\color{incolor}In [{\color{incolor}16}]:} \PY{k}{def} \PY{n+nf}{BreakFieldAttribution}\PY{p}{(}\PY{n}{fieldAtributtion}\PY{p}{)}\PY{p}{:}              
             \PY{n}{names} \PY{o}{=} \PY{p}{[}\PY{n}{nodes} \PY{k}{for} \PY{n}{nodes}\PY{p}{,}\PY{n}{field} \PY{o+ow}{in} \PY{n}{fieldAtributtion}\PY{p}{]}  \PY{c+c1}{\PYZsh{} funcion para separar los keys y los values }
             \PY{n}{field} \PY{o}{=} \PY{p}{[}\PY{n}{field} \PY{k}{for} \PY{n}{nodes}\PY{p}{,}\PY{n}{field} \PY{o+ow}{in} \PY{n}{fieldAtributtion}\PY{p}{]}  \PY{c+c1}{\PYZsh{}          en un diccionario}
             \PY{k}{return} \PY{n}{names}\PY{p}{,}\PY{n}{field}
         
         \PY{n}{DistributionModularity} \PY{o}{=} \PY{p}{[}\PY{p}{]}                          \PY{c+c1}{\PYZsh{} Vector para guardar las modularidades de cada iteracion}
         
         \PY{n}{names} \PY{o}{=} \PY{n}{BreakFieldAttribution}\PY{p}{(}\PY{n}{dolphinsGender}\PY{p}{)}\PY{p}{[}\PY{l+m+mi}{0}\PY{p}{]}     \PY{c+c1}{\PYZsh{} genero una lista con los nombres de los delfines}
         \PY{n}{genders} \PY{o}{=} \PY{n}{BreakFieldAttribution}\PY{p}{(}\PY{n}{dolphinsGender}\PY{p}{)}\PY{p}{[}\PY{l+m+mi}{1}\PY{p}{]}   \PY{c+c1}{\PYZsh{} genero una lista con los generos de los delfines }
         \PY{n}{it} \PY{o}{=} \PY{l+m+mi}{1000}                                           \PY{c+c1}{\PYZsh{} numero de iteraciones }
         \PY{n}{H} \PY{o}{=} \PY{n}{nx}\PY{o}{.}\PY{n}{read\PYZus{}gml}\PY{p}{(}\PY{l+s+s1}{\PYZsq{}}\PY{l+s+s1}{dolphins.gml}\PY{l+s+s1}{\PYZsq{}}\PY{p}{)}                      \PY{c+c1}{\PYZsh{} trabajamos sobre H (grafo nuevo)}
         
         
         \PY{k}{for} \PY{n}{i} \PY{o+ow}{in} \PY{n+nb}{range}\PY{p}{(}\PY{n}{it}\PY{p}{)}\PY{p}{:}
             \PY{n}{np}\PY{o}{.}\PY{n}{random}\PY{o}{.}\PY{n}{shuffle}\PY{p}{(}\PY{n}{genders}\PY{p}{)}                       \PY{c+c1}{\PYZsh{} reordeno al azar la lista de generos}
             \PY{n}{dict\PYZus{}gender\PYZus{}together} \PY{o}{=} \PY{n+nb}{dict}\PY{p}{(}\PY{n+nb}{zip}\PY{p}{(}\PY{n}{names}\PY{p}{,}\PY{n}{genders}\PY{p}{)}\PY{p}{)}  \PY{c+c1}{\PYZsh{} genero un diccionario nuevo \PYZob{}delfines: generos reordenados\PYZcb{} }
             \PY{n}{AssignGender}\PY{p}{(}\PY{n}{H}\PY{p}{,}\PY{n}{dict\PYZus{}gender\PYZus{}together}\PY{p}{)}             \PY{c+c1}{\PYZsh{} Asigno el genero correspondiente a cada nodo de H }
             \PY{n}{DistributionModularity}\PY{o}{.}\PY{n}{append}\PY{p}{(}\PY{n}{modularity}\PY{p}{(}\PY{n}{H}\PY{p}{,}\PY{l+s+s1}{\PYZsq{}}\PY{l+s+s1}{gender}\PY{l+s+s1}{\PYZsq{}}\PY{p}{)}\PY{p}{)}  \PY{c+c1}{\PYZsh{} Calculo y guardo la modularidad para este grafo}
\end{Verbatim}


    Histograma

    \begin{Verbatim}[commandchars=\\\{\}]
{\color{incolor}In [{\color{incolor}17}]:} \PY{n}{font} \PY{o}{=} \PY{p}{\PYZob{}}\PY{l+s+s1}{\PYZsq{}}\PY{l+s+s1}{family}\PY{l+s+s1}{\PYZsq{}} \PY{p}{:} \PY{l+s+s1}{\PYZsq{}}\PY{l+s+s1}{DejaVu Sans}\PY{l+s+s1}{\PYZsq{}}\PY{p}{,}
                 \PY{l+s+s1}{\PYZsq{}}\PY{l+s+s1}{weight}\PY{l+s+s1}{\PYZsq{}} \PY{p}{:} \PY{l+s+s1}{\PYZsq{}}\PY{l+s+s1}{bold}\PY{l+s+s1}{\PYZsq{}}\PY{p}{,}
                 \PY{l+s+s1}{\PYZsq{}}\PY{l+s+s1}{size}\PY{l+s+s1}{\PYZsq{}}   \PY{p}{:} \PY{l+m+mi}{20}\PY{p}{\PYZcb{}}
         
         \PY{n}{plt}\PY{o}{.}\PY{n}{rc}\PY{p}{(}\PY{l+s+s1}{\PYZsq{}}\PY{l+s+s1}{font}\PY{l+s+s1}{\PYZsq{}}\PY{p}{,} \PY{o}{*}\PY{o}{*}\PY{n}{font}\PY{p}{)}
         
         \PY{n}{plt}\PY{o}{.}\PY{n}{figure}\PY{p}{(}\PY{n}{figsize}\PY{o}{=}\PY{p}{(}\PY{l+m+mi}{15}\PY{p}{,}\PY{l+m+mi}{10}\PY{p}{)}\PY{p}{)}
         \PY{n}{plt}\PY{o}{.}\PY{n}{hist}\PY{p}{(}\PY{n}{DistributionModularity}\PY{p}{,}\PY{n}{bins}\PY{o}{=}\PY{l+m+mi}{20}\PY{p}{)}
         \PY{n}{plt}\PY{o}{.}\PY{n}{axvline}\PY{p}{(}\PY{n}{modularity}\PY{p}{(}\PY{n}{G}\PY{p}{,} \PY{l+s+s1}{\PYZsq{}}\PY{l+s+s1}{gender}\PY{l+s+s1}{\PYZsq{}}\PY{p}{)}\PY{p}{,} \PY{n}{c}\PY{o}{=}\PY{l+s+s2}{\PYZdq{}}\PY{l+s+s2}{red}\PY{l+s+s2}{\PYZdq{}}\PY{p}{)}
         \PY{n}{plt}\PY{o}{.}\PY{n}{plot}\PY{p}{(}\PY{n}{modularity}\PY{p}{(}\PY{n}{G}\PY{p}{,} \PY{l+s+s1}{\PYZsq{}}\PY{l+s+s1}{gender}\PY{l+s+s1}{\PYZsq{}}\PY{p}{)}\PY{p}{,}\PY{l+m+mi}{1}\PY{p}{,}\PY{l+s+s1}{\PYZsq{}}\PY{l+s+s1}{o}\PY{l+s+s1}{\PYZsq{}}\PY{p}{,}\PY{n}{label}\PY{o}{=}\PY{l+s+s1}{\PYZsq{}}\PY{l+s+s1}{modularidad red real}\PY{l+s+s1}{\PYZsq{}}\PY{p}{)}
         \PY{n}{plt}\PY{o}{.}\PY{n}{title}\PY{p}{(}\PY{l+s+s2}{\PYZdq{}}\PY{l+s+s2}{Histograma de la distribución de modularidad }\PY{l+s+se}{\PYZbs{}n}\PY{l+s+s2}{ (}\PY{l+s+si}{\PYZpc{}i}\PY{l+s+s2}{ asignaciones aleatorias de sexo)}\PY{l+s+s2}{\PYZdq{}} \PY{o}{\PYZpc{}}\PY{k}{it})
         \PY{n}{plt}\PY{o}{.}\PY{n}{xlabel}\PY{p}{(}\PY{l+s+s2}{\PYZdq{}}\PY{l+s+s2}{Modularidad}\PY{l+s+s2}{\PYZdq{}}\PY{p}{)}
         \PY{n}{plt}\PY{o}{.}\PY{n}{ylabel}\PY{p}{(}\PY{l+s+s2}{\PYZdq{}}\PY{l+s+s2}{Frecuencia}\PY{l+s+s2}{\PYZdq{}}\PY{p}{)}
         \PY{n}{plt}\PY{o}{.}\PY{n}{legend}\PY{p}{(}\PY{p}{)}
         \PY{n}{show}\PY{p}{(}\PY{p}{)}
\end{Verbatim}


    \begin{center}
    \adjustimage{max size={0.9\linewidth}{0.9\paperheight}}{output_19_0.png}
    \end{center}
    { \hspace*{\fill} \\}
    
    \section{ii. A partir de lo obtenido proponga una estimación para el
valor y el error de dicha cantidad cuando no existe vínculo entre
topolgía de la red medio y asignación de género. Compare su estimación
con el valor medio
esperado.}\label{ii.-a-partir-de-lo-obtenido-proponga-una-estimaciuxf3n-para-el-valor-y-el-error-de-dicha-cantidad-cuando-no-existe-vuxednculo-entre-topolguxeda-de-la-red-medio-y-asignaciuxf3n-de-guxe9nero.-compare-su-estimaciuxf3n-con-el-valor-medio-esperado.}

    \begin{Verbatim}[commandchars=\\\{\}]
{\color{incolor}In [{\color{incolor}18}]:} \PY{n}{np}\PY{o}{.}\PY{n}{mean}\PY{p}{(}\PY{n}{DistributionModularity}\PY{p}{)}\PY{p}{,} \PY{n}{modularity}\PY{p}{(}\PY{n}{G}\PY{p}{,} \PY{l+s+s1}{\PYZsq{}}\PY{l+s+s1}{gender}\PY{l+s+s1}{\PYZsq{}}\PY{p}{)}\PY{p}{,} \PY{n}{np}\PY{o}{.}\PY{n}{std}\PY{p}{(}\PY{n}{DistributionModularity}\PY{p}{)}
\end{Verbatim}


\begin{Verbatim}[commandchars=\\\{\}]
{\color{outcolor}Out[{\color{outcolor}18}]:} (-0.013000395553973473, 0.12370950516197912, 0.03515009700942713)
\end{Verbatim}
            
    Entonces, el valor que sale aleatorio es: \(-0.013 \pm 0.035\) en
unidades arbitrarias Y el valor real es 0.12 que se encuentra fuera del
intervalo del valor medio de la distribucion aleatoria

    \section{(iii) Estime la significancia estadística (p-valor) del valor
observado en el caso de la red
real.}\label{iii-estime-la-significancia-estaduxedstica-p-valor-del-valor-observado-en-el-caso-de-la-red-real.}

    \begin{Verbatim}[commandchars=\\\{\}]
{\color{incolor}In [{\color{incolor}19}]:} \PY{c+c1}{\PYZsh{}apariciones de valores por encima del que presenta nuestro grafo con la atribucion genero}
         \PY{n}{frecuencia}\PY{o}{=}\PY{n+nb}{sum}\PY{p}{(}\PY{n}{DistributionModularity}\PY{o}{\PYZgt{}}\PY{o}{=}\PY{n}{modularity}\PY{p}{(}\PY{n}{G}\PY{p}{,} \PY{l+s+s1}{\PYZsq{}}\PY{l+s+s1}{gender}\PY{l+s+s1}{\PYZsq{}}\PY{p}{)}\PY{p}{)}
\end{Verbatim}


    \begin{Verbatim}[commandchars=\\\{\}]
{\color{incolor}In [{\color{incolor}20}]:} \PY{c+c1}{\PYZsh{} p\PYZhy{}value: area a derecha del valor real en el histograma}
         \PY{n}{pvalue} \PY{o}{=} \PY{p}{(}\PY{n+nb}{max}\PY{p}{(}\PY{n}{DistributionModularity}\PY{p}{)} \PY{o}{\PYZhy{}} \PY{n}{modularity}\PY{p}{(}\PY{n}{G}\PY{p}{,} \PY{l+s+s1}{\PYZsq{}}\PY{l+s+s1}{gender}\PY{l+s+s1}{\PYZsq{}}\PY{p}{)}\PY{p}{)} \PY{o}{*} \PY{p}{(}\PY{n}{frecuencia}\PY{o}{/}\PY{n}{it}\PY{p}{)}
         \PY{n+nb}{print}\PY{p}{(}\PY{l+s+s1}{\PYZsq{}}\PY{l+s+s1}{p\PYZhy{}value:}\PY{l+s+s1}{\PYZsq{}}\PY{p}{,}\PY{n}{pvalue}\PY{p}{)}
         
         \PY{n}{pcrit}\PY{o}{=}\PY{l+m+mi}{1}\PY{o}{/}\PY{l+m+mi}{1000} \PY{c+c1}{\PYZsh{}en general se toma ese valor }
         
         \PY{k}{if} \PY{n}{pvalue} \PY{o}{\PYZlt{}}\PY{o}{=} \PY{p}{(}\PY{n}{pcrit}\PY{p}{)}\PY{p}{:}                                                          
             \PY{n+nb}{print}\PY{p}{(}\PY{l+s+s2}{\PYZdq{}}\PY{l+s+s2}{fuera de la hipotesis nula}\PY{l+s+s2}{\PYZdq{}}\PY{p}{)}
         \PY{k}{else}\PY{p}{:}
             \PY{n+nb}{print} \PY{p}{(}\PY{l+s+s2}{\PYZdq{}}\PY{l+s+s2}{en la hipotesis nula}\PY{l+s+s2}{\PYZdq{}}\PY{p}{)}
\end{Verbatim}


    \begin{Verbatim}[commandchars=\\\{\}]
p-value: -0.0
fuera de la hipotesis nula

    \end{Verbatim}

    


    % Add a bibliography block to the postdoc
    
    
    
    \end{document}
