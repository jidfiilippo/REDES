
% Default to the notebook output style

    


% Inherit from the specified cell style.




    
\documentclass[11pt]{article}

    
    
    \usepackage[T1]{fontenc}
    % Nicer default font (+ math font) than Computer Modern for most use cases
    \usepackage{mathpazo}

    % Basic figure setup, for now with no caption control since it's done
    % automatically by Pandoc (which extracts ![](path) syntax from Markdown).
    \usepackage{graphicx}
    % We will generate all images so they have a width \maxwidth. This means
    % that they will get their normal width if they fit onto the page, but
    % are scaled down if they would overflow the margins.
    \makeatletter
    \def\maxwidth{\ifdim\Gin@nat@width>\linewidth\linewidth
    \else\Gin@nat@width\fi}
    \makeatother
    \let\Oldincludegraphics\includegraphics
    % Set max figure width to be 80% of text width, for now hardcoded.
    \renewcommand{\includegraphics}[1]{\Oldincludegraphics[width=.8\maxwidth]{#1}}
    % Ensure that by default, figures have no caption (until we provide a
    % proper Figure object with a Caption API and a way to capture that
    % in the conversion process - todo).
    \usepackage{caption}
    \DeclareCaptionLabelFormat{nolabel}{}
    \captionsetup{labelformat=nolabel}

    \usepackage{adjustbox} % Used to constrain images to a maximum size 
    \usepackage{xcolor} % Allow colors to be defined
    \usepackage{enumerate} % Needed for markdown enumerations to work
    \usepackage{geometry} % Used to adjust the document margins
    \usepackage{amsmath} % Equations
    \usepackage{amssymb} % Equations
    \usepackage{textcomp} % defines textquotesingle
    % Hack from http://tex.stackexchange.com/a/47451/13684:
    \AtBeginDocument{%
        \def\PYZsq{\textquotesingle}% Upright quotes in Pygmentized code
    }
    \usepackage{upquote} % Upright quotes for verbatim code
    \usepackage{eurosym} % defines \euro
    \usepackage[mathletters]{ucs} % Extended unicode (utf-8) support
    \usepackage[utf8x]{inputenc} % Allow utf-8 characters in the tex document
    \usepackage{fancyvrb} % verbatim replacement that allows latex
    \usepackage{grffile} % extends the file name processing of package graphics 
                         % to support a larger range 
    % The hyperref package gives us a pdf with properly built
    % internal navigation ('pdf bookmarks' for the table of contents,
    % internal cross-reference links, web links for URLs, etc.)
    \usepackage{hyperref}
    \usepackage{longtable} % longtable support required by pandoc >1.10
    \usepackage{booktabs}  % table support for pandoc > 1.12.2
    \usepackage[inline]{enumitem} % IRkernel/repr support (it uses the enumerate* environment)
    \usepackage[normalem]{ulem} % ulem is needed to support strikethroughs (\sout)
                                % normalem makes italics be italics, not underlines
    

    
    
    % Colors for the hyperref package
    \definecolor{urlcolor}{rgb}{0,.145,.698}
    \definecolor{linkcolor}{rgb}{.71,0.21,0.01}
    \definecolor{citecolor}{rgb}{.12,.54,.11}

    % ANSI colors
    \definecolor{ansi-black}{HTML}{3E424D}
    \definecolor{ansi-black-intense}{HTML}{282C36}
    \definecolor{ansi-red}{HTML}{E75C58}
    \definecolor{ansi-red-intense}{HTML}{B22B31}
    \definecolor{ansi-green}{HTML}{00A250}
    \definecolor{ansi-green-intense}{HTML}{007427}
    \definecolor{ansi-yellow}{HTML}{DDB62B}
    \definecolor{ansi-yellow-intense}{HTML}{B27D12}
    \definecolor{ansi-blue}{HTML}{208FFB}
    \definecolor{ansi-blue-intense}{HTML}{0065CA}
    \definecolor{ansi-magenta}{HTML}{D160C4}
    \definecolor{ansi-magenta-intense}{HTML}{A03196}
    \definecolor{ansi-cyan}{HTML}{60C6C8}
    \definecolor{ansi-cyan-intense}{HTML}{258F8F}
    \definecolor{ansi-white}{HTML}{C5C1B4}
    \definecolor{ansi-white-intense}{HTML}{A1A6B2}

    % commands and environments needed by pandoc snippets
    % extracted from the output of `pandoc -s`
    \providecommand{\tightlist}{%
      \setlength{\itemsep}{0pt}\setlength{\parskip}{0pt}}
    \DefineVerbatimEnvironment{Highlighting}{Verbatim}{commandchars=\\\{\}}
    % Add ',fontsize=\small' for more characters per line
    \newenvironment{Shaded}{}{}
    \newcommand{\KeywordTok}[1]{\textcolor[rgb]{0.00,0.44,0.13}{\textbf{{#1}}}}
    \newcommand{\DataTypeTok}[1]{\textcolor[rgb]{0.56,0.13,0.00}{{#1}}}
    \newcommand{\DecValTok}[1]{\textcolor[rgb]{0.25,0.63,0.44}{{#1}}}
    \newcommand{\BaseNTok}[1]{\textcolor[rgb]{0.25,0.63,0.44}{{#1}}}
    \newcommand{\FloatTok}[1]{\textcolor[rgb]{0.25,0.63,0.44}{{#1}}}
    \newcommand{\CharTok}[1]{\textcolor[rgb]{0.25,0.44,0.63}{{#1}}}
    \newcommand{\StringTok}[1]{\textcolor[rgb]{0.25,0.44,0.63}{{#1}}}
    \newcommand{\CommentTok}[1]{\textcolor[rgb]{0.38,0.63,0.69}{\textit{{#1}}}}
    \newcommand{\OtherTok}[1]{\textcolor[rgb]{0.00,0.44,0.13}{{#1}}}
    \newcommand{\AlertTok}[1]{\textcolor[rgb]{1.00,0.00,0.00}{\textbf{{#1}}}}
    \newcommand{\FunctionTok}[1]{\textcolor[rgb]{0.02,0.16,0.49}{{#1}}}
    \newcommand{\RegionMarkerTok}[1]{{#1}}
    \newcommand{\ErrorTok}[1]{\textcolor[rgb]{1.00,0.00,0.00}{\textbf{{#1}}}}
    \newcommand{\NormalTok}[1]{{#1}}
    
    % Additional commands for more recent versions of Pandoc
    \newcommand{\ConstantTok}[1]{\textcolor[rgb]{0.53,0.00,0.00}{{#1}}}
    \newcommand{\SpecialCharTok}[1]{\textcolor[rgb]{0.25,0.44,0.63}{{#1}}}
    \newcommand{\VerbatimStringTok}[1]{\textcolor[rgb]{0.25,0.44,0.63}{{#1}}}
    \newcommand{\SpecialStringTok}[1]{\textcolor[rgb]{0.73,0.40,0.53}{{#1}}}
    \newcommand{\ImportTok}[1]{{#1}}
    \newcommand{\DocumentationTok}[1]{\textcolor[rgb]{0.73,0.13,0.13}{\textit{{#1}}}}
    \newcommand{\AnnotationTok}[1]{\textcolor[rgb]{0.38,0.63,0.69}{\textbf{\textit{{#1}}}}}
    \newcommand{\CommentVarTok}[1]{\textcolor[rgb]{0.38,0.63,0.69}{\textbf{\textit{{#1}}}}}
    \newcommand{\VariableTok}[1]{\textcolor[rgb]{0.10,0.09,0.49}{{#1}}}
    \newcommand{\ControlFlowTok}[1]{\textcolor[rgb]{0.00,0.44,0.13}{\textbf{{#1}}}}
    \newcommand{\OperatorTok}[1]{\textcolor[rgb]{0.40,0.40,0.40}{{#1}}}
    \newcommand{\BuiltInTok}[1]{{#1}}
    \newcommand{\ExtensionTok}[1]{{#1}}
    \newcommand{\PreprocessorTok}[1]{\textcolor[rgb]{0.74,0.48,0.00}{{#1}}}
    \newcommand{\AttributeTok}[1]{\textcolor[rgb]{0.49,0.56,0.16}{{#1}}}
    \newcommand{\InformationTok}[1]{\textcolor[rgb]{0.38,0.63,0.69}{\textbf{\textit{{#1}}}}}
    \newcommand{\WarningTok}[1]{\textcolor[rgb]{0.38,0.63,0.69}{\textbf{\textit{{#1}}}}}
    
    
    % Define a nice break command that doesn't care if a line doesn't already
    % exist.
    \def\br{\hspace*{\fill} \\* }
    % Math Jax compatability definitions
    \def\gt{>}
    \def\lt{<}
    % Document parameters
    \title{EJ1}
    
    
    

    % Pygments definitions
    
\makeatletter
\def\PY@reset{\let\PY@it=\relax \let\PY@bf=\relax%
    \let\PY@ul=\relax \let\PY@tc=\relax%
    \let\PY@bc=\relax \let\PY@ff=\relax}
\def\PY@tok#1{\csname PY@tok@#1\endcsname}
\def\PY@toks#1+{\ifx\relax#1\empty\else%
    \PY@tok{#1}\expandafter\PY@toks\fi}
\def\PY@do#1{\PY@bc{\PY@tc{\PY@ul{%
    \PY@it{\PY@bf{\PY@ff{#1}}}}}}}
\def\PY#1#2{\PY@reset\PY@toks#1+\relax+\PY@do{#2}}

\expandafter\def\csname PY@tok@w\endcsname{\def\PY@tc##1{\textcolor[rgb]{0.73,0.73,0.73}{##1}}}
\expandafter\def\csname PY@tok@c\endcsname{\let\PY@it=\textit\def\PY@tc##1{\textcolor[rgb]{0.25,0.50,0.50}{##1}}}
\expandafter\def\csname PY@tok@cp\endcsname{\def\PY@tc##1{\textcolor[rgb]{0.74,0.48,0.00}{##1}}}
\expandafter\def\csname PY@tok@k\endcsname{\let\PY@bf=\textbf\def\PY@tc##1{\textcolor[rgb]{0.00,0.50,0.00}{##1}}}
\expandafter\def\csname PY@tok@kp\endcsname{\def\PY@tc##1{\textcolor[rgb]{0.00,0.50,0.00}{##1}}}
\expandafter\def\csname PY@tok@kt\endcsname{\def\PY@tc##1{\textcolor[rgb]{0.69,0.00,0.25}{##1}}}
\expandafter\def\csname PY@tok@o\endcsname{\def\PY@tc##1{\textcolor[rgb]{0.40,0.40,0.40}{##1}}}
\expandafter\def\csname PY@tok@ow\endcsname{\let\PY@bf=\textbf\def\PY@tc##1{\textcolor[rgb]{0.67,0.13,1.00}{##1}}}
\expandafter\def\csname PY@tok@nb\endcsname{\def\PY@tc##1{\textcolor[rgb]{0.00,0.50,0.00}{##1}}}
\expandafter\def\csname PY@tok@nf\endcsname{\def\PY@tc##1{\textcolor[rgb]{0.00,0.00,1.00}{##1}}}
\expandafter\def\csname PY@tok@nc\endcsname{\let\PY@bf=\textbf\def\PY@tc##1{\textcolor[rgb]{0.00,0.00,1.00}{##1}}}
\expandafter\def\csname PY@tok@nn\endcsname{\let\PY@bf=\textbf\def\PY@tc##1{\textcolor[rgb]{0.00,0.00,1.00}{##1}}}
\expandafter\def\csname PY@tok@ne\endcsname{\let\PY@bf=\textbf\def\PY@tc##1{\textcolor[rgb]{0.82,0.25,0.23}{##1}}}
\expandafter\def\csname PY@tok@nv\endcsname{\def\PY@tc##1{\textcolor[rgb]{0.10,0.09,0.49}{##1}}}
\expandafter\def\csname PY@tok@no\endcsname{\def\PY@tc##1{\textcolor[rgb]{0.53,0.00,0.00}{##1}}}
\expandafter\def\csname PY@tok@nl\endcsname{\def\PY@tc##1{\textcolor[rgb]{0.63,0.63,0.00}{##1}}}
\expandafter\def\csname PY@tok@ni\endcsname{\let\PY@bf=\textbf\def\PY@tc##1{\textcolor[rgb]{0.60,0.60,0.60}{##1}}}
\expandafter\def\csname PY@tok@na\endcsname{\def\PY@tc##1{\textcolor[rgb]{0.49,0.56,0.16}{##1}}}
\expandafter\def\csname PY@tok@nt\endcsname{\let\PY@bf=\textbf\def\PY@tc##1{\textcolor[rgb]{0.00,0.50,0.00}{##1}}}
\expandafter\def\csname PY@tok@nd\endcsname{\def\PY@tc##1{\textcolor[rgb]{0.67,0.13,1.00}{##1}}}
\expandafter\def\csname PY@tok@s\endcsname{\def\PY@tc##1{\textcolor[rgb]{0.73,0.13,0.13}{##1}}}
\expandafter\def\csname PY@tok@sd\endcsname{\let\PY@it=\textit\def\PY@tc##1{\textcolor[rgb]{0.73,0.13,0.13}{##1}}}
\expandafter\def\csname PY@tok@si\endcsname{\let\PY@bf=\textbf\def\PY@tc##1{\textcolor[rgb]{0.73,0.40,0.53}{##1}}}
\expandafter\def\csname PY@tok@se\endcsname{\let\PY@bf=\textbf\def\PY@tc##1{\textcolor[rgb]{0.73,0.40,0.13}{##1}}}
\expandafter\def\csname PY@tok@sr\endcsname{\def\PY@tc##1{\textcolor[rgb]{0.73,0.40,0.53}{##1}}}
\expandafter\def\csname PY@tok@ss\endcsname{\def\PY@tc##1{\textcolor[rgb]{0.10,0.09,0.49}{##1}}}
\expandafter\def\csname PY@tok@sx\endcsname{\def\PY@tc##1{\textcolor[rgb]{0.00,0.50,0.00}{##1}}}
\expandafter\def\csname PY@tok@m\endcsname{\def\PY@tc##1{\textcolor[rgb]{0.40,0.40,0.40}{##1}}}
\expandafter\def\csname PY@tok@gh\endcsname{\let\PY@bf=\textbf\def\PY@tc##1{\textcolor[rgb]{0.00,0.00,0.50}{##1}}}
\expandafter\def\csname PY@tok@gu\endcsname{\let\PY@bf=\textbf\def\PY@tc##1{\textcolor[rgb]{0.50,0.00,0.50}{##1}}}
\expandafter\def\csname PY@tok@gd\endcsname{\def\PY@tc##1{\textcolor[rgb]{0.63,0.00,0.00}{##1}}}
\expandafter\def\csname PY@tok@gi\endcsname{\def\PY@tc##1{\textcolor[rgb]{0.00,0.63,0.00}{##1}}}
\expandafter\def\csname PY@tok@gr\endcsname{\def\PY@tc##1{\textcolor[rgb]{1.00,0.00,0.00}{##1}}}
\expandafter\def\csname PY@tok@ge\endcsname{\let\PY@it=\textit}
\expandafter\def\csname PY@tok@gs\endcsname{\let\PY@bf=\textbf}
\expandafter\def\csname PY@tok@gp\endcsname{\let\PY@bf=\textbf\def\PY@tc##1{\textcolor[rgb]{0.00,0.00,0.50}{##1}}}
\expandafter\def\csname PY@tok@go\endcsname{\def\PY@tc##1{\textcolor[rgb]{0.53,0.53,0.53}{##1}}}
\expandafter\def\csname PY@tok@gt\endcsname{\def\PY@tc##1{\textcolor[rgb]{0.00,0.27,0.87}{##1}}}
\expandafter\def\csname PY@tok@err\endcsname{\def\PY@bc##1{\setlength{\fboxsep}{0pt}\fcolorbox[rgb]{1.00,0.00,0.00}{1,1,1}{\strut ##1}}}
\expandafter\def\csname PY@tok@kc\endcsname{\let\PY@bf=\textbf\def\PY@tc##1{\textcolor[rgb]{0.00,0.50,0.00}{##1}}}
\expandafter\def\csname PY@tok@kd\endcsname{\let\PY@bf=\textbf\def\PY@tc##1{\textcolor[rgb]{0.00,0.50,0.00}{##1}}}
\expandafter\def\csname PY@tok@kn\endcsname{\let\PY@bf=\textbf\def\PY@tc##1{\textcolor[rgb]{0.00,0.50,0.00}{##1}}}
\expandafter\def\csname PY@tok@kr\endcsname{\let\PY@bf=\textbf\def\PY@tc##1{\textcolor[rgb]{0.00,0.50,0.00}{##1}}}
\expandafter\def\csname PY@tok@bp\endcsname{\def\PY@tc##1{\textcolor[rgb]{0.00,0.50,0.00}{##1}}}
\expandafter\def\csname PY@tok@fm\endcsname{\def\PY@tc##1{\textcolor[rgb]{0.00,0.00,1.00}{##1}}}
\expandafter\def\csname PY@tok@vc\endcsname{\def\PY@tc##1{\textcolor[rgb]{0.10,0.09,0.49}{##1}}}
\expandafter\def\csname PY@tok@vg\endcsname{\def\PY@tc##1{\textcolor[rgb]{0.10,0.09,0.49}{##1}}}
\expandafter\def\csname PY@tok@vi\endcsname{\def\PY@tc##1{\textcolor[rgb]{0.10,0.09,0.49}{##1}}}
\expandafter\def\csname PY@tok@vm\endcsname{\def\PY@tc##1{\textcolor[rgb]{0.10,0.09,0.49}{##1}}}
\expandafter\def\csname PY@tok@sa\endcsname{\def\PY@tc##1{\textcolor[rgb]{0.73,0.13,0.13}{##1}}}
\expandafter\def\csname PY@tok@sb\endcsname{\def\PY@tc##1{\textcolor[rgb]{0.73,0.13,0.13}{##1}}}
\expandafter\def\csname PY@tok@sc\endcsname{\def\PY@tc##1{\textcolor[rgb]{0.73,0.13,0.13}{##1}}}
\expandafter\def\csname PY@tok@dl\endcsname{\def\PY@tc##1{\textcolor[rgb]{0.73,0.13,0.13}{##1}}}
\expandafter\def\csname PY@tok@s2\endcsname{\def\PY@tc##1{\textcolor[rgb]{0.73,0.13,0.13}{##1}}}
\expandafter\def\csname PY@tok@sh\endcsname{\def\PY@tc##1{\textcolor[rgb]{0.73,0.13,0.13}{##1}}}
\expandafter\def\csname PY@tok@s1\endcsname{\def\PY@tc##1{\textcolor[rgb]{0.73,0.13,0.13}{##1}}}
\expandafter\def\csname PY@tok@mb\endcsname{\def\PY@tc##1{\textcolor[rgb]{0.40,0.40,0.40}{##1}}}
\expandafter\def\csname PY@tok@mf\endcsname{\def\PY@tc##1{\textcolor[rgb]{0.40,0.40,0.40}{##1}}}
\expandafter\def\csname PY@tok@mh\endcsname{\def\PY@tc##1{\textcolor[rgb]{0.40,0.40,0.40}{##1}}}
\expandafter\def\csname PY@tok@mi\endcsname{\def\PY@tc##1{\textcolor[rgb]{0.40,0.40,0.40}{##1}}}
\expandafter\def\csname PY@tok@il\endcsname{\def\PY@tc##1{\textcolor[rgb]{0.40,0.40,0.40}{##1}}}
\expandafter\def\csname PY@tok@mo\endcsname{\def\PY@tc##1{\textcolor[rgb]{0.40,0.40,0.40}{##1}}}
\expandafter\def\csname PY@tok@ch\endcsname{\let\PY@it=\textit\def\PY@tc##1{\textcolor[rgb]{0.25,0.50,0.50}{##1}}}
\expandafter\def\csname PY@tok@cm\endcsname{\let\PY@it=\textit\def\PY@tc##1{\textcolor[rgb]{0.25,0.50,0.50}{##1}}}
\expandafter\def\csname PY@tok@cpf\endcsname{\let\PY@it=\textit\def\PY@tc##1{\textcolor[rgb]{0.25,0.50,0.50}{##1}}}
\expandafter\def\csname PY@tok@c1\endcsname{\let\PY@it=\textit\def\PY@tc##1{\textcolor[rgb]{0.25,0.50,0.50}{##1}}}
\expandafter\def\csname PY@tok@cs\endcsname{\let\PY@it=\textit\def\PY@tc##1{\textcolor[rgb]{0.25,0.50,0.50}{##1}}}

\def\PYZbs{\char`\\}
\def\PYZus{\char`\_}
\def\PYZob{\char`\{}
\def\PYZcb{\char`\}}
\def\PYZca{\char`\^}
\def\PYZam{\char`\&}
\def\PYZlt{\char`\<}
\def\PYZgt{\char`\>}
\def\PYZsh{\char`\#}
\def\PYZpc{\char`\%}
\def\PYZdl{\char`\$}
\def\PYZhy{\char`\-}
\def\PYZsq{\char`\'}
\def\PYZdq{\char`\"}
\def\PYZti{\char`\~}
% for compatibility with earlier versions
\def\PYZat{@}
\def\PYZlb{[}
\def\PYZrb{]}
\makeatother


    % Exact colors from NB
    \definecolor{incolor}{rgb}{0.0, 0.0, 0.5}
    \definecolor{outcolor}{rgb}{0.545, 0.0, 0.0}



    
    % Prevent overflowing lines due to hard-to-break entities
    \sloppy 
    % Setup hyperref package
    \hypersetup{
      breaklinks=true,  % so long urls are correctly broken across lines
      colorlinks=true,
      urlcolor=urlcolor,
      linkcolor=linkcolor,
      citecolor=citecolor,
      }
    % Slightly bigger margins than the latex defaults
    
    \geometry{verbose,tmargin=1in,bmargin=1in,lmargin=1in,rmargin=1in}
    
    

    \begin{document}
    
    
    \maketitle
    
    

    
    \section{EJERCICIO 1: INTERACCION DE
PROTEINAS}\label{ejercicio-1-interaccion-de-proteinas}

    Considere las tres redes de interacción de proteínas relevadas para
levadura disponibles en la página de la materia. Se trata de: una red de
interacciones binarias (yeast\_Y2H.txt), de copertenencia a complejos
proteicos (yeast\_AP-MS.txt) y obtenida de literatura (yeast\_LIT.txt)
obtenidas del Yeast Interactome Database.

    \begin{Verbatim}[commandchars=\\\{\}]
{\color{incolor}In [{\color{incolor}29}]:} \PY{c+c1}{\PYZsh{}paquetes }
         \PY{k+kn}{import} \PY{n+nn}{numpy} \PY{k}{as} \PY{n+nn}{np}
         \PY{k+kn}{import} \PY{n+nn}{networkx} \PY{k}{as} \PY{n+nn}{nx}
         \PY{k+kn}{import} \PY{n+nn}{matplotlib}\PY{n+nn}{.}\PY{n+nn}{pylab} \PY{k}{as} \PY{n+nn}{plt}
         \PY{o}{\PYZpc{}}\PY{k}{matplotlib} inline
\end{Verbatim}


    FUNCIÓN PARA ABRIR ARCHIVOS .TXT

    \begin{Verbatim}[commandchars=\\\{\}]
{\color{incolor}In [{\color{incolor}30}]:} \PY{k}{def} \PY{n+nf}{ldata}\PY{p}{(}\PY{n}{archive}\PY{p}{)}\PY{p}{:}
             \PY{n}{f}\PY{o}{=}\PY{n+nb}{open}\PY{p}{(}\PY{n}{archive}\PY{p}{)}
             \PY{n}{data}\PY{o}{=}\PY{p}{[}\PY{p}{]}
             \PY{k}{for} \PY{n}{line} \PY{o+ow}{in} \PY{n}{f}\PY{p}{:}
                 \PY{n}{line}\PY{o}{=}\PY{n}{line}\PY{o}{.}\PY{n}{strip}\PY{p}{(}\PY{p}{)}
                 \PY{n}{col}\PY{o}{=}\PY{n}{line}\PY{o}{.}\PY{n}{split}\PY{p}{(}\PY{p}{)}
                 \PY{n}{data}\PY{o}{.}\PY{n}{append}\PY{p}{(}\PY{n}{col}\PY{p}{)}
             \PY{k}{return} \PY{n}{data}
\end{Verbatim}


    ABRIMOS LAS 3 REDES

    \begin{Verbatim}[commandchars=\\\{\}]
{\color{incolor}In [{\color{incolor}31}]:} \PY{n}{redInteraccionesBinarias} \PY{o}{=} \PY{n}{ldata}\PY{p}{(}\PY{l+s+s1}{\PYZsq{}}\PY{l+s+s1}{yeast\PYZus{}Y2H.txt}\PY{l+s+s1}{\PYZsq{}}\PY{p}{)}
         \PY{n}{redComplejosProteicos} \PY{o}{=} \PY{n}{ldata}\PY{p}{(}\PY{l+s+s1}{\PYZsq{}}\PY{l+s+s1}{yeast\PYZus{}AP\PYZhy{}MS.txt}\PY{l+s+s1}{\PYZsq{}}\PY{p}{)}
         \PY{n}{redLiteratura} \PY{o}{=} \PY{n}{ldata}\PY{p}{(}\PY{l+s+s1}{\PYZsq{}}\PY{l+s+s1}{yeast\PYZus{}LIT.txt}\PY{l+s+s1}{\PYZsq{}}\PY{p}{)}
\end{Verbatim}


    definimos una funcion que nos va a hacer los grafos

    \begin{Verbatim}[commandchars=\\\{\}]
{\color{incolor}In [{\color{incolor}32}]:} \PY{k}{def} \PY{n+nf}{grafo}\PY{p}{(}\PY{n}{datosRed}\PY{p}{)}\PY{p}{:}
             \PY{n}{G} \PY{o}{=} \PY{n}{nx}\PY{o}{.}\PY{n}{Graph}\PY{p}{(}\PY{p}{)}
             \PY{k}{for} \PY{n}{i} \PY{o+ow}{in} \PY{n+nb}{range}\PY{p}{(}\PY{n}{np}\PY{o}{.}\PY{n}{shape}\PY{p}{(}\PY{n}{datosRed}\PY{p}{)}\PY{p}{[}\PY{l+m+mi}{0}\PY{p}{]}\PY{p}{)}\PY{p}{:}
                 \PY{n}{G}\PY{o}{.}\PY{n}{add\PYZus{}edges\PYZus{}from}\PY{p}{(}\PY{p}{[}\PY{p}{(}\PY{n}{datosRed}\PY{p}{[}\PY{n}{i}\PY{p}{]}\PY{p}{[}\PY{l+m+mi}{0}\PY{p}{]}\PY{p}{,}\PY{n}{datosRed}\PY{p}{[}\PY{n}{i}\PY{p}{]}\PY{p}{[}\PY{l+m+mi}{1}\PY{p}{]}\PY{p}{)}\PY{p}{]}\PY{p}{)}
             \PY{k}{return} \PY{n}{G} 
\end{Verbatim}


    \begin{Verbatim}[commandchars=\\\{\}]
{\color{incolor}In [{\color{incolor}33}]:} \PY{n}{grafoRedInteraccionesBinarias} \PY{o}{=} \PY{n}{grafo}\PY{p}{(}\PY{n}{redInteraccionesBinarias}\PY{p}{)}
         \PY{n}{grafoRedComplejosProteicos} \PY{o}{=} \PY{n}{grafo}\PY{p}{(}\PY{n}{redComplejosProteicos}\PY{p}{)}
         \PY{n}{grafoRedLiteratura} \PY{o}{=} \PY{n}{grafo}\PY{p}{(}\PY{n}{redLiteratura}\PY{p}{)}
\end{Verbatim}


    \section{a. Presente una comparación gráfica de las 3
redes.}\label{a.-presente-una-comparaciuxf3n-gruxe1fica-de-las-3-redes.}

    \begin{Verbatim}[commandchars=\\\{\}]
{\color{incolor}In [{\color{incolor}34}]:} \PY{c+c1}{\PYZsh{}ponemos estas opciones para que grafique igual en todos los subplots}
         \PY{n}{options} \PY{o}{=} \PY{p}{\PYZob{}}
              \PY{l+s+s1}{\PYZsq{}}\PY{l+s+s1}{node\PYZus{}color}\PY{l+s+s1}{\PYZsq{}}\PY{p}{:} \PY{l+s+s1}{\PYZsq{}}\PY{l+s+s1}{black}\PY{l+s+s1}{\PYZsq{}}\PY{p}{,}
              \PY{l+s+s1}{\PYZsq{}}\PY{l+s+s1}{node\PYZus{}size}\PY{l+s+s1}{\PYZsq{}}\PY{p}{:} \PY{l+m+mi}{30}\PY{p}{,}
              \PY{l+s+s1}{\PYZsq{}}\PY{l+s+s1}{width}\PY{l+s+s1}{\PYZsq{}}\PY{p}{:} \PY{l+m+mi}{5}\PY{p}{,}
         \PY{p}{\PYZcb{}}
         
         \PY{n}{f} \PY{o}{=} \PY{n}{plt}\PY{o}{.}\PY{n}{figure}\PY{p}{(}\PY{n}{figsize}\PY{o}{=}\PY{p}{(}\PY{l+m+mi}{26}\PY{p}{,}\PY{l+m+mi}{13}\PY{p}{)}\PY{p}{)}                               \PY{c+c1}{\PYZsh{}con esta linea le damos el tamano a cada subplot}
         \PY{n}{f}\PY{o}{.}\PY{n}{suptitle}\PY{p}{(}\PY{l+s+s1}{\PYZsq{}}\PY{l+s+s1}{Tres redes}\PY{l+s+s1}{\PYZsq{}}\PY{p}{,}\PY{n}{fontweight}\PY{o}{=}\PY{l+s+s2}{\PYZdq{}}\PY{l+s+s2}{bold}\PY{l+s+s2}{\PYZdq{}}\PY{p}{,} \PY{n}{size}\PY{o}{=}\PY{l+m+mi}{40}\PY{p}{)} 
         \PY{n}{sub1} \PY{o}{=} \PY{n}{f}\PY{o}{.}\PY{n}{add\PYZus{}subplot}\PY{p}{(}\PY{l+m+mi}{221}\PY{p}{)}
         \PY{n}{nx}\PY{o}{.}\PY{n}{draw}\PY{p}{(}\PY{n}{grafoRedInteraccionesBinarias}\PY{p}{,} \PY{o}{*}\PY{o}{*}\PY{n}{options}\PY{p}{)}
         \PY{n}{sub2} \PY{o}{=} \PY{n}{plt}\PY{o}{.}\PY{n}{subplot}\PY{p}{(}\PY{l+m+mi}{222}\PY{p}{)}
         \PY{n}{nx}\PY{o}{.}\PY{n}{draw}\PY{p}{(}\PY{n}{grafoRedComplejosProteicos}\PY{p}{,} \PY{o}{*}\PY{o}{*}\PY{n}{options}\PY{p}{)}
         \PY{n}{sub3} \PY{o}{=} \PY{n}{plt}\PY{o}{.}\PY{n}{subplot}\PY{p}{(}\PY{l+m+mi}{223}\PY{p}{)}
         \PY{n}{nx}\PY{o}{.}\PY{n}{draw}\PY{p}{(}\PY{n}{grafoRedLiteratura}\PY{p}{,} \PY{o}{*}\PY{o}{*}\PY{n}{options}\PY{p}{)}
         \PY{n}{sub1}\PY{o}{.}\PY{n}{set\PYZus{}title}\PY{p}{(}\PY{l+s+s1}{\PYZsq{}}\PY{l+s+s1}{Red de Interacciones Binarias}\PY{l+s+s1}{\PYZsq{}}\PY{p}{,} \PY{n}{size}\PY{o}{=}\PY{l+m+mi}{30}\PY{p}{)}
         \PY{n}{sub2}\PY{o}{.}\PY{n}{set\PYZus{}title}\PY{p}{(}\PY{l+s+s1}{\PYZsq{}}\PY{l+s+s1}{Red de Complejos Proteicos}\PY{l+s+s1}{\PYZsq{}}\PY{p}{,} \PY{n}{size}\PY{o}{=}\PY{l+m+mi}{30}\PY{p}{)}
         \PY{n}{sub3}\PY{o}{.}\PY{n}{set\PYZus{}title}\PY{p}{(}\PY{l+s+s1}{\PYZsq{}}\PY{l+s+s1}{Red de la Literatura}\PY{l+s+s1}{\PYZsq{}}\PY{p}{,} \PY{n}{size}\PY{o}{=}\PY{l+m+mi}{30}\PY{p}{)}
         \PY{n}{plt}\PY{o}{.}\PY{n}{show}\PY{p}{(}\PY{p}{)}
\end{Verbatim}


    \begin{center}
    \adjustimage{max size={0.9\linewidth}{0.9\paperheight}}{output_11_0.png}
    \end{center}
    { \hspace*{\fill} \\}
    
    \section{NUMERO DE NODOS}\label{numero-de-nodos}

    \begin{Verbatim}[commandchars=\\\{\}]
{\color{incolor}In [{\color{incolor}35}]:} \PY{n}{nodosInteraccionesBinarias} \PY{o}{=} \PY{n}{grafoRedInteraccionesBinarias}\PY{o}{.}\PY{n}{number\PYZus{}of\PYZus{}nodes}\PY{p}{(}\PY{p}{)}
         \PY{n}{nodosComplejosProteicos} \PY{o}{=} \PY{n}{grafoRedComplejosProteicos}\PY{o}{.}\PY{n}{number\PYZus{}of\PYZus{}nodes}\PY{p}{(}\PY{p}{)}
         \PY{n}{nodosRedLiteratura} \PY{o}{=} \PY{n}{grafoRedLiteratura}\PY{o}{.}\PY{n}{number\PYZus{}of\PYZus{}nodes}\PY{p}{(}\PY{p}{)}
         
         \PY{n+nb}{print} \PY{p}{(}\PY{n}{nodosInteraccionesBinarias}\PY{p}{,} \PY{n}{nodosComplejosProteicos}\PY{p}{,} \PY{n}{nodosRedLiteratura}\PY{p}{)}
\end{Verbatim}


    \begin{Verbatim}[commandchars=\\\{\}]
2018 1622 1536

    \end{Verbatim}

    \section{ENLACES}\label{enlaces}

    \begin{Verbatim}[commandchars=\\\{\}]
{\color{incolor}In [{\color{incolor}36}]:} \PY{n}{enlacesInteraccionesBinarias} \PY{o}{=} \PY{n}{grafoRedInteraccionesBinarias}\PY{o}{.}\PY{n}{number\PYZus{}of\PYZus{}edges}\PY{p}{(}\PY{p}{)}
         \PY{n}{enlacesComplejosProteicos} \PY{o}{=} \PY{n}{grafoRedComplejosProteicos}\PY{o}{.}\PY{n}{number\PYZus{}of\PYZus{}edges}\PY{p}{(}\PY{p}{)}
         \PY{n}{enlacesRedLiteratura} \PY{o}{=} \PY{n}{grafoRedLiteratura}\PY{o}{.}\PY{n}{number\PYZus{}of\PYZus{}edges}\PY{p}{(}\PY{p}{)}
         
         \PY{n+nb}{print} \PY{p}{(}\PY{n}{enlacesInteraccionesBinarias}\PY{p}{,} \PY{n}{enlacesComplejosProteicos}\PY{p}{,} \PY{n}{enlacesRedLiteratura}\PY{p}{)}
\end{Verbatim}


    \begin{Verbatim}[commandchars=\\\{\}]
2930 9070 2925

    \end{Verbatim}

    DIRIGIDA O NO DIRIGIDA: ESTO SE INTERPRETA POR CONOCIMIENTO DEL PROCESO
POR EJEMPLO: EN REDES SOCIALES COMO TWITTER O INSTAGRAM PUEDO SEGUIR A
ALGUIEN Y QUE ESA PERSONA NO ME SIGA, ENTONCES ES DIRIGIDO. EN CASOS DE
PROTEINAS ES NO DIRIGIDO PORQUE AMBAS INTERACTUAN ENTRE ELLAS.

    \section{ALGUNOS PARAMETROS DE LAS
REDES}\label{algunos-parametros-de-las-redes}

    \begin{Verbatim}[commandchars=\\\{\}]
{\color{incolor}In [{\color{incolor}37}]:} \PY{n}{G}\PY{o}{=}\PY{n}{grafoRedComplejosProteicos}
         \PY{n}{H}\PY{o}{=}\PY{n}{grafoRedInteraccionesBinarias}
         \PY{n}{I}\PY{o}{=}\PY{n}{grafoRedLiteratura}
\end{Verbatim}


    como hay dos comunidades no conectadas no puede definir una longitud
entonces.. definimos la siguiente funcion para obtener la componente
gigante y ahi calcularle el diametro y la densidad.

    \begin{Verbatim}[commandchars=\\\{\}]
{\color{incolor}In [{\color{incolor}38}]:} \PY{k}{def} \PY{n+nf}{get\PYZus{}giant}\PY{p}{(}\PY{n}{G}\PY{p}{)}\PY{p}{:}
             \PY{n}{Gcc}\PY{o}{=}\PY{n+nb}{sorted}\PY{p}{(}\PY{n}{nx}\PY{o}{.}\PY{n}{connected\PYZus{}component\PYZus{}subgraphs}\PY{p}{(}\PY{n}{G}\PY{p}{)}\PY{p}{,} \PY{n}{key} \PY{o}{=} \PY{n+nb}{len}\PY{p}{,} \PY{n}{reverse}\PY{o}{=}\PY{k+kc}{True}\PY{p}{)}
             \PY{n}{G0}\PY{o}{=}\PY{n}{Gcc}\PY{p}{[}\PY{l+m+mi}{0}\PY{p}{]}
             \PY{k}{return}\PY{p}{(}\PY{n}{G0}\PY{p}{)}
\end{Verbatim}


    \begin{Verbatim}[commandchars=\\\{\}]
{\color{incolor}In [{\color{incolor}39}]:} \PY{k}{def} \PY{n+nf}{props}\PY{p}{(}\PY{n}{G}\PY{p}{)}\PY{p}{:}
             \PY{n}{giant\PYZus{}G}\PY{o}{=}\PY{n}{get\PYZus{}giant}\PY{p}{(}\PY{n}{G}\PY{p}{)}
             \PY{k}{return}\PY{p}{(}\PY{l+s+s2}{\PYZdq{}}\PY{l+s+s2}{diameter: }\PY{l+s+si}{\PYZpc{}d}\PY{l+s+s2}{\PYZdq{}} \PY{o}{\PYZpc{}} \PY{n}{nx}\PY{o}{.}\PY{n}{diameter}\PY{p}{(}\PY{n}{giant\PYZus{}G}\PY{p}{)}\PY{p}{,}\PY{l+s+s1}{\PYZsq{}}\PY{l+s+s1}{density}\PY{l+s+s1}{\PYZsq{}}\PY{p}{,}\PY{n}{nx}\PY{o}{.}\PY{n}{density}\PY{p}{(}\PY{n}{G}\PY{p}{)}\PY{p}{)}
\end{Verbatim}


    \begin{Verbatim}[commandchars=\\\{\}]
{\color{incolor}In [{\color{incolor}40}]:} \PY{n}{props}\PY{p}{(}\PY{n}{G}\PY{p}{)}\PY{p}{,}\PY{n}{props}\PY{p}{(}\PY{n}{H}\PY{p}{)}\PY{p}{,}\PY{n}{props}\PY{p}{(}\PY{n}{I}\PY{p}{)}
\end{Verbatim}


\begin{Verbatim}[commandchars=\\\{\}]
{\color{outcolor}Out[{\color{outcolor}40}]:} (('diameter: 15', 'density', 0.006899274397150227),
          ('diameter: 14', 'density', 0.0014396951973635397),
          ('diameter: 19', 'density', 0.002481168566775244))
\end{Verbatim}
            
    \section{CLUSTERING}\label{clustering}

    LOCAL

    \begin{Verbatim}[commandchars=\\\{\}]
{\color{incolor}In [{\color{incolor}41}]:} \PY{n}{nx}\PY{o}{.}\PY{n}{average\PYZus{}clustering}\PY{p}{(}\PY{n}{G}\PY{p}{)}\PY{p}{,}\PY{n}{nx}\PY{o}{.}\PY{n}{average\PYZus{}clustering}\PY{p}{(}\PY{n}{H}\PY{p}{)}\PY{p}{,}\PY{n}{nx}\PY{o}{.}\PY{n}{average\PYZus{}clustering}\PY{p}{(}\PY{n}{I}\PY{p}{)}
\end{Verbatim}


\begin{Verbatim}[commandchars=\\\{\}]
{\color{outcolor}Out[{\color{outcolor}41}]:} (0.5546360657013013, 0.046194001297365166, 0.2924923005815711)
\end{Verbatim}
            
    GLOBAL

    \begin{Verbatim}[commandchars=\\\{\}]
{\color{incolor}In [{\color{incolor}42}]:} \PY{n}{nx}\PY{o}{.}\PY{n}{transitivity}\PY{p}{(}\PY{n}{G}\PY{p}{)}\PY{p}{,}\PY{n}{nx}\PY{o}{.}\PY{n}{transitivity}\PY{p}{(}\PY{n}{H}\PY{p}{)}\PY{p}{,}\PY{n}{nx}\PY{o}{.}\PY{n}{transitivity}\PY{p}{(}\PY{n}{I}\PY{p}{)}
\end{Verbatim}


\begin{Verbatim}[commandchars=\\\{\}]
{\color{outcolor}Out[{\color{outcolor}42}]:} (0.6185901626483971, 0.02361415364051535, 0.3461926495315878)
\end{Verbatim}
            
    \section{MEAN DEGREE}\label{mean-degree}

    \begin{Verbatim}[commandchars=\\\{\}]
{\color{incolor}In [{\color{incolor}43}]:} \PY{c+c1}{\PYZsh{}forma rapida y linda}
         \PY{n}{maxdg\PYZus{}G}\PY{o}{=}\PY{n}{np}\PY{o}{.}\PY{n}{array}\PY{p}{(}\PY{p}{[}\PY{n}{j} \PY{k}{for} \PY{p}{(}\PY{n}{i}\PY{p}{,}\PY{n}{j}\PY{p}{)} \PY{o+ow}{in} \PY{n}{G}\PY{o}{.}\PY{n}{degree}\PY{p}{]}\PY{p}{)}\PY{o}{.}\PY{n}{max}\PY{p}{(}\PY{p}{)}
         \PY{n}{maxdg\PYZus{}H}\PY{o}{=}\PY{n}{np}\PY{o}{.}\PY{n}{array}\PY{p}{(}\PY{p}{[}\PY{n}{j} \PY{k}{for} \PY{p}{(}\PY{n}{i}\PY{p}{,}\PY{n}{j}\PY{p}{)} \PY{o+ow}{in} \PY{n}{H}\PY{o}{.}\PY{n}{degree}\PY{p}{]}\PY{p}{)}\PY{o}{.}\PY{n}{max}\PY{p}{(}\PY{p}{)}
         \PY{n}{maxdg\PYZus{}I}\PY{o}{=}\PY{n}{np}\PY{o}{.}\PY{n}{array}\PY{p}{(}\PY{p}{[}\PY{n}{j} \PY{k}{for} \PY{p}{(}\PY{n}{i}\PY{p}{,}\PY{n}{j}\PY{p}{)} \PY{o+ow}{in} \PY{n}{I}\PY{o}{.}\PY{n}{degree}\PY{p}{]}\PY{p}{)}\PY{o}{.}\PY{n}{max}\PY{p}{(}\PY{p}{)}
         
         \PY{n}{maxdg\PYZus{}G}\PY{p}{,}\PY{n}{maxdg\PYZus{}H}\PY{p}{,}\PY{n}{maxdg\PYZus{}I}
\end{Verbatim}


\begin{Verbatim}[commandchars=\\\{\}]
{\color{outcolor}Out[{\color{outcolor}43}]:} (127, 91, 40)
\end{Verbatim}
            
    \begin{Verbatim}[commandchars=\\\{\}]
{\color{incolor}In [{\color{incolor}44}]:} \PY{n}{mindg\PYZus{}G}\PY{o}{=}\PY{n}{np}\PY{o}{.}\PY{n}{array}\PY{p}{(}\PY{p}{[}\PY{n}{j} \PY{k}{for} \PY{p}{(}\PY{n}{i}\PY{p}{,}\PY{n}{j}\PY{p}{)} \PY{o+ow}{in} \PY{n}{G}\PY{o}{.}\PY{n}{degree}\PY{p}{]}\PY{p}{)}\PY{o}{.}\PY{n}{min}\PY{p}{(}\PY{p}{)}
         \PY{n}{mindg\PYZus{}H}\PY{o}{=}\PY{n}{np}\PY{o}{.}\PY{n}{array}\PY{p}{(}\PY{p}{[}\PY{n}{j} \PY{k}{for} \PY{p}{(}\PY{n}{i}\PY{p}{,}\PY{n}{j}\PY{p}{)} \PY{o+ow}{in} \PY{n}{H}\PY{o}{.}\PY{n}{degree}\PY{p}{]}\PY{p}{)}\PY{o}{.}\PY{n}{min}\PY{p}{(}\PY{p}{)}
         \PY{n}{mindg\PYZus{}I}\PY{o}{=}\PY{n}{np}\PY{o}{.}\PY{n}{array}\PY{p}{(}\PY{p}{[}\PY{n}{j} \PY{k}{for} \PY{p}{(}\PY{n}{i}\PY{p}{,}\PY{n}{j}\PY{p}{)} \PY{o+ow}{in} \PY{n}{I}\PY{o}{.}\PY{n}{degree}\PY{p}{]}\PY{p}{)}\PY{o}{.}\PY{n}{min}\PY{p}{(}\PY{p}{)}
         
         \PY{n}{mindg\PYZus{}G}\PY{p}{,}\PY{n}{mindg\PYZus{}H}\PY{p}{,}\PY{n}{mindg\PYZus{}I}
\end{Verbatim}


\begin{Verbatim}[commandchars=\\\{\}]
{\color{outcolor}Out[{\color{outcolor}44}]:} (1, 1, 1)
\end{Verbatim}
            
    \begin{Verbatim}[commandchars=\\\{\}]
{\color{incolor}In [{\color{incolor}45}]:} \PY{n}{mv\PYZus{}G}\PY{o}{=}\PY{n}{np}\PY{o}{.}\PY{n}{mean}\PY{p}{(}\PY{n}{np}\PY{o}{.}\PY{n}{array}\PY{p}{(}\PY{p}{[}\PY{n}{j} \PY{k}{for} \PY{p}{(}\PY{n}{i}\PY{p}{,}\PY{n}{j}\PY{p}{)} \PY{o+ow}{in} \PY{n}{G}\PY{o}{.}\PY{n}{degree}\PY{p}{]}\PY{p}{)}\PY{p}{)}
         \PY{n}{mv\PYZus{}H}\PY{o}{=}\PY{n}{np}\PY{o}{.}\PY{n}{mean}\PY{p}{(}\PY{n}{np}\PY{o}{.}\PY{n}{array}\PY{p}{(}\PY{p}{[}\PY{n}{j} \PY{k}{for} \PY{p}{(}\PY{n}{i}\PY{p}{,}\PY{n}{j}\PY{p}{)} \PY{o+ow}{in} \PY{n}{H}\PY{o}{.}\PY{n}{degree}\PY{p}{]}\PY{p}{)}\PY{p}{)}
         \PY{n}{mv\PYZus{}I}\PY{o}{=}\PY{n}{np}\PY{o}{.}\PY{n}{mean}\PY{p}{(}\PY{n}{np}\PY{o}{.}\PY{n}{array}\PY{p}{(}\PY{p}{[}\PY{n}{j} \PY{k}{for} \PY{p}{(}\PY{n}{i}\PY{p}{,}\PY{n}{j}\PY{p}{)} \PY{o+ow}{in} \PY{n}{I}\PY{o}{.}\PY{n}{degree}\PY{p}{]}\PY{p}{)}\PY{p}{)}
         
         \PY{n}{mv\PYZus{}G}\PY{p}{,}\PY{n}{mv\PYZus{}H}\PY{p}{,}\PY{n}{mv\PYZus{}I}
\end{Verbatim}


\begin{Verbatim}[commandchars=\\\{\}]
{\color{outcolor}Out[{\color{outcolor}45}]:} (11.183723797780518, 2.9038652130822595, 3.80859375)
\end{Verbatim}
            
    \section{RED DIRIGIDA O NO DIRIGIDA}\label{red-dirigida-o-no-dirigida}

    \begin{Verbatim}[commandchars=\\\{\}]
{\color{incolor}In [{\color{incolor}46}]:} \PY{c+c1}{\PYZsh{}PARA VER SI ERA RED DIRIGIDA}
         \PY{c+c1}{\PYZsh{}la idea es que si hay interseccion entre set1 y set2 es que se da el enlace nodo1\PYZhy{}nodo2 y nodo2\PYZhy{}nodo1 entonces uno interpreta que }
         \PY{c+c1}{\PYZsh{}NO se hace distincion y que el nodo1 interactua con el nodo2 y al reves entonces.. es no dirigida. En cambio, si no hay intersecc}
         \PY{c+c1}{\PYZsh{}es claramente dirigida. Y la no intersecc es si el array inters tiene length 0. }
         
         \PY{k}{def} \PY{n+nf}{direccion}\PY{p}{(}\PY{n}{G}\PY{p}{)}\PY{p}{:}
             \PY{n}{set1}\PY{o}{=}\PY{n+nb}{set}\PY{p}{(}\PY{n}{G}\PY{o}{.}\PY{n}{edges}\PY{p}{)}
             \PY{n}{set2} \PY{o}{=} \PY{p}{\PYZob{}}\PY{p}{(}\PY{n}{nombre2}\PY{p}{,}\PY{n}{nombre1}\PY{p}{)} \PY{k}{for} \PY{n}{nombre1}\PY{p}{,} \PY{n}{nombre2}  \PY{o+ow}{in} \PY{n}{set1}\PY{p}{\PYZcb{}}
             \PY{n}{inters} \PY{o}{=} \PY{n}{set1}\PY{o}{.}\PY{n}{intersection}\PY{p}{(}\PY{n}{set2}\PY{p}{)}
             \PY{k}{if} \PY{n+nb}{len}\PY{p}{(}\PY{n}{inters}\PY{p}{)} \PY{o}{==} \PY{l+m+mi}{0}\PY{p}{:}
                 \PY{n+nb}{print}\PY{p}{(}\PY{l+s+s2}{\PYZdq{}}\PY{l+s+s2}{Dirigida}\PY{l+s+s2}{\PYZdq{}}\PY{p}{)}
             \PY{k}{else}\PY{p}{:}
                 \PY{n+nb}{print}\PY{p}{(}\PY{l+s+s2}{\PYZdq{}}\PY{l+s+s2}{No dirigida}\PY{l+s+s2}{\PYZdq{}}\PY{p}{)}
\end{Verbatim}


    \begin{Verbatim}[commandchars=\\\{\}]
{\color{incolor}In [{\color{incolor}47}]:} \PY{n}{direccion}\PY{p}{(}\PY{n}{G}\PY{p}{)}
         \PY{n}{direccion}\PY{p}{(}\PY{n}{H}\PY{p}{)}
         \PY{n}{direccion}\PY{p}{(}\PY{n}{I}\PY{p}{)}
\end{Verbatim}


    \begin{Verbatim}[commandchars=\\\{\}]
Dirigida
No dirigida
No dirigida

    \end{Verbatim}

    \subsection{COMENTARIOS para el grupo}\label{comentarios-para-el-grupo}

    \begin{Verbatim}[commandchars=\\\{\}]
{\color{incolor}In [{\color{incolor}48}]:} \PY{p}{[}\PY{p}{(}\PY{n}{i}\PY{p}{,}\PY{n}{j}\PY{p}{)} \PY{k}{for} \PY{p}{(}\PY{n}{i}\PY{p}{,}\PY{n}{j}\PY{p}{)} \PY{o+ow}{in} \PY{n}{H}\PY{o}{.}\PY{n}{degree}\PY{p}{]} \PY{c+c1}{\PYZsh{} este me crea una lista con las truplas que le digo}
         \PY{c+c1}{\PYZsh{}como me quiero quedar solo con los degrees, me conviene pedirle los j\PYZsq{}s:}
         \PY{p}{[}\PY{p}{(}\PY{n}{j}\PY{p}{)} \PY{k}{for} \PY{p}{(}\PY{n}{i}\PY{p}{,}\PY{n}{j}\PY{p}{)} \PY{o+ow}{in} \PY{n}{H}\PY{o}{.}\PY{n}{degree}\PY{p}{]}
         \PY{c+c1}{\PYZsh{} lo paso a array y despues le pido el maximo}
         \PY{n}{asd}\PY{o}{=}\PY{n}{np}\PY{o}{.}\PY{n}{array}\PY{p}{(}\PY{p}{[}\PY{p}{(}\PY{n}{j}\PY{p}{)} \PY{k}{for} \PY{p}{(}\PY{n}{i}\PY{p}{,}\PY{n}{j}\PY{p}{)} \PY{o+ow}{in} \PY{n}{H}\PY{o}{.}\PY{n}{degree}\PY{p}{]}\PY{p}{)}
         \PY{n}{asd}\PY{o}{.}\PY{n}{max}\PY{p}{(}\PY{p}{)}
\end{Verbatim}


\begin{Verbatim}[commandchars=\\\{\}]
{\color{outcolor}Out[{\color{outcolor}48}]:} 91
\end{Verbatim}
            
    \section{TABLA CON RESULTADOS}\label{tabla-con-resultados}

    \begin{Verbatim}[commandchars=\\\{\}]
{\color{incolor}In [{\color{incolor}2}]:}  \PY{k+kn}{from} \PY{n+nn}{IPython}\PY{n+nn}{.}\PY{n+nn}{display} \PY{k}{import} \PY{n}{HTML}\PY{p}{,} \PY{n}{display}
        
         \PY{n}{data} \PY{o}{=} \PY{p}{[}\PY{p}{[}\PY{l+s+s2}{\PYZdq{}}\PY{l+s+s2}{\PYZdq{}}\PY{p}{,}\PY{l+s+s2}{\PYZdq{}}\PY{l+s+s2}{Red de compl proteicos}\PY{l+s+s2}{\PYZdq{}}\PY{p}{,}\PY{l+s+s2}{\PYZdq{}}\PY{l+s+s2}{Red de int binarias}\PY{l+s+s2}{\PYZdq{}}\PY{p}{,}\PY{l+s+s2}{\PYZdq{}}\PY{l+s+s2}{Red de literatura}\PY{l+s+s2}{\PYZdq{}}\PY{p}{]}\PY{p}{,}
                 \PY{p}{[}\PY{l+s+s2}{\PYZdq{}}\PY{l+s+s2}{Nodos}\PY{l+s+s2}{\PYZdq{}}\PY{p}{,}\PY{l+m+mi}{2018}\PY{p}{,} \PY{l+m+mi}{1622}\PY{p}{,} \PY{l+m+mi}{1536}\PY{p}{]}\PY{p}{,}
                 \PY{p}{[}\PY{l+s+s2}{\PYZdq{}}\PY{l+s+s2}{Total enlaces}\PY{l+s+s2}{\PYZdq{}}\PY{p}{,}\PY{l+m+mi}{2930}\PY{p}{,} \PY{l+m+mi}{9070}\PY{p}{,} \PY{l+m+mi}{2925}\PY{p}{]}\PY{p}{,}
                 \PY{p}{[}\PY{l+s+s2}{\PYZdq{}}\PY{l+s+s2}{Dirigida}\PY{l+s+s2}{\PYZdq{}}\PY{p}{,}\PY{l+s+s2}{\PYZdq{}}\PY{l+s+s2}{SI}\PY{l+s+s2}{\PYZdq{}}\PY{p}{,}\PY{l+s+s2}{\PYZdq{}}\PY{l+s+s2}{NO}\PY{l+s+s2}{\PYZdq{}}\PY{p}{,}\PY{l+s+s2}{\PYZdq{}}\PY{l+s+s2}{NO}\PY{l+s+s2}{\PYZdq{}}\PY{p}{]}\PY{p}{,}
                 \PY{p}{[}\PY{l+s+s2}{\PYZdq{}}\PY{l+s+s2}{Grado medio}\PY{l+s+s2}{\PYZdq{}}\PY{p}{,}\PY{l+m+mf}{11.18}\PY{p}{,} \PY{l+m+mf}{2.90}\PY{p}{,} \PY{l+m+mf}{3.81}\PY{p}{]}\PY{p}{,}
                 \PY{p}{[}\PY{l+s+s2}{\PYZdq{}}\PY{l+s+s2}{Grado max}\PY{l+s+s2}{\PYZdq{}}\PY{p}{,}\PY{l+m+mi}{127}\PY{p}{,} \PY{l+m+mi}{91}\PY{p}{,} \PY{l+m+mi}{40}\PY{p}{]}\PY{p}{,}
                 \PY{p}{[}\PY{l+s+s2}{\PYZdq{}}\PY{l+s+s2}{Grado min}\PY{l+s+s2}{\PYZdq{}}\PY{p}{,}\PY{l+m+mi}{1}\PY{p}{,}\PY{l+m+mi}{1}\PY{p}{,}\PY{l+m+mi}{1}\PY{p}{]}\PY{p}{,}
                 \PY{p}{[}\PY{l+s+s2}{\PYZdq{}}\PY{l+s+s2}{Densidad}\PY{l+s+s2}{\PYZdq{}}\PY{p}{,}\PY{l+m+mf}{0.00690}\PY{p}{,} \PY{l+m+mf}{0.00144}\PY{p}{,} \PY{l+m+mf}{0.00248}\PY{p}{]}\PY{p}{,}
                 \PY{p}{[}\PY{l+s+s2}{\PYZdq{}}\PY{l+s+s2}{Coef. clustering local}\PY{l+s+s2}{\PYZdq{}}\PY{p}{,}\PY{l+m+mf}{0.555}\PY{p}{,} \PY{l+m+mf}{0.046}\PY{p}{,} \PY{l+m+mf}{0.292}\PY{p}{]}\PY{p}{,}
                 \PY{p}{[}\PY{l+s+s2}{\PYZdq{}}\PY{l+s+s2}{Coef. clustering global}\PY{l+s+s2}{\PYZdq{}}\PY{p}{,}\PY{l+m+mf}{0.619}\PY{p}{,} \PY{l+m+mf}{0.024}\PY{p}{,} \PY{l+m+mf}{0.346}\PY{p}{]}\PY{p}{,}
                 \PY{p}{[}\PY{l+s+s2}{\PYZdq{}}\PY{l+s+s2}{Diametro}\PY{l+s+s2}{\PYZdq{}}\PY{p}{,}\PY{l+m+mi}{15}\PY{p}{,}\PY{l+m+mi}{14}\PY{p}{,}\PY{l+m+mi}{19}\PY{p}{]}
                 \PY{p}{]}
        
         \PY{n}{display}\PY{p}{(}\PY{n}{HTML}\PY{p}{(}
            \PY{l+s+s1}{\PYZsq{}}\PY{l+s+s1}{\PYZlt{}table\PYZgt{}\PYZlt{}tr\PYZgt{}}\PY{l+s+si}{\PYZob{}\PYZcb{}}\PY{l+s+s1}{\PYZlt{}/tr\PYZgt{}\PYZlt{}/table\PYZgt{}}\PY{l+s+s1}{\PYZsq{}}\PY{o}{.}\PY{n}{format}\PY{p}{(}
                \PY{l+s+s1}{\PYZsq{}}\PY{l+s+s1}{\PYZlt{}/tr\PYZgt{}\PYZlt{}tr\PYZgt{}}\PY{l+s+s1}{\PYZsq{}}\PY{o}{.}\PY{n}{join}\PY{p}{(}
                    \PY{l+s+s1}{\PYZsq{}}\PY{l+s+s1}{\PYZlt{}td\PYZgt{}}\PY{l+s+si}{\PYZob{}\PYZcb{}}\PY{l+s+s1}{\PYZlt{}/td\PYZgt{}}\PY{l+s+s1}{\PYZsq{}}\PY{o}{.}\PY{n}{format}\PY{p}{(}\PY{l+s+s1}{\PYZsq{}}\PY{l+s+s1}{\PYZlt{}/td\PYZgt{}\PYZlt{}td\PYZgt{}}\PY{l+s+s1}{\PYZsq{}}\PY{o}{.}\PY{n}{join}\PY{p}{(}\PY{n+nb}{str}\PY{p}{(}\PY{n}{\PYZus{}}\PY{p}{)} \PY{k}{for} \PY{n}{\PYZus{}} \PY{o+ow}{in} \PY{n}{row}\PY{p}{)}\PY{p}{)} \PY{k}{for} \PY{n}{row} \PY{o+ow}{in} \PY{n}{data}\PY{p}{)}
                \PY{p}{)}
         \PY{p}{)}\PY{p}{)}
\end{Verbatim}


    
    \begin{verbatim}
<IPython.core.display.HTML object>
    \end{verbatim}

    

    % Add a bibliography block to the postdoc
    
    
    
    \end{document}
